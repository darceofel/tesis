\chapter{¿Qué tan grande es \texorpdfstring{$\mathbf{\mathcal{ND}[0,1]}$}{ND[0,1]}?}
Las matemáticas nacieron a partir de un impulso (¿o necesidad?)  por medir las cosas: medir lados, áreas, volúmenes o cantidades, era el trabajo de todos los días de los primeros matemáticos. 
Este impulso primitivo está tan presente en las matemáticas ahora como nunca. Tal es así que, en este capítulo, teniendo demostrado ya que nuestro espacio de interés no es vacío, la siguiente pregunta que atacaremos es: ¿qué tan grande es este espacio? 

Uno podría pensar que la vaguedad de la pregunta da entrada a respuestas de un carácter más subjetivo, pero no es así: la naturaleza de la pregunta no nos impone ninguna restricción para responderla de manera rigurosa, existen varias formas de definirla y abordarla precisamente. 

En primera instancia podemos hablar de su cardinalidad, y, usando el \cref{thm:implica_densidad_lynch} o la función de Weierstrass, podemos demostrar que este espacio es, como mínimo, infinito\footnote{Sin embargo, la cardinalidad no es la mejor de las respuestas a este tipo de preguntas: consideremos, por ejemplo, a los naturales, $\mathbb{N}$, como subconjunto de $\mathbb{R}$; este conjunto es infinito, pero sería difícil defender que es un subconjunto \textit{grande} de $\mathbb{R}$.}. Por otro lado, existen algunas nociones topológicas que nos pueden acercar un poco a una respuesta más sensible: entre ellas destacan la densidad y las categorías de Baire (ambas abordadas en este capítulo). Por último, y más importante para el enfoque de esta tesis, está el estudio desde la teoría de la medida; es este el camino que nos puede llevar a resultados de la forma: 
\begin{center}
    ``Casi todo elemento de un conjunto $A$ está en un conjunto $B$''
\end{center}
en espacios arbitrarios. Es decir, esquivando todos los obstáculos teóricos que traen los espacios dimensionalmente infinitos, la teoría de la medida nos ayudará a darle un enfoque probabilístico al estudio del tamaño de los subconjuntos de éstos.