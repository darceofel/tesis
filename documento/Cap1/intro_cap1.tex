\chapter{Ejemplos de funciones en \texorpdfstring{$\mathbf{\mathcal{ND}[0,1]}$}{ND[0,1]}} %% \textorpdfstring para los bookmarks
Las funciones continuas y nunca diferenciables son, además de bastante contraintuitivas, difíciles de encontrar sin suficiente herramienta matemática. Es por eso que, hace no mucho tiempo, matemáticos de renombre pensaban que las funciones continuas tenían que ser derivables (en al menos algunos puntos). Esta convicción llegó a tal punto que, en 1806, un famoso matemático francés intentó formalizar un teorema que la demostrara. Su nombre era André-Marie Ampère, y su teorema se podría interpretar como sigue (ver \cite{Ampere}):

\begin{displayquote}
    Dada una función $f:I\subseteq \mathbb{R}\to\mathbb{R}$ \textit{continua}\footnote{Cabe señalar que el concepto de continuidad, tal como lo entendemos hoy, fue formalizado después de la época de Ampère, por lo que juzgar con precisión su formulación original requiere cierto cuidado histórico.}, $f$ es derivable excepto en puntos aislados de $I$.
\end{displayquote}

El objetivo de este capítulo es demostrar analíticamente la falsedad de este teorema. Para ello, presentaremos cuatro ejemplos sencillos de funciones continuas nunca derivables, así como la herramienta necesaria para comprenderlos. Comenzaremos con la función más emblemática de este tipo, la función $W$ de Weierstrass. Ésta se publicó en 1875, y marcó un punto de inflexión en el estudio del análisis, pues, a pesar de que no fue la primera en ser descubierta (el primer descubrimiento es atribuido a Bernard Bolzano en el año 1830, publicándose éste poco menos de cien años después), tomó por sorpresa a gran parte del mundo matemático: varios, escépticos por naturaleza, dudaron de la veracidad de esta afirmación; otros, curiosos pero timoratos, veían la existencia de estas funciones como una amenaza al orden y la simpleza de las matemáticas.