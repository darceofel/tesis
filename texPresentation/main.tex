\documentclass{beamer}
\usetheme{SimpleDarkBlue}

% ------- Paquetes adicionales -------
\usepackage{multicol, comment}
\usepackage[most]{tcolorbox}
\tcbuselibrary{skins}
\usepackage{etoolbox}

% ------- Comandos personalizados -------
\newcommand{\thesisChapter}[1]{%
    \section{#1}
    \begin{frame}
        \centering
        \Huge
        \begin{tcolorbox}[
            colframe=structure.fg, 
            colback=structure.fg, 
            coltitle=white, 
            arc=1mm,
            boxrule=0.8mm, 
            width=\textwidth
        ]
            \color{white} \centering #1
        \end{tcolorbox}
    \end{frame}
}

\newenvironment{thesisSection}[1]{%
    \subsection{#1}
    \def\SectionTitle{#1}%
}{}

\newcommand{\sectionFrame}[1]{%
    \begin{frame}
        \frametitle{\SectionTitle}
        #1
    \end{frame}
}

\newcommand{\contentBlock}[2]{%
    {%
        \ifstrequal{#1}{Teorema}{%
            \setbeamercolor{block title example}{bg=red!70!black, fg=white}%
            \setbeamercolor{block body example}{bg=red!5!white, fg=black}%
        }{%
            \ifstrequal{#1}{Observación}{%
                \setbeamercolor{block title example}{bg=green!70!black, fg=white}%
                \setbeamercolor{block body example}{bg=green!5!white, fg=black}%
            }{%
                \setbeamercolor{block title example}{bg=structure.fg!75!white, fg=white}%
                \setbeamercolor{block body example}{bg=structure.fg!5!white, fg=black}%
            }%
        }%
        \begin{exampleblock}{#1}
            #2
        \end{exampleblock}
    }%
}



% ------- Información del título -------
\title{Sobre el espacio de funciones \newline continuas, nunca derivables}
\author{
    {\small Presenta:} \\ \normalsize Diego Arceo Félix
    \vspace{0.2cm} \\
    {\small Director de tesis:} \\ \normalsize Dr. Alejandro Darío Rojas Sánchez 
    \vspace{0.5cm}
}
\institute
{
    Licenciatura en Matemáticas Aplicadas \\
    Universidad Panamericana 
}
\date{31 de octubre 2025}


% ------- Inicio del documento -------
\begin{document}
\frame{\titlepage}

\begin{frame}{Índice} \tableofcontents \end{frame}

%%%%% Introducción %%%%%
\thesisChapter{Introducción}
\begin{frame}
    \frametitle{Un poco de contexto histórico}
    \begin{itemize}
        \item En el siglo XIX, el análisis matemático empezaba a desarrolarse como lo que conocemos hoy en día.
        \item Nociones como continuidad y derivabilidad estaban siendo formalizadas.
        \item Se creía que las funciones continuas eran casi siempre derivables.
        \item En 1872, Weierstrass presentó el primer ejemplo explícito de una función continua pero derivable en ninguna parte.
    \end{itemize}
\end{frame}

\begin{frame}
    \centering
    {\usebeamerfont{normal text}
    \rmfamily
    \large
    \textit{Me aparto con temor y horror de la lamentable plaga de funciones continuas que no tienen derivadas...}
    
    \vspace{0.2em}
    \begin{flushright}
        \footnotesize
        --- \textbf{Hermite en una carta a Stieltjes,\\ 20 de mayo de 1893}
    \end{flushright}
    }
\end{frame}

%%%%% Capítulo 1 %%%%%
\thesisChapter{Algunos Ejemplos}

% Weierstrass
\thesisSection{Función \texorpdfstring{$W$}{W} de Weierstrass}
\sectionFrame{
    \contentBlock{Definición}{
        Sea $a \in (0,1)$ y $b$ un entero impar tal que $ab > 1+\frac{3\pi}{2}$. La función $W$ de Weierstrass está dada por:
        \[W(x) = \sum_{n=0}^{\infty} a^n \cos(b^n \pi x)\]
    }
}

\sectionFrame{
    \begin{figure}[h!]
        \centering
        \includegraphics[width=0.7\textwidth]{images/weier.png}
        \caption{Gráfica de $W$ para $a=0.35$ y $b=6$.}
    \end{figure}
}

% Peano
\thesisSection{La curva de Peano}
\sectionFrame{
    La curva $P$ de Peano es una función continua que llena el cuadrado unitario $[0,1] \times [0,1]$. Fue introducida por Giuseppe Peano en 1890 como el primer ejemplo conocido de una curva que llena un área.
    \newline
    \contentBlock{Definición}{}
}
\sectionFrame{
    \begin{figure}[h!]
    \centering
    \includegraphics[width=0.21\textwidth]{images/peano_1.png}  
    \hspace{0.1cm}  
    \includegraphics[width=0.21\textwidth]{images/peano_2.png}  
    \hspace{0.1cm}  
    \includegraphics[width=0.21\textwidth]{images/peano_3.png}
    \hspace{0.1cm}  
    \includegraphics[width=0.21\textwidth]{images/peano_4.png}
    
    \caption{Primeras cuatro iteraciones de la construcción geométrica de la curva de Peano.}   
    \end{figure}
}

% Mccarthy
\thesisSection{Función \texorpdfstring{$M$}{M} de Mccarthy}
\sectionFrame{
    Introducida por John Mccarthy en 1953, la función $M:\mathbb{R}\to\mathbb{R}$ es un ejemplo contemporáneo de una función continua en todas partes pero derivable en ninguna parte.
    \newline 
    \contentBlock{Definición}{Sea $M:\mathbb{R}\to\mathbb{R}$ definida como:
    \[
        M(x) = \sum_{n=0}^{\infty} \frac{g \left(2^{2^n} x \right)}{2^n}
    \]
    donde g está dada por $g(x) = 1 - |x|$, para $x\in[-2, 2]$, y $g(x + 4) = g(x)$ para todo $x \in\mathbb{R}$.}
}

\sectionFrame{
    \begin{figure}[h!]
        \centering
        \includegraphics[width=0.7\textwidth]{images/McCarthy.png}
        \caption{Gráfica de $M$ en $[0,1]$.}
    \end{figure}
}


% Lynch
\thesisSection{Funciones de Lynch}{}

%%%%% Capítulo 2 %%%%%
\thesisChapter{¿Qué tan grande es \texorpdfstring{$\mathcal{ND}[0,1]$}{ND[0,1]}?}
\thesisSection{Topológicamente...}
\sectionFrame{}

\thesisSection{Prevalencia y timidez}
\sectionFrame{}

%%%%% Capítulo 3 %%%%%
\thesisChapter{Prevalencia de \texorpdfstring{$\mathcal{ND}[0,1]$}{ND[0,1]}}
\thesisSection{Sondas}
\sectionFrame{
    \contentBlock{Definición}{
        Sean $P$ un subespacio finito-dimensional de $V$ y $T$ un subconjunto de $V$. Decimos que $P$ es una \textit{sonda} de $T$ si existen una base $A$ para $P$ y un conjunto $B\in \mathcal{B}(V)$ con $V\setminus T \subseteq B$ de tal manera que $\mu_A$ está concentrada en $P$ y es transversal a $B$.
    }
}

\sectionFrame{
    \contentBlock{Teorema}{
        Si $T\subseteq V$ tiene una sonda entonces es prevalente. 
    }
}

\sectionFrame{
    \contentBlock{Observación}{
        Una sonda para un conjunto boreliano $T$ es un subespacio de dimensión finita $P$ que está casi completamente contenido en cualquier traslación de $T$ (respecto a alguna medida de Lebesgue concentrada en $P$).    
    }
}

\thesisSection{Prevalencia de \texorpdfstring{$\mathcal{ND}[0,1]$}{ND[0,1]}}{}
\sectionFrame{}

\end{document}