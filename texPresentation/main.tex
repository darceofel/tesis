\documentclass{beamer}
\usetheme{SimpleDarkBlue}

% ------- Paquetes adicionales -------
\usepackage[dvipsnames]{xcolor}
\usepackage[spanish]{babel}
\usepackage{multicol, comment}
\usepackage[most]{tcolorbox}
\tcbuselibrary{skins}
\usepackage{etoolbox}

%% Algunas definiciones usuales
\DeclareMathOperator{\dom}{dom} 
\DeclareMathOperator{\interior}{int}
\DeclareMathOperator{\cerradura}{cl}
\DeclareMathOperator{\supp}{supp}

% ------- Comandos personalizados -------
\newcommand{\thesisChapter}[1]{%
    \section{#1}
    \begin{frame}
        \centering
        \Huge
        \begin{tcolorbox}[
            colframe=structure.fg, 
            colback=structure.fg, 
            coltitle=white, 
            arc=1mm,
            boxrule=0.8mm, 
            width=\textwidth
        ]
            \color{white} \centering #1
        \end{tcolorbox}
    \end{frame}
}

\newenvironment{thesisSection}[1]{%
    \subsection{#1}
    \def\SectionTitle{#1}%
}{}

\newcommand{\sectionFrame}[1]{%
    \begin{frame}
        \frametitle{\SectionTitle}
        #1
    \end{frame}
}


% Bloques de contenido personalizados
\newcommand{\contentBlock}[4][structure.fg]{%
    % Parámetros:
    % #1: Color del bloque (opcional, por defecto structure.fg)
    % #2: Título del bloque
    % #3: Subtítulo del bloque (opcional)
    % #4: Contenido del bloque

    {%
        \setbeamercolor{block title example}{bg=#1, fg=white}%
        \setbeamercolor{block body example}{bg=#1!5!white, fg=black}%
        \if\relax\detokenize{#3}\relax
            \begin{exampleblock}{#2}
                #4
            \end{exampleblock}
        \else
            \begin{exampleblock}{#2 (#3)}
                #4
            \end{exampleblock}
        \fi
    }%
}

% Parámetros de los bloques:
% #1: Subtítulo (opcional)
% #2: Contenido
\newcommand{\theoremBlock}[2][]{%
    \contentBlock[structure.fg!90!gray]{Teorema}{#1}{#2}
}

\newcommand{\definitionBlock}[2][]{%
    \contentBlock[teal!55!black]{Definición}{#1}{#2}
}

\newcommand{\propositionBlock}[2][]{%
    \contentBlock[cyan!25!black]{Proposición}{#1}{#2}
}

\newcommand{\lemmaBlock}[2][]{%
    \contentBlock[gray!95!black]{Lema}{#1}{#2}
}

\newcommand{\exampleBlock}[2][]{%
    \contentBlock[green!25!black]{Ejemplo}{#1}{#2}
}

\newcommand{\observationBlock}[2][]{%
    \contentBlock[NavyBlue!35!gray]{Observación}{#1}{#2}
}



% ------- Información del título -------
\addtobeamertemplate{title page}{\vspace*{-1cm}}{} % adjust the -1cm to move higher/lower

\title{Sobre el espacio de funciones \newline continuas, nunca derivables}
\author{
    {\small Presenta:} \\ \normalsize Diego Arceo Félix
    \vspace{0.2cm} \\
    {\small Director de tesis:} \\ \normalsize Dr. Alejandro Darío Rojas Sánchez 
    \vspace{0.5cm}
}
\institute
{
    Licenciatura en Matemáticas Aplicadas \\
    Universidad Panamericana 
}
\date{31 de octubre 2025}


% ------- Inicio del documento -------
\begin{document}
\frame{\titlepage}

\begin{frame}{Índice} \tableofcontents \end{frame}

%%%%% Introducción %%%%%
\thesisChapter{Introducción}
\begin{frame}
    \frametitle{Un poco de contexto histórico}
    \begin{itemize}
        \item En el siglo XIX, el análisis matemático empezaba a desarrolarse como lo que conocemos hoy en día.
        \item Nociones como la continuidad y la derivabilidad estaban siendo formalizadas.
        \item Se creía que las funciones continuas eran casi siempre derivables.
        \item En 1872, Weierstrass presentó el primer ejemplo explícito de una función continua pero derivable en ninguna parte.
    \end{itemize}
\end{frame}

\begin{frame}
    \centering
    {\usebeamerfont{normal text}
    \rmfamily
    \large
    \textit{Me aparto con temor y horror de la lamentable plaga de funciones continuas que no tienen derivadas...}
    
    \vspace{0.2em}
    \begin{flushright}
        \footnotesize
        --- \textbf{Hermite en una carta a Stieltjes,\\ 20 de mayo de 1893}
    \end{flushright}
    }
\end{frame}

%%%%% Capítulo 1 %%%%%
\thesisChapter{Algunos Ejemplos}

% Weierstrass
\thesisSection{Función \texorpdfstring{$W$}{W} de Weierstrass}
\sectionFrame{
    \theoremBlock[]{}
    \propositionBlock[]{}
    \definitionBlock[]{}
    \observationBlock[]{}
    \lemmaBlock[]{}
}
\sectionFrame{
    \definitionBlock{
        Sea $a \in (0,1)$ y $b$ un entero impar tal que $ab > 1+\frac{3\pi}{2}$. La función $W$ de Weierstrass está dada por:
        \[W(x) = \sum_{n=0}^{\infty} a^n \cos(b^n \pi x)\]
    }
}

\sectionFrame{
    \begin{figure}[h!]
        \centering
        \includegraphics[width=0.7\textwidth]{images/weier.png}
        \caption{Gráfica de $W$ para $a=0.35$ y $b=6$.}
    \end{figure}
}

% Peano
\thesisSection{La curva de Peano}
\sectionFrame{
    La curva $P$ de Peano es una función continua que llena el cuadrado unitario $[0,1] \times [0,1]$. Fue introducida por Giuseppe Peano en 1890 como el primer ejemplo conocido de una curva que llena un área.
    \newline
    \begin{figure}[h!]
        \centering
        \includegraphics[width=0.21\textwidth]{images/peano_1.png}  
        \hspace{0.1cm}  
        \includegraphics[width=0.21\textwidth]{images/peano_2.png}  
        \hspace{0.1cm}  
        \includegraphics[width=0.21\textwidth]{images/peano_3.png}
        \hspace{0.1cm}  
        \includegraphics[width=0.21\textwidth]{images/peano_4.png}
        
        \caption{La curva de Peano también tiene una construcción geométrica iterativa, estas son sus primeras cuatro iteraciones.}   
    \end{figure}
}
\sectionFrame{
    \definitionBlock{
        Dado $t\in\{0,1,2\}$ definimos, para $n\in\mathbb{Z},$
        \[A(n,t) = 
            \begin{cases}
                t, &\text{ si $n$ es par},\\
                2-t, & \text{ si $n$ es impar}.
            \end{cases}
        \]
        Ahora, para $t\in[0,1]$, si $\{t_n\}_{n\in\mathbb{N}}$ es una representación en base 3 de $t$, entonces definimos
        \[B_n(t) = 
            \begin{cases}
                t_1, & \text{si }n = 1 \\[6pt]
                A\!\left(\,\sum\limits_{k=1}^{n-1}t_{2k},\, t_{2n-1}\right), &\text{si } n>1.
            \end{cases}
        \]
    }
}
\sectionFrame{
    \definitionBlock{
        Definimos $\varphi:[0,1]\to[0,1]$ como 
        \[\varphi(t) = \sum_{n = 1}^{\infty}\frac{B_n(t)}{3^n}.\]
    }
    \theoremBlock{
        La función $\varphi$ está bien definida, es decir, no depende de la representación en base 3 que tomemos de $t$.
    }
}
\sectionFrame{
    \definitionBlock{
        La curva de Peano $P:[0,1]\to[0,1]\times[0,1]$ está dada por
        \[P(t) = \left(\varphi(t), 3\varphi\left(\frac{t}{3}\right)\right).\]
    }

    \theoremBlock{
        $\varphi$ es continua en es continua y nunca derivable.
    }
}

% Mccarthy
\thesisSection{Función \texorpdfstring{$M$}{M} de Mccarthy}
\sectionFrame{
    Introducida por John Mccarthy en 1953, la función $M:\mathbb{R}\to\mathbb{R}$ es un ejemplo contemporáneo de una función continua y nunca derivable.
    \newline 
    \definitionBlock{Sea $M:\mathbb{R}\to\mathbb{R}$ definida como:
    \[
        M(x) = \sum_{n=0}^{\infty} \frac{g \left(2^{2^n} x \right)}{2^n}
    \]
    donde g está dada por $g(x) = 1 - |x|$, para $x\in[-2, 2]$, y $g(x + 4) = g(x)$ para todo $x \in\mathbb{R}$.}
}

\sectionFrame{
    \begin{figure}[h!]
        \centering
        \includegraphics[width=0.7\textwidth]{images/McCarthy.png}
        \caption{Gráfica de $M$ en $[0,1]$.}
    \end{figure}
}


% Lynch
\thesisSection{Funciones de Lynch}
\sectionFrame{
    Este último ejemplo no tiene una regla de correspondencia explícita, sino que presenta una forma recursiva de construir funciones continuas y nunca derivables.
    \newline
    \theoremBlock[]{
        Si $a,b\in \mathbb{R}$ cumplen $a<b$ y $f : [a,b] \to \mathbb{R}$, entonces $f$ es continua si y sólo si la gráfica de $f$, $G(f) := \{(x,f(x)) : x\in [a,b]\}$, es un subconjunto compacto de $\mathbb{R}^{2}$.
    }
}
\sectionFrame{
    \definitionBlock[]{
        Decimos que $P\subseteq \mathbb{R}^{2}$ es una \textit{poligonal} si es la gráfica de una función recta a trozos y continua en $\pi_{1}[P]$, donde $\pi_{1}:\mathbb{R}^{2}\to \mathbb{R}$ es la proyección sobre la primera coordenada.
    }
}
\sectionFrame{
    Con lo anterior en mente, podemos finalmente abordar la construcción de las funciones de Lynch. La estrategia es la siguiente: construiremos una sucesión de compactos anidados en $\mathbb{R}^{2}$, digamos $\{C_n\}_{n\in\mathbb{N}}$, tales que, para todo $n\in \mathbb{N}$,
    \begin{itemize}\setlength\itemsep{0.2cm}
        \item[(a)]{$\pi_{1}[C_{n}] = [0,1]$,}
        \item[(b)]{diam$\,C_{n}[x]<1/n$, para todo $x\in[0,1]$, y}
        \item[(c)]{para cada $x\in[0,1]$ existe $y\in[0,1]$, con $0<|x-y|<1/n$ tal que si $p\in C_n[x]$ y $q\in C_n[y]$, entonces $|(p-q)/(x-y)|>n$.}
    \end{itemize}
}
\sectionFrame{
    \begin{figure}[h!]
        \centering
        \includegraphics[width=0.7\textwidth]{images/Lynch_example.png}
        \caption{Construcción recursiva de $C_n$.}
    \end{figure}
}
\sectionFrame{
    A partir de esto,
    \[
        C = \bigcap_{n=1}^{\infty} C_n
    \]
    es un compacto no vacío en $\mathbb{R}^{2}$ que cumple que $\pi_{1}[C] = [0,1]$. Además, para cada $x\in[0,1]$, $C[x]$ es un conjunto unitario, digamos $C[x] = \{y_x\}$. Así, si definimos $f:[0,1]\to\mathbb{R}$ como $f(x) = y_x$, entonces $G(f) = C$, y por tanto $f$ es continua.
    Por otro lado, la condición (c) garantiza que $f$ es nunca derivable.

}

%%%%% Capítulo 2 %%%%%
\thesisChapter{¿Qué tan grande es \texorpdfstring{$\mathcal{ND}[0,1]$}{ND[0,1]}?}
\thesisSection{Topológicamente}
\sectionFrame{
    Existen algunos conceptos topológicos que nos acercan a nociones sensibles sobre el tamaño de conjuntos. En particular, en esta sección nos enfocamos en la densidad y la categoría de Baire.
}

% Densidad
\sectionFrame{
    \definitionBlock{
        Decimos que un conjunto $A\subseteq V$ es \textit{denso} en $V$ si para todo abierto no vacío $U\subseteq V$, se cumple que $A\cap U \neq \emptyset$.
    }
    \vspace{0.3cm}
    O, analíticamente,
    \propositionBlock{
        Sean $(X,\,d)$ un espacio métrico y $D\subseteq X$. $D$ es denso en $X$ si y sólo si se cumple que, para cualesquiera $x\in X$ y $\varepsilon > 0$, $B(x,\varepsilon)\cap D\neq \varnothing$. 
    }
}
\sectionFrame{
    \theoremBlock{
        $\mathcal{ND}[0,1]$ es denso en $C[0,1]$.
    }
}
\sectionFrame{
    Ahora bien, la densidad de un conjunto sobre un espacio solo nos habla de cuán disperso está en este:
    \exampleBlock{
        Sea $X$ un conjunto no vacío equipado con la topología
        \[\tau = \{A\subseteq X\,|\, p\in A\}\cup\{\varnothing\}\]
        donde $p\in X$ es algún punto arbitrario. Observemos que, independientemente de la cardinalidad de $X$, el conjunto unitario $D=\{p\}$ es denso en $(X, \,\tau)$. 
    }
}

% Categoría de Baire
\sectionFrame{
    \definitionBlock{
        Sea $(X, \tau)$ un espacio topológico. Dado $A\subseteq X$, decimos que $A$ es \textit{denso en ninguna parte} si se cumple que
        \[\interior(\cerradura(A)) = \varnothing.\]
    }
    \propositionBlock{
        Sean $(X, \tau)$  un espacio topológico y $A\subseteq X$. El conjunto $A$ es denso en ninguna parte si y sólo si $X\setminus \cerradura(A)$ es denso en $X$.
    }
}
\sectionFrame{
    \definitionBlock{
        Sean $(X, \tau)$  un espacio topológico y $A\subseteq X$. Decimos que $A$ es \textit{de primera categoría} si es unión numerable de conjuntos densos en ninguna parte. Por el contrario, decimos que $A$ es \textit{de segunda categoría} si no es de primera categoría.
    }
}
\sectionFrame{
    \exampleBlock{
        El conjunto de los números racionales, como subespacio de los reales con la topología euclideana, es de primera categoría. 
    }
    \observationBlock{
        Sea $(X, \tau)$ un espacio topológico. Si $\{A_{n}\subseteq X\,|\,n\in\mathbb{N}\}$ es una sucesión de conjuntos de primera categoría, entonces $\bigcup_{n\in\mathbb{N}} A_n$ es de primera categoría. 
    }
}
\sectionFrame{
    \lemmaBlock{
        Sea $(X, \tau)$ un espacio topológico. Si toda intersección numerable de abiertos densos resulta en un conjunto denso, entonces $X$ es de segunda categoría. 
    }
    \theoremBlock[Baire]{Todo espacio métrico completo es de segunda categoría.}
}
\sectionFrame{
    \theoremBlock[Banach-Marzukiewicz]{
        $\mathcal{ND}[0,1]$ es de segunda categoría en $\mathcal{C}[0,1]$. 
    }
}


\thesisSection{Prevalencia y timidez}
\sectionFrame{
    A partir de ahora, $V$ será un espacio de Banach sobre $\mathbb{R}$, y $\mathcal{B}(V)$ será el conjunto de los borelianos de  $V$, esto es, la $\sigma$-álgebra generada por la topología de $V$. 
}
\sectionFrame{
    En esta sección nos intentamos acercar a teoremas de la forma ``casi todo elemento de $V$ tiene la propiedad $P$''. Para esto, antes debemos responder una pregunta fundamental:
    \newline
    \begin{center}
        ¿Existe alguna medida \textit{util} en $C[0,1]$?
    \end{center}

    \observationBlock{
        Sea $\mu$ una medida de borel en $V$ invariante bajo traslaciones. Si $V$ es dimensionalmente infinito y separable, entonces $\mu$ es la constante 0, o todos los abiertos no vacíos en $V$ tienen medida infinita.
    }
}
\sectionFrame{
    Ante la ausencia de una medida de Lebesgue en espacios de Banach infinitamente dimensionales, surge la necesidad de definir una noción alternativa de ``casi todo'' que nos permita trabajar en estos espacios.
}
\sectionFrame{
    Buscamos generalizar las siguientes características de los conjuntos de medida cero:
    \begin{enumerate}\setlength\itemsep{0.2cm}
        \item[(1)] Los conjuntos de medida cero no tienen interior,
        \item[(2)] Todo subconjunto de un conjunto con medida cero también tiene medida cero,
        \item[(3)] Unión numerable de conjuntos de medida cero también tiene medida cero, y 
        \item[(4)] Traslaciones de conjuntos de medida cero también tienen medida cero.
    \end{enumerate}
}
\sectionFrame{
    \definitionBlock[]{
        Decimos que una medida de Borel $\mu$ es \textit{transversal} a $S\in\mathcal{B}(V)$ si se cumplen las siguientes condiciones:
        \begin{enumerate}
            \item[(1)] Existe un compacto $K\subseteq V$ tal que $0<\mu(K)<\infty$, y
            \item[(2)] $\mu(S + v) = 0$ para todo $v\in V$.  
        \end{enumerate}
    }
    \definitionBlock[]{
        Decimos que $B\subseteq V$ es \textit{tímido} si existen $S\in \mathcal{B}(V)$ tal que $B\subseteq S$, y una medida $\mu$ transversal a $S$. Por otro lado, decimos que $B$ es \textit{prevalente} si $V\setminus B$ es tímido.
    }
}

%%%%% Capítulo 3 %%%%%
\thesisChapter{Prevalencia de \texorpdfstring{$\mathcal{ND}[0,1]$}{ND[0,1]}}
\thesisSection{Sondas}
\sectionFrame{
    Hasta ahora, demostrar que un conjunto es prevalente presenta un verdadero reto. En esta sección nos enfocamos en encontrar alguna herramienta que nos facilite esta tarea.
}

\sectionFrame{
    \definitionBlock{
        Sean $P$ un subespacio finito-dimensional de $V$ y $T$ un subconjunto de $V$. Decimos que $P$ es una \textit{sonda} de $T$ si existen una base $A$ para $P$ y un conjunto $B\in \mathcal{B}(V)$ con $V\setminus T \subseteq B$ de tal manera que $\mu_A$ está concentrada en $P$ y es transversal a $B$.
    }
    O, analíticamente,
    \propositionBlock{}
}

\sectionFrame{
    \theoremBlock{
        Si $T\subseteq V$ tiene una sonda entonces es prevalente. 
    }
}

\sectionFrame{
    \observationBlock{
        Una sonda para un conjunto boreliano $T$ es un subespacio de dimensión finita $P$ que está casi completamente contenido en cualquier traslación de $T$ (respecto a alguna medida de Lebesgue concentrada en $P$).    
    }
}

\thesisSection{Prevalencia de \texorpdfstring{$\mathcal{ND}[0,1]$}{ND[0,1]}}{}
\sectionFrame{}

\end{document}
