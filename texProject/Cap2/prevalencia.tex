\section{Prevalencia y timidez}
% Cuando uno piensa en medidas sobre algún espacio \textit{parecido} a $\mathbb{R}^n$, usualmente, la primera medida que se le viene a la cabeza es la medida de Lebesgue. Teniendo en cuenta el problema que buscamos resolver, uno se preguntaría si podemos extender esta medida a $C[0,1]$, y con eso ver cuánto mide $\mathcal{ND}[0,1]$. Sin embargo, el siguiente argumento rompe la ilusión de resolver nuestro problema tan fácilmente, veamos:
Teniendo ya una intuición topológica sobre el tamaño de $\mathcal{ND}[0,1]$, en esta sección buscaremos acercarnos a una definición más precisa de este concepto; en primera instancia abordaremos la siguiente pregunta: ¿podemos definir una medida ``útil'' en $C[0,1]?$


Es importante remarcar que ``útil'' es la palabra clave de la pregunta, pues, en general, medidas hay en todos los conjuntos, pero no todas tienen una interpretación natural o intuitiva. 
Tomando inspiración en la medida de Lebesgue, buscamos una medida tal que, como mínimo, esté definida en todos los abiertos del espacio, sea invariante bajo traslaciones y no sea constante. 

Un primer examen nos empuja a afirmar que esta medida (o al menos una parecida) debe existir en $\mathcal{C}[0,1]$, pero el siguiente argumento rompe esta ilusión (su demostración se puede encontrar en la \cref{appendix:obs_cap_2}).
\begin{observation*}
    Sea $V$ un $\mathbb{R}$-espacio de Banach dimensionalmente infinito y separable. Si $\mu$ es una medida de borel en $V$ invariante bajo traslaciones, entonces $\mu$ es la constante 0, o todos los abiertos no vacíos en $V$ tienen medida infinita.
\end{observation*}

De nuevo nos encontramos con un obstáculo. Buscamos darle una intuición analítica al tamaño de $\mathcal{ND}[0,1]$, pero una de las herramientas más útiles para esto, las medidas, se queda corta en nuestro caso. Sin embargo, la única razón por la cual buscábamos tener una medida en $\mathcal{C}[0,1]$ era para llegar a tener una definición formal para afirmaciones de la forma \textit{casi toda función continua en [0,1] $\dots$} Es decir, para nuestros propósitos no necesitamos de todo el rigor de tal herramienta, sino que nos bastaría con abstraer algunas de sus propiedades y generalizarlas en otro concepto. 
Puntualmente, considerando a la medida de Lebesgue, algunas de las propiedades que buscamos preservar son las siguientes:

\begin{enumerate}[label = (\arabic*{})]
    \item Los conjuntos de medida cero no tienen interior,
    \item Todo subconjunto de un conjunto con medida cero también tiene medida cero,
    \item Unión numerable de conjuntos de medida cero también tiene medida cero, y
    \item Traslaciones de conjuntos de medida cero también tienen medida cero.
\end{enumerate}

En resumen, buscamos un concepto que generalice a los conjuntos que tienen medida de Lebesgue cero, y que preserve las propiedades esenciales de la medida de Lebesgue. A lo largo de este capítulo desarrollaremos la herramienta necesaria para introducir tal abstracción: la ``timidez'' y la ``prevalencia''.
\begin{notation*}
    A partir de ahora, $V$ será un espacio de Banach sobre $\mathbb{R}$, y $\mathcal{B}(V)$ será el conjunto de los borelianos de  $V$, esto es, la $\sigma$-álgebra generada por la topología de $V$. 
\end{notation*}
\begin{observation*}
    Notemos que $V$ es Haussdorff, y que las funciones suma, $+:V\times V \to V$, y producto por escalares, $\cdot:\mathbb{R}\times V\to V$, son continuas en sus dominios dotados de la topología producto\footnote{Formalmente, $V$ es un espacio vectorial topológico.}. Además, dados $v\in V$ y $S\in \mathcal{B}(V)$, como la función $f_v(x) = x-v$ es continua, entonces 
    \[(S+v) = f_{v}^{-1}[S]\in\mathcal{B}(V).\]
\end{observation*}
\begin{definition}
    Decimos que una medida de Borel $\mu$ es \textit{transversal} a $S\in\mathcal{B}(V)$ si se cumplen las siguientes condiciones:
    \begin{enumerate}
        \item[(1)] Existe un compacto $K\subseteq V$ tal que $0<\mu(K)<\infty$, y
        \item[(2)] $\mu(S + v) = 0$ para todo $v\in V$.  
    \end{enumerate}
\end{definition}
\begin{definition}\label{def:timido}
    Decimos que $B\subseteq V$ es \textit{tímido} si existen $S\in \mathcal{B}(V)$ tal que $B\subseteq S$, y una medida $\mu$ transversal a $S$. Por otro lado, decimos que $B$ es \textit{prevalente} si $V\setminus B$ es tímido.
\end{definition}
En otras palabras, los conjuntos tímidos serán nuestra generalización de los conjuntos de medida cero. Observemos que, por definición, los conjuntos tímidos cumplen la segunda de nuestras propiedades a preservar. Veamos que también satisfacen la cuarta y la primera:
\begin{proposition}
    Si $S\subseteq V$ es tímido, entonces toda traslación de $S$ también es tímida.
\end{proposition}
\begin{proposition}
    Todo conjunto tímido tiene interior vacío.
\end{proposition}
\begin{proof}
    Sean $B\subseteq V$ tímido, y supongamos que $U = \interior B \neq \varnothing$. Existen $S\in\mathcal{B}(V)$ tal que $B\subseteq S$, una medida de Borel $\mu$ transversal a $S$, y un compacto $K\subseteq V$ tal que $0<\mu(K)<\infty$.\\\\ 
    Notemos que $\{U+v\,|\,v\in V\}$ es cubierta abierta de $K$, y por tanto existen $v_1,\dots,v_n\in V$ tales que \[K\subseteq \bigcup_{i=1}^{n}\,(U+v_i).\]
    Como $\mu(K) >0$, entonces existe $i\leq n$ tal que $\mu(U+v_i) > 0$. Sin embargo, como $\mu$ es transversal a $S$, tenemos que $0<\mu(U+v_i)\leq \mu(S+v_i) = 0$, que es absurdo.
\end{proof}
\begin{corollary}
    Todo conjunto prevalente es denso.
\end{corollary}

Antes de demostrar que los conjuntos tímidos satisfacen la tercera propiedad, debemos dar una vuelta por algunos resultados sobre medidas en espacios producto:
\begin{definition}
    Sea $(X,\mathcal{A},\mu)$ un espacio de medida. Decimos que $\mu$ es \textit{$\sigma$-finita} si existe $\{A_{i}\}_{i\in\mathbb{N}}\subseteq\mathcal{A}$ tal que $\mu(A_i)<\infty$ para todo $i\in\mathbb{N}$, y 
    \[\bigcup_{i\in\mathbb{N}}A_i = X.\] 
\end{definition}
\begin{definition}
    Sean $(X,\mathcal{A})$, y $(Y,\mathcal{B})$ dos espacios medibles. Definimos la \textit{$\sigma$-álgebra producto} $\mathcal{A}\otimes\mathcal{B}$ como la $\sigma$-álgebra generada por la colección de todos los \textit{rectángulos medibles}, es decir,
    \[\mathcal{A}\otimes\mathcal{B} = \sigma(\{A\times B\,|\,A\in\mathcal{A},\,B\in\mathcal{B}\}).\]
\end{definition}
Los siguientes dos teoremas son de gran utilidad para medir en espacios producto. Sus demostraciones se pueden consultar en \cite{Measure_theory_cohn}.
\begin{theorem}\label{thm:medida_producto}
    Sean $(X,\mathcal{A},\mu)$, y $(Y,\mathcal{B},\nu)$ dos espacios de medida. Si $\mu$ y $\nu$ son $\sigma$-finitas, entonces existe una única medida $\lambda:\mathcal{A}\otimes\mathcal{B}\to[0,\infty]$ tal que 
    \[\lambda(A\times B) = \mu(A)\nu(B).\]
    para cualesquiera $A\in\mathcal{A}$ y $\,B\in\mathcal{B}$. A esta medida la llamaremos la \textit{medida producto} y la denotaremos por $\mu \times \nu$.
\end{theorem}

El siguiente teorema, un importante antecesor al teorema de Fubini, nos da suficientes herramientas para calcular integrales de funciones medibles en espacios producto. 

\begin{theorem}[Tonelli]\label{thm:Tonelli}
    Sean $(X,\mathcal{A},\mu)$, y $(Y,\mathcal{B},\nu)$ dos espacios de medida, en donde $\mu$ y $\nu$ son medidas $\sigma$-finitas. Si $f:X\times Y\to [0,\infty)$ es medible respecto a $\mu\times \nu$, y definimos $g_{x}:Y\to[0,\infty)$ y $h_{y}:X\to[0,\infty)$ como  $g_{x}(y) = f(x,y) = h_{y}(x)$, entonces:
    \begin{enumerate}
        \item[(1)] Las funciones 
        \[x\mapsto \int_{Y}g_{x}\;\text{d}\nu\quad \text{y}\quad y\mapsto  \int_{X}h_{y}\;\text{d}\mu\] son $\mathcal{A}$ y $\mathcal{B}$ medibles, respectivamente, y
        \item[(2)] $f$ cumple que  $\begin{aligned}
            \;\int_{X\times Y}f \;\text{d}(\mu\times\nu) = \int_{X}\left(\int_{Y}g_x\; \text{d}\nu\right)\text{d} \mu = \int_{Y}\left(\int_{X}h_y\; \text{d}\mu\right)\text{d} \nu.
        \end{aligned}$ 
    \end{enumerate}
\end{theorem}
Para demostrar la tercer condición de los conjuntos tímidos, antes consideraremos un caso más simple: demostraremos que unión finita de conjuntos tímidos resulta en un conjunto tímido. Para ello, teniendo dos medidas transversales a conjuntos tímidos (o a borelianos que los contienen), debemos encontrar una tercera que también sea transversal a ambos:
\begin{notation*}
    Para cada $S\subseteq V$, denotaremos $S^{+} = \{(x,y)\in V\times V\,:\,x+y\in S\}.$
\end{notation*}
Importante notar que, como la función suma de vectores es continua en $V\times V$, para todo boreliano $S\subseteq V$, $S^{+}$ es un boreliano de $V\times V$. Con esto podemos definir la medida buscada en este último propósito:
\begin{definition}
    Sean $\mu$ y $\nu$ dos medidas de Borel $\sigma$-finitas. Definimos \textit{la convolución} $\mu * \nu: \mathcal{B}(V)\to[0,\infty)$ como:
    \[(\mu * \nu) (S) = (\mu\times\nu)\,(S^{+}).\]
\end{definition}
\begin{observation*}
     Si $\mu$ y $\nu$ son dos medidas de Borel $\sigma$-finitas en $V$, entonces su convolución también es una medida de Borel en $V$. Además, por el segundo inciso del  \cref{thm:Tonelli}, $\mu * \nu = \nu *\mu$.
\end{observation*}
\begin{definition}
    Decimos que una medida $\mu:\mathcal{A}\to\mathbb{R}$ está \textit{concentrada} en $P\in \mathcal{A}$ si se cumple que, para todo $S\in \mathcal{A}$,
    \[\mu(S) = \mu(S\cap P).\]
\end{definition}
\begin{lemma}\label{lem:conv_transversal}
    Si $\mu : \mathcal{B}(V) \to [0, \infty)$ es una medida de Borel transversal a $S \in \mathcal{B}(V)$, concentrada en un compacto de medida positiva y finita, entonces para cualquier medida de Borel $\sigma$-finita $\nu : \mathcal{B}(V) \to [0, \infty)$, la convolución $\mu * \nu$ satisface que
    \[(\mu * \nu)(S+z) = 0\]
    para cualquier $z\in V$. Si, además, $\nu$ es finita y no nula, entonces $\mu * \nu$ es transversal a $S$.
\end{lemma}
\begin{proof}
    Para cada $z\in V$, tomemos $f_z:V\times V\to[0,\infty)$ dada por $f_z = \chi_{(S+z)^+}$.
    Por el \cref{thm:Tonelli},
    \begin{align*}
        (\mu * \nu) (S+z) &= (\mu\times\nu)(S^+) =  \int_{V\times V}f_z\;\text{d}(\mu\times\nu) = \int_{V}\int_{V} f_z \;\text{d}\mu \;\text{d}\nu \\[8pt]
        &= \int_{V}\int_{V} \chi_{(S+z-y)^+} \;\text{d}\mu \;\text{d}\nu = \int_{V}\mu(S+z-y)\;\text{d}\nu = 0.
    \end{align*}
    Ahora supongamos que $\nu$ es finita y no nula. Sea $K\subseteq V$ el compacto en el que está concentrada $\mu$. Observemos que    
    \[(\mu*\nu)(K) = (\mu\times\nu)(K^+) \leq \mu(V)\nu(V) = \mu(K)\nu(V).\]
    Por tanto 
    \[0<(\mu*\nu)(K)<\infty,\]
    de donde $\mu*\nu$ es transversal a $S$.
\end{proof}
Teniendo esto, y notando que todo conjunto tímido tiene asociada una medida transversal finita y concentrada en un compacto, podemos demostrar lo siguiente:
\begin{corollary}
    Unión finita de conjuntos tímidos es tímida.
\end{corollary}
Además de acercarnos un poco más a nuestro propósito, este lema responde una importante pregunta que aún no hemos planteado: ¿Quiénes son los conjuntos tímidos en $\mathbb{R}^n$? ¿Qué relación tienen con los conjuntos de medida Lebesgue cero?
\begin{theorem}\label{thm:Timidos_en_Rn}
    Un conjunto $S\subseteq \mathbb{R}^{n}$ es tímido si y sólo si tiene medida de Lebesgue cero.
\end{theorem}
\begin{proof} Sea $S\subseteq\mathbb{R}^n$ y $\lambda:\mathcal{B}(\mathbb{R}^n)\to[0,\infty)$ la medida de Lebesgue en $\mathbb{R}^n$.\\[8pt]
Supongamos que $\lambda(S) = 0$. Entonces $S$ está contenido en algún boreliano con medida de Lebesgue cero, y, como $\lambda$ es invariante bajo traslaciones, $\lambda$ es transversal a tal boreliano. Por lo tanto, $S$ es tímido.\\[8pt] 
Sea $S\subseteq \mathbb{R}^{n}$ tímido, y supongamos sin pérdida de generalidad que es boreliano. Entonces existe una medida de Borel $\mu$ transversal a $S$. Como $\mu$ es transversal a $S$, existe un compacto $K\subseteq \mathbb{R}^{n}$ de medida positiva y finita. Consideremos $\hat{\mu}:\mathcal{B}(\mathbb{R}^n)\to[0,\infty)$ dada por
    \[\hat{\mu}(A) = \mu(A\cap K) / \mu(K).\]
    Así, $\hat{\mu}$ es una medida de Borel finita y transversal a $S$. Entonces, por el \cref{lem:conv_transversal}, $(\hat{\mu} * \lambda) (S) = 0$. 
    Ahora, como $\lambda$ es invariante bajo traslaciones, por el \cref{thm:Tonelli},
    \begin{align*}
        \lambda(S) &= \lambda (S)\hat{\mu}(\mathbb{R}^n) 
        = \int_{\mathbb{R}^n}\lambda(S)\;\text{d}\hat{\mu} = \int_{\mathbb{R}^n}\lambda(S-x)\;\text{d}\hat{\mu} 
        = \int_{\mathbb{R}^n}\int_{\mathbb{R}^n}\chi_{S-x}\;\text{d}\lambda\;\text{d}\hat{\mu}\\[8pt] 
        &= \int_{\mathbb{R}^n}\int_{\mathbb{R}^n}\chi_{S^+}\;\text{d}\lambda\;\text{d}\hat{\mu} 
        = \int_{\mathbb{R}^n\times\mathbb{R}^n}\chi_{S^{+}}\;\text{d}(\hat{\mu}\times\lambda) = (\hat{\mu}*\lambda) (S) = 0.
    \end{align*}
    Es decir, $S$ tiene medida Lebesgue cero.
\end{proof}
En otras palabras, ser ``tímido'' en $\mathbb{R}^n$ no es una generalización burda de ``tener medida Lebesgue cero'', sino que es una definición equivalente. A partir de este teorema podemos generalizar a los conjuntos de medida de Lebesgue plena (es decir, aquellos cuyo complemento tiene medida de Lebesgue cero):
\begin{definition}
    Si $A\subseteq V$ es prevalente, entonces decimos que \textit{casi todo elemento de $V$ pertenece a $A$}.
\end{definition}
Existe un último teorema que debemos enunciar antes de demostrar la cuarta propiedad de los conjuntos tímidos. Éste, generalizando al  \cref{thm:medida_producto}, nos habla sobre medidas en espacios producto de tamaño arbitario:
\begin{definition}
    Sean $\mathcal{A} = \{(X_{\alpha},\,\mathcal{B}_{\alpha})\,|\, \alpha\in I\}$ una familia de espacios medibles y $X = \prod_{\alpha \in I}X_{\alpha}$. Decimos que $E\subseteq X$ es un \textit{conjunto elemental} si existe $F\subseteq I$ finito tal que 
    \[E = \bigcap_{\alpha \in F}\pi_{\alpha}^{-1}[B_{\alpha}],\]
    en donde $\pi_{\alpha}:X\to X_{\alpha}$ es la proyección sobre $X_{\alpha}$, y $B_\alpha\in\mathcal{B}_\alpha$, para cada $\alpha \in I$.  A la familia de todos estos conjuntos la denotaremos por 
    \[\mathcal{E}(X) = \{E\subseteq X\,|\, E \text{ es elemental}\}.\]
\end{definition}
\begin{theorem}\label{thm:medidas_producto_inf}
    Sea $\mathcal{A} = \{(X_{\alpha},\,\mathcal{B}_{\alpha},\,\mu_{\alpha})\,|\, \alpha\in I\}$ una familia de espacios de medida, con $\mu_{\alpha}(X_{\alpha}) = 1$ para todo $\alpha\in I$, y $\mathcal{B}_\alpha$ la $\sigma$-álgebra de Borel en $X_{\alpha}$. 
    Llamemos 
    \[X = \prod_{\alpha\in I}X_{\alpha}.\]
    Entonces existe una única medida $\mu:\sigma(\mathcal{E}(X))\to[0,1]$ tal que, para todo 
    \[E = \bigcap_{\alpha \in F}\pi_{\alpha}^{-1}[B_{\alpha}]\in\mathcal{E}(X),\]
    se cumple que
    \[\mu(E) = \prod_{\alpha \in F} \mu(B_{\alpha}).\]
\end{theorem}
A la medida de este teorema la llamaremos la \textit{medida producto de $X$}. Su demostración se puede consultar en \cite{InfiniteProducts}. Si bien este teorema es intersante por sí mismo, nosotros necesitamos que la medida producto esté definida los borelianos del espacio, no en la $\sigma$-álgebra generada por los elementales.

\begin{proposition}
    Sea $\mathcal{A} = \{(X_{\alpha},, \mathcal{B}_{\alpha}) \mid \alpha \in I\}$ una familia de espacios medibles, donde $\mathcal{B}_\alpha$ es la $\sigma$-álgebra de Borel de $X_\alpha$, y cada $X_\alpha$ es un espacio segundo numerable. Si $|I|\leq \omega$, y \[X = \prod_{\alpha \in I}X_{\alpha}\]
    está dotado de la topología producto, entonces la $\sigma$-álgebra de Borel en $X$ coincide con la $\sigma$-álgebra generada por los conjuntos elementales, es decir,
    \[\mathcal{B}(X) = \sigma(\mathcal{E}(X)).\]
\end{proposition}
\begin{proof}
    Por definición de la topología producto, cada conjunto elemental es un boreliano, de donde $\sigma(\mathcal{E}(X))\subseteq \mathcal{B}(X)$.\\\\
    Ahora, llamemos $C_\alpha$ a la base numerable de cada $X_\alpha$. Como $|I|\leq \omega$, entonces la familia
    \[\mathcal{U} = \left\{\bigcap_{\alpha\in F} \pi^{-1}[U_\alpha]\,:\,F\subseteq I \text{ finito},\; U_{\alpha}\in C_\alpha \right\}\]
    es una base numerable para $X$. Notemos que $\mathcal{U}\subseteq \mathcal{E}(X)$. Por tanto, como $\mathcal{U}$ es numerable, $\mathcal{B}(X)\subseteq \sigma(\mathcal{U})\subseteq \sigma(\mathcal{E}(X))$.
\end{proof}

Finalmente, demostremos la última de las propiedades de los conjuntos tímidos:
\begin{theorem}
    Unión numerable de conjuntos tímidos es tímido.
\end{theorem}
\begin{proof}
    Sea $\{S_n\}_{n\in\mathbb{N}}$ una sucesión de conjuntos tímidos en $V$, y llamemos $\mu_n$ a una medida transversal de cada $S_n$. Sin pérdida de generalidad, supongamos que todo $S_n$ es boreliano. \\\\
    Cada $\mu_n$ tiene asociado un compacto $K_n\subseteq V$ de medida positiva.
    Para cada $n\in \mathbb{N}$, la familia \[\mathcal{U}_n=\{B(x, 2^{-n-2}):x\in K_n\}\] es una cubierta abierta de $K_n$, que es compacto. Por tanto, para cada $n\in\mathbb{N}$ existe una subfamilia finita de $\mathcal{U}_n$ que cubre a $K_n$, llamemos $U^n_i$ a los elementos de esta familia. Observemos que 
    \[K_n = K_n\cap \left(\bigcup_{i = 1}^k\cerradura U^n_i\right) = \bigcup_{i = 1}^kK_n\cap \cerradura U^n_i.\]
    Como cada $U^n_i$ tiene diametro menor que $2^{-n}$, con esto podemos encontrar un subconjunto compacto $U_n\subseteq K_n$ de medida positiva y diámetro menor que $2^{-n}$. Ahora, notemos que para cualquier $v\in V$, la medida $\mu^v_n$ dada por
    \[\mu^v_n(A) = \mu_n(A + v)\]
    también es transversal a $S_n$. Por tanto, sin pérdida de generalidad, podemos suponer que $U_n$ contiene al origen de $V$. Además, por un argumento análogo al que dimos en la ida del \cref{thm:Timidos_en_Rn}, podemos suponer, sin pérdida de generalidad, que $\mu_n(U_n) = \mu_n(V) = 1$. \\\\
    Consideremos, para $n\in\mathbb{N}\cup\{0\}$,
    \[W_n = \prod_{m\neq n} U_m.\]
    Por el teorema de Tychonoff, cada $W_n$ es compacto. Tomemos $f:W_0\to V$ dada por
    \[f((v_n)_{n\in\mathbb{N}}) = \sum_{n=1}^{\infty}v_n.\]
    Observemos que $f$ está bien definida, pues cada $v_n$ tiene norma menor que $2^{-n}$, de donde las sumas parciales de la serie forman una sucesión de Cauchy en $V$, que es completo. Veamos que, además, $f$ es continua: sean $(v_n)_{n\in\mathbb{N}}\in W_0$, $\varepsilon>0$ y $N\in\mathbb{N}$ tal que $2^{2-N}<\varepsilon$. Definamos 
    \[C = \prod_{n=1}^{N}B\left(v_n,\, \frac{\varepsilon}{2N}\right) \times \prod_{n>N}U_n.\]
    Por definición de la topología producto, $C$ es una vecindad abierta de $(v_n)_{n\in\mathbb{N}}$. Tomemos $(w_n)_{n\in\mathbb{N}}\in C$, entonces
    \begin{align*}
        ||f(v_n)-f(w_n)|| &\leq \sum_{n = 1}^{N}\left|\left| v_n - w_n\right|\right| + \sum_{n>N}\left|\left| v_n\right|\right| + \sum_{n>N}\left|\left| w_n\right|\right| \\
        &\leq \sum_{n = 1}^{N}\frac{\varepsilon}{2N} + 2^{-N} + 2^{-N} < \varepsilon,
    \end{align*}
    de donde $f$ es continua. Así, como $W_0$ es compacto, $f[W_0]$ es compacto. \\\\
    Ahora, por el \cref{thm:medidas_producto_inf} podemos llamar $\hat{\mu}_n$ a la medida producto de $W_n$.  Finalmente, consideremos a la medida $\nu_n:\mathcal{B}(V) \to[0,1]$ dada por\footnote{Intentando ahorrar un poco de notación, en esta definición estamos pensando a cada $W_n$ como subespacio de $W_0$.}
    \[\nu_n(A) = \hat{\mu}_n(f^{-1}[A]\cap W_n).\]
    Observemos que, en particular, $\nu_0$ es algo así como la ``convolución infinita'' de todas las $\mu_n$. Además, notemos que, para cualquier $n\in\mathbb{N}$, $\hat{\mu}_0 = \mu_n\times \hat{\mu}_n$, de donde para cualquier boreliano $A \subseteq V$
    \begin{align*}
        \nu_0(A) &= \hat{\mu}_0 (f^{-1}[A])\\[7pt]
        &= (\mu_n\times \hat{\mu}_n)(f^{-1}[A])\\[7pt]
        &= (\mu_n\times \hat{\mu}_n) \left(\left\{(v_m)_{m\in\mathbb{N}} \in W_0\,:\,\sum_{m\in\mathbb{N}}v_m\in A\right\}\right) \\[7pt]
        &= (\mu_n\times \hat{\mu}_n) \left(\left\{(v,\, (v_m)_{m_\neq n})\in U_n\times W_n \,:\,v+\sum_{m\neq n}v_m\in A\right\}\right) \\[7pt]
        &= (\mu_n\times \hat{\mu}_n) \left(\left\{(v,\, w)\in U_n\times  W_n \,:\,v+f|_{W_n}(w) \in A\right\}\right)\\[7pt]
        &=\int_{U_n}\int_{W_n}\chi_{f^{-1}[A-v]\cap W_n}\; \text{d} \hat{\mu}_n\; \text{d}\mu_n\\[7pt]
        &=\int_{U_n}\hat{\mu}_n(f^{-1}[A-v]\cap W_n)\;\text{d}\mu_n\\[7pt]
        &=\int_{U_n}\nu_n(A-v)\;\text{d}\mu_n\\[7pt]    
        &= (\mu_n * \nu_n)(A),
    \end{align*}
     y por tanto, como $\mu_n$ es transversal a cada $S_n$, se tiene que $\nu_0$ es transversal a cada $S_n$. Por lo tanto, $\nu_0$ es transversal a $\bigcup_{n\in\mathbb{N}}S_n$.
\end{proof}
