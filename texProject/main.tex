%%%% Configuración general del archivo
\documentclass[12pt, letterpaper, twoside]{report}
\usepackage[top=3cm, bottom=3cm, left=3.5cm, right=2.5cm]{geometry}

%%%% Paquetes de matemáticas
\usepackage{amsmath,amsthm,amsfonts,amssymb,amscd, mathtools}

%%%% Vincular teoremas, proposiciones, etc.
% \usepackage[hidelinks]{hyperref} % Por si se quieren quitar las cajas rojas sobre los links/referencias
\usepackage{hyperref}
\usepackage{xcolor}
\usepackage{float}
\hypersetup{
    colorlinks,
    linkcolor={red!50!black},
    citecolor={blue!80!black},
    urlcolor={blue!80!black}
}
\usepackage[nameinlink]{cleveref}

% Incluir páginas PDF (para portada)
\usepackage{pdfpages}

\usepackage{emptypage} % para páginas en blanco sin número ni encabezado
% Redefinir \chapter para que inicie en página impar
\let\oldchapter\chapter
\renewcommand{\chapter}{\cleardoublepage\oldchapter}

%%%% Redefinir nombre al vincular teoremas, proposiciones, etc.
\crefname{lemma}{lema}{lemas}
\crefname{proposition}{proposición}{proposiciones}
\crefname{theorem}{teorema}{teoremas}
\crefname{definition}{definición}{definiciones}
\crefname{example}{ejemplo}{ejemplos}
\crefname{corollary}{corolario}{corolarios}

%%%% Enumeración de teoremas
\theoremstyle{definition}
\newtheorem{definition}{Definición}[section] %% Definir a partir de qué tipo de división se enumeran
\newtheorem{theorem}[definition]{Teorema}
\newtheorem{proposition}[definition]{Proposición}
\newtheorem{lemma}[definition]{Lema}
\newtheorem{corollary}[definition]{Corolario}
\newtheorem{example}[definition]{Ejemplo}
\newtheorem*{notation*}{Notación}
\newtheorem*{observation*}{Observación}
\newtheorem*{proposition*}{Proposición}

%%%% Cambiar nombre de títulos predefinidos a español.
\usepackage[spanish]{babel}
\addtolength{\footnotesep}{2mm}

%%%% bibliografía
\usepackage[backend=biber, style=ieee]{biblatex}
\DeclareFieldFormat[article]{title}{\mkbibemph{#1}}
\DeclareFieldFormat{journaltitle}{\textnormal{#1}}
\DeclareFieldFormat[thesis]{title}{\mkbibemph{#1}}
\addbibresource{referencias.bib} 
\usepackage{csquotes}

%% En la bibliografía, marcar fuentes que no se revisaron directamente
\DeclareBibliographyCategory{asterisk}
\renewbibmacro*{begentry}{\ifcategory{asterisk}{*\addspace}{}}
\addtocategory{asterisk}{weier, Ampere}

%% Para subrayar texto 
\usepackage{color,soul}
\setul{0.5ex}{0.3ex}
\setulcolor{yellow}

%% Para crear texto para rellenar
\usepackage{lipsum}

\usepackage[shortlabels]{enumitem} %% Para enumerar
\usepackage{longfbox} %% caja de ida/regreso en demostraciones

\renewcommand{\qedsymbol}{\raisebox{-\baselineskip}{\llap{\openbox}}} %% bajar una línea el cuadrado al final de las pruebas
\newcommand{\overbar}[1]{\mkern 1.5mu\overline{\mkern-1.5mu#1\mkern-1.5mu}\mkern 1.5mu} %% Definir línea para notación de convergencia

%% Algunas definiciones usuales
\DeclareMathOperator{\dom}{dom} 
\DeclareMathOperator{\interior}{int}
\DeclareMathOperator{\cerradura}{cl}
\DeclareMathOperator{\supp}{supp} 

\usepackage{graphicx}
\newcommand*{\medcap}{\mathbin{\scalebox{1}{\ensuremath{\bigcap}}}}
\newcommand*{\medcup}{\mathbin{\scalebox{1}{\ensuremath{\bigcup}}}}

%%%%%%%%%%%%%%%%%%%%%%%%%%%%%%%%%%%%%%%%%%%%%%%%%%%%%%%%%%%%%%%%%%%%%%%%%%%%%%%%%%%%%
\makeindex

\title{Sobre el espacio de funciones continuas, nunca derivables}
\author{Diego Arceo Félix}

\begin{document}

% --- Portada ---
\includepdf[pages=1]{portada.pdf}

% --- Páginas preliminares ---
\cleardoublepage
\pagenumbering{roman}  
\setcounter{page}{1}

\chapter*{Agradecimientos}

A mis papás, por su amor y apoyo incondicional, por enseñarme a siempre intentar buscar una solución a los problemas sin importar cuán grandes o imponentes se vean.
y, finalmente, por estar constantemente molestándome con la pregunta \textit{¿Y cómo va tu tesis?}.

A mi director de tesis, Darío, por mostrarme lo que verdaderamente son las matemáticas, 

\cleardoublepage
\tableofcontents

% --- Cuerpo principal ---
\cleardoublepage
\pagenumbering{arabic} % cambia a numeración arábiga
\setcounter{page}{1}

\chapter*{Introducción}
\addcontentsline{toc}{chapter}{\textbf{Introducción}}
En el estudio del cálculo real, la derivabilidad de una función es una de las primeras definiciones importantes a las que nos enfrentamos. Con ella, a raíz de sus muchas aplicaciones, llegan una  serie interminable de teoremas, ejemplos y equivalencias que nos hacen olvidarnos de una pregunta que, a pesar de que no es tan importante en la construcción de teoría, sí es igual de interesante:
\begin{center}
    ¿Existen funciones continuas que no sean derivables?
\end{center}
Un examen rápido responde positivamente la pregunta: tomando, por ejemplo, la función valor absoluto $f(x) = |x|$, podemos ver que esta es continua en $x = 0$, pero no es derivable en tal punto. Sin embargo, los más curiosos se quedarán insatisfechos con esta respuesta, y ahora buscarán responder la siguiente:
\begin{center}
    ¿Existen funciones continuas y no derivables en muchos puntos,\\ o en una infinidad de puntos?
\end{center}
La pregunta sigue siendo igual de fácil de entender, pero esta vez la respuesta no es tan inmediata. No obstante, un truco ingenioso vuelve a dar una respuesta afirmativa: consideremos la misma función anterior, $f(x) = |x|$, pero ahora restrinjamos su dominio al intervalo $[-1,1]$, y extendamos su comportamiento de manera periódica, con periodo 2, a todos los números reales. La función resultante será continua en todos los números reales, pero no será derivable en ningún entero par. Naturalmente, la siguiente pregunta a resolver es:
\begin{center}
    ¿Existen funciones continuas y no derivables en una \\infinidad no numerable de puntos?
\end{center}
O, incluso:
\begin{center}
    ¿Existen funciones continuas que no tengan puntos de derivabilidad?
\end{center}
Es aquí donde se encontraba la comunidad matemática hacia mediados del siglo XIX: muchos apostaban por el orden, la simpleza y la regularidad de las matemáticas, afirmando que era imposible que una función continua pudiera tener un comportamiento tan errático. Sin embargo, en 1872, Karl Wilhelm Theodor Weierstrass, un matemático en gran parte autodidacta, desafió la intuición de la mayoría de sus colegas al presentar el primer ejemplo de una función continua y nunca derivable.

\begin{quote}
    \emph{Me aparto con temor y horror de la lamentable plaga de funciones continuas que no tienen derivadas...}
    \vspace{-0.5cm}
    \begin{flushright}
        - Hermite en una carta a Stieltjes, 20 de mayo de 1893\footnote{Traducido del inglés, recuperado de \cite{letter_cite}.}.
    \end{flushright}
\end{quote}

Este fue el tipo de reacciones que trajo la publicación del ejemplo de Weierstrass. Con los años llegaron más, y con estos nuevos ejemplos llegó un entendimiento general sobre el tamaño del espacio de este tipo de funciones. La comunidad matemática se estaba dando cuenta de algo inesperado: no sólo había algunas cuantas funciones continuas y nunca derivables, más bien, parecía que la mayoría de las funciones continuas no tenían puntos de derivabilidad. 

Ese es, precisamente, el objetivo de esta tesis: demostrar que casi toda función real, continua en un intervalo cerrado, es nunca derivable. De manera más precisa, si definimos $\mathcal{ND}[0,1]$ como el espacio de funciones reales, continuas y nunca derivables en el intervalo $[0,1]$, demostraremos que ``casi toda'' función perteneciente a  $(\mathcal{C}[0,1],\; ||\!\cdot\!||_{\infty})$ también pertenece a $\mathcal{ND}[0,1]$. Se recomienda al lector interesado tener un conocimiento sólido en los temarios de Análisis Real, Teoría de la medida y Topología.\\

\noindent El trabajo lleva la siguiente estructura: 

En el primer capítulo expondremos algunos ejemplos de funciones continuas y nunca derivables, incluyendo la teoría necesaria para entender sus demostraciones, y llevando un orden cronológico que expone la madurez y el cambio en la estructura de las matemáticas a lo largo del tiempo.

El segundo capítulo empieza abordando algunos resultados topológicos sobre el espacio, continúa exponiendo las limitaciones que tienen las medidas en los espacios de Banach dimensionalmente infinitos, y finaliza dando la definición de ``prevalencia'', una noción que generaliza la frase ``casi todo elemento de un conjunto $A$ está en un conjunto $B$'' en este tipo de espacios.

El tercer y útltimo capítulo ataca directamente el problema central de la tesis. En éste se aterrizan las definiciones dadas en el capítulo anterior, y se presenta una forma sencilla de probar que un espacio es prevalente. Finalmente, tras exponer algunos resultados bastante técnicos, y utilizando la herramienta desarrollada anteriormente, se demuestra que casi toda función continua es nunca derivable.

%%%% Capítulo 1
\chapter{Ejemplos de funciones en \texorpdfstring{$\mathbf{\mathcal{ND}[0,1]}$}{ND[0,1]}} %% \textorpdfstring para los bookmarks
Las funciones continuas y nunca diferenciables son, además de bastante contraintuitivas, difíciles de encontrar sin suficiente herramienta matemática. Es por eso que, hace no mucho tiempo, matemáticos de renombre pensaban que las funciones continuas tenían que ser derivables (en al menos algunos puntos). Esta convicción llegó a tal punto que, en 1806, un famoso matemático francés intentó formalizar un teorema que la demostrara. Su nombre era André-Marie Ampère, y su teorema se podría interpretar como sigue (ver \cite{Ampere}):

\begin{displayquote}
    Dada una función $f:I\subseteq \mathbb{R}\to\mathbb{R}$ \textit{continua}\footnote{Cabe señalar que el concepto de continuidad, tal como lo entendemos hoy, fue formalizado después de la época de Ampère, por lo que juzgar con precisión su formulación original requiere cierto cuidado histórico.}, $f$ es derivable excepto en puntos aislados de $I$.
\end{displayquote}

El objetivo de este capítulo es demostrar analíticamente la falsedad de este teorema. Para ello, presentaremos cuatro ejemplos sencillos de funciones continuas nunca derivables, así como la herramienta necesaria para comprenderlos. Comenzaremos con la función más emblemática de este tipo, la función $W$ de Weierstrass. Ésta se publicó en 1875, y marcó un punto de inflexión en el estudio del análisis, pues, a pesar de que no fue la primera en ser descubierta (el primer descubrimiento es atribuido a Bernard Bolzano en el año 1830, publicándose éste poco menos de cien años después), tomó por sorpresa a gran parte del mundo matemático: varios, escépticos por naturaleza, dudaron de la veracidad de esta afirmación; otros, curiosos pero timoratos, veían la existencia de estas funciones como una amenaza al orden y la simpleza de las matemáticas.
\section{Función \texorpdfstring{$W$}{W} de Weierstrass}
\begin{definition}
    Dados $a,b\in\mathbb{R}$, con $a<b$, definimos 
    \[\mathcal{ND}[a,b] = \{f\in C[a,b]\,:\,\forall x\in[a,b],\; f \text{ no es derivable en } x\}.\]
\end{definition}

Los primeros tres ejemplos que daremos de funciones en $\mathcal{ND}[0,1]$ serán series de funciones; por tanto, es necesario abordar algunos resultados relevantes sobre la convergencia y continuidad de este tipo de funciones. Comenzamos con algunas definiciones y resultados de convergencia:
\begin{definition}
    Decimos que una sucesión $\{a_n\}_{n\in\mathbb{N}}$ de números reales es \textit{de Cauchy} si se cumple que para todo $\varepsilon > 0$ existe $N\in\mathbb{N}$ tal que para cualesquiera enteros $n,m> N$ se tiene que 
    \[|a_n-a_m| <\varepsilon.\]
\end{definition}

\begin{proposition}\label[proposition]{prop:conv_cauchy}
    Sea $\{a_n\}_{n\in\mathbb{N}}\subseteq\mathbb{R}$ una sucesión. $\{a_n\}_{n\in\mathbb{N}}$ converge si y sólo si es de Cauchy.
\end{proposition}

\begin{definition}\label[definition]{def:conv_func} 
    Sean $I\subseteq \mathbb{R}$, y  $\{f_n:I\to\mathbb{R}\,|\,n\in\mathbb{N}\}$ una sucesión de funciones. Dada $f:I\to\mathbb{R}$, decimos que:
    \begin{enumerate}
        \item[(1)]{$f_{n}$ \textit{converge puntualmente} a $f$ en $I$, y lo denotamos por $f_n \xrightarrow{\;p\;} f$, si se cumple que
        \[\lim_{n\to\infty}f_n(x)= f(x),\quad \text{ para todo }\, x\in I.\]}
        \item[(2)]{$f_n$ \textit{converge uniformemente} a $f$ en $I$, y lo denotamos por $f_n \xrightarrow{\;u\;} f$, si se cumple que para todo $\varepsilon > 0$ existe $N\in \mathbb{N}$ tal que para todo entero $m > N$ se tiene que
        \[|f_{m}(x) - f(x)|<\varepsilon,\quad  \text{ para todo }\, x\in I.\]}
        \item[(3)]{la serie $\sum_{n = 1}^\infty f_n$ \textit{converge (uniformemente)} a $f$, si la sucesión de sumas parciales, $S_m = \sum_{n = 1}^m f_n$, converge (uniformemente) a $f$.}
        \item[(4)]{$f_n$ \textit{es uniformemente de Cauchy}, si para todo $\varepsilon > 0$ existe $N\in\mathbb{N}$ tal que para cualesquiera enteros $n,m > N$ se tiene que 
        \[|f_{n}(x) - f_{m}(x)|<\varepsilon,\quad  \text{ para todo }\, x\in I.\]}
    \end{enumerate}
\end{definition}
\begin{proposition}
    Sean $I\subseteq \mathbb{R}$, y  $\{f_n:I\to\mathbb{R}\,|\,n\in\mathbb{N}\}$ una sucesión de funciones. $f_{n}$ converge uniformemente a $f:I\subseteq\mathbb{R}\to\mathbb{R}$ si y sólo si
    \[\lim_{n\to\infty}\; \sup_{x\in I}\,|f_n(x)-f(x)| = 0.\]
\end{proposition}

\newpage
\begin{theorem}\label[theorem]{thm:func_cauchy}
    Sean $I\subseteq \mathbb{R}$, y  $\{f_n:I\to\mathbb{R}\,|\,n\in\mathbb{N}\}$ una sucesión de funciones. $f_n$ converge uniformemente en $I$ si y sólo si $f_{n}$ es uniformemente de Cauchy en $I$.
\end{theorem}
\begin{proof} Supongamos que $f_n \xrightarrow{\;u\;} f$ en $I$.     
    Entonces para todo $\varepsilon> 0$ existe $N\in \mathbb{N}$ tal que para todo entero $m > N$ se tiene que 
    \[|f_{m}(x)-f(x)|<\frac{\varepsilon}{2}.\]
    Así, para cualesquiera enteros $n,m>N$ se tiene que
    \[|f_{n}(x)-f_{m}(x)|\leq |f_{n}(x) - f(x)| + |f_{m}(x)-f(x)| < \varepsilon.\]
    Por lo tanto, $f_n$ es uniformemente de Cauchy en $I$.\\[8pt]
    \noindent Ahora supongamos que $f_n$ es uniformemente de Cauchy en $I$.
    Observemos que, como $\{f_{n}(x)\}_{n\in\mathbb{N}}$ es una sucesión de Cauchy para todo $x\in I$, la \cref{prop:conv_cauchy} nos asegura la existencia de $f:I\to\mathbb{R}$ dada por:
    \[f(x) = \lim_{n\to\infty}f_{n}(x)\,.\]
    Veamos que $f_n \xrightarrow{\;u\;} f$. Sea $\varepsilon > 0$. Como $f_n$ es de Cauchy, existe $N\in \mathbb{N}$ tal que para cualesquiera enteros $n,m\in\mathbb{N}$ se tiene que
    \[|f_{n}(x) - f_{m}(x)|<\frac{\varepsilon}{2},\]
    para todo $x\in I$. Por otro lado, como $f_n \xrightarrow{\;p\;} f$, para todo $x\in I$ existe $m_{x}\in\mathbb{N}$ tal que $m_{x} > N$ y, si $n \geq m_x$, entonces
    \[|f_{n}(x) - f(x)|<\frac{\varepsilon}{2}.\]
    Así, dados $x\in I$ y $n > N$, tenemos que 
    \[|f_{n}(x)-f(x)|\leq |f_{n}(x) - f_{m_x}(x)| + |f_{m_{x}}(x)-f(x)| < \varepsilon.\]
    Es decir, $f_n \xrightarrow{\;u\;} f$.
\end{proof}

El siguiente teorema es de gran utilidad para demostrar la convergencia uniforme (y por tanto, la continuidad, como veremos) de series de funciones.
\newpage
\begin{theorem}[Prueba M de Weierstrass]\label[theorem]{thm:M_Weier}
     Sean $I\subseteq \mathbb{R}$, y  $\{f_n:I\to\mathbb{R}\,|\,n\in\mathbb{N}\}$ una sucesión de funciones. Supongamos que para todo $k\in\mathbb{N}$ existe $M_k\in \mathbb{R}$ tal que $|f_{k}(x)|\leq M_{k}$ para todo $x\in I$. Si la serie $\sum_{k = 0}^{\infty}M_{k}$ converge, entonces 
    \[\sum_{k = 0}^{\infty}f_{k}(x) \; \text{ converge uniformemente en  } I.\]
\end{theorem}
\begin{proof}[Demostración]
    Para cada $n\in\mathbb{N}$, sea $S_{n}:I\to\mathbb{R}$ la $n$-ésima suma parcial de la serie, es decir, $S_n(x) = \sum_{k = 0}^{n}f_{k}(x)$. Sean $n,m\in\mathbb{N}$ con $n>m$. Observemos que:
    \[\sup_{x\in I} |S_{n}(x)-S_{m}(x)| = \sup_{x\in I}\left|\sum_{k = m+1}^{n}f_{k}(x)\right|\leq \sum_{k = m+1}^{n}\sup_{x\in I}\,|f_{k}(x)| \leq \sum_{k=m+1}^{n} M_{k}\,.\]
    Como $\sum_{k = 0}^{\infty}M_k$ converge, tenemos que 
    \[\lim_{m\to \infty} \lim_{n\to \infty} \sum_{k=m+1}^{n} M_k = \lim_{m\to \infty} \lim_{n\to \infty} \left(\sum_{k=0}^{n} M_k - \sum_{k=0}^{m} M_k \right) = 0.\]
    Por lo tanto, $S_n$ es uniformemente de Cauchy. Así, por el \cref{thm:func_cauchy} tenemos que $S_{n}$ converge uniformemente.
\end{proof} 

\begin{theorem}\label[theorem]{thm:serie_continua}
    Sean $I\subseteq \mathbb{R}$, $f:I\to\mathbb{R}$, y  $\{f_n:I\to\mathbb{R}\,|\,n\in\mathbb{N}\}$ una sucesión de funciones. Si $f_{n}$ converge uniformemente a $f$ en $I$, entonces $f$ es continua en $I$.
\end{theorem}
\begin{proof}[Demostración] Tomemos $x_{0}\in I$ y $\varepsilon > 0$. Como $f_n \xrightarrow{\;u\;} f$ en $I$, existe $N\in \mathbb{N}$ tal que para cualesquiera  $n\geq N$ y $x\in I$ tenemos que
\[|f_{n}(x)-f(x)| < \frac{\varepsilon}{3}.\]
Por otro lado, como $f_{N}$ es continua en $x_0$, existe $\delta > 0$ tal que si $x\in I$ satisface que $|x-x_0|<\delta$, entonces
\[|f_N(x)-f_N(x_{0})| < \frac{\varepsilon}{3}.\]
Así, para todo $x\in I$ tal que $|x-x_0|<\delta$ se cumple que
\[|f(x)-f(x_0)|\leq|f(x)-f_N(x)|+|f_N(x) - f_N(x_0)|+|f_N(x_0)-f(x_0)| < \varepsilon,\]
es decir, $f$ es continua en $x_0$.
\end{proof}
En pocas palabras, este teorema nos dice que la convergencia uniforme preserva la continuidad de una sucesión de funciones. Escribiéndolo de una forma útil para lo que buscamos probar, tenemos el siguiente corolario:
\begin{corollary}\label[corollary]{cor:serie_continua}
    Si $I\subseteq\mathbb{R}$, y $\{f_k:I\to\mathbb{R}\,|\,n\in\mathbb{N}\}$ es una sucesión de funciones continuas tal que $\sum_{k = 0}^{\infty}f_k(x)$ converge uniformemente a $S:I\to\mathbb{R}$, entonces $S$ es continua en $I$.
\end{corollary}

A pesar de que existen varios ejemplos anteriores a este (ver \cite{Thim2003ContinuousND}), comenzaremos con la primera función que se publicó: la función $W$ de Weierstrass. Está definida como sigue (ver \cite{weier}):
\begin{definition}\label[definition]{def:W} 
    La función $\,W\!:\mathbb{R}\to\mathbb{R}$ de Weierstrass está dada por
    \[W(x) = \sum_{k=0}^{\infty}a^{k}\cos(b^{k}\pi x)\]
    donde $0<a<1$, y $b > 1$ es un entero impar tal que $ab > 1 + \frac{3\pi}{2}$.
\end{definition}

Teniendo esto en cuenta podemos abordar la demostración de que $W$ es continua y nunca diferenciable.\\

\textbf{Nota}: existe una prueba con hipótesis más débiles que las que usaremos: basta con tener $0<a<1, b>1,$ y $ab\geq 1$ (ver \cite{Hardy}), pero nosotros seguiremos una prueba bastante parecida a la que dio Weierstrass en su tiempo (ver \cite{Thim2003ContinuousND}).
\begin{figure}[ht]
    \centering
    \includegraphics[width=10cm]{Cap1/Weierstrass_035_6.png}
    \caption{Gráfica de $W$ con $a =$ 0.35 y $b = 6$ en $[-3,3]$.}
    \label{fig:graf_W}
\end{figure}
\begin{theorem}
    La función $W$ de Weierstrass es continua y nunca derivable en $\mathbb{R}$.
\end{theorem}
\begin{proof}[Demostración] Primero veamos que esta función es continua. Observemos que, como $0<a<1$, tenemos que 
\[\sum_{k = 0}^{\infty} a^{k} = \frac{1}{1-a} < \infty.\]
Además, como $|a^{n}\cos{(b^{n}\pi x)}| \leq a^{n}$, $W$ converge uniformemente por la prueba M de Weierstrass (\cref{thm:M_Weier}). Finalmente, como cada sumando \[f_{k}(x) = a^{k}\cos(b^{k}\pi x)\] es una función continua, $W$ es continua por el \cref{cor:serie_continua}.\\\\
Pasemos ahora a la parte más técnica de la prueba: veamos que $W$ no es diferenciable en ninguna parte. Sea $x_{0}\in\mathbb{R}$. Para cada $m\in\mathbb{N}$, tomemos \[\alpha_m = \begin{cases}
    \lfloor{b^{m}x_0}\rfloor,\quad \text{ si }\quad b^mx_0-\lfloor{b^{m}x_0}\rfloor \leq \frac{1}{2}, \\
    \lceil{b^{m}x_0}\rceil,\quad \text{ en otro caso.}
\end{cases}\]
en donde $\lfloor \cdot \rfloor$ y $\lceil \cdot \rceil$ son las funciones piso y techo, respectivamente.  Así, para cada $m\in \mathbb{N}$ tenemos $\alpha_{m}\in\mathbb{Z}$ tal que 
\[b^mx_{0}-\alpha_{m}\in\left(-\frac{1}{2},\frac{1}{2}\right].\]
Definimos:
\[x_{m+1} = b^mx_{0}-\alpha_{m} \quad\text{y}\quad  y_{m} = \frac{\alpha_{m}-1}{b^m}.\]
Teniendo estas definiciones, observemos que:
\[y_m-x_0 = -\frac{1+x_{m+1}}{b^m} < 0.\]
De donde $y_m < x_0$, y, si tendemos $m\to\infty$, como $x_{m+1}\in(-\frac{1}{2},\frac{1}{2}]$, tenemos que  $y_m$ converge a $x_0$ por la izquierda.\\\\
\textbf{Afirmación:}
\[\lim_{m\to\infty} \left|\frac{W(y_m)-W(x_0)}{y_m-x_0}\right| = \infty.\]
Sean 
\[S_1 = \sum_{n= 0}^{m-1}(ab)^n\frac{\cos(b^n\pi y_m)-\cos(b^n\pi x_0)}{b^n(y_m-x_0)} \;\text{ y }\;S_2 =  \sum_{n=0}^{\infty}a^{n+m}\frac{\cos(b^{n+m}\pi y_m)-\cos(b^{n+m}\pi x_0)}{y_m-x_0}.\]
Observemos que:
\begin{align*}
    \frac{W(y_m)-W(x_0)}{y_m-x_0}& = \frac{1}{y_m-x_0}\left(\sum_{n=0}^{\infty}a^n\cos(b^n\pi y_m) - \sum_{n=0}^{\infty}a^n\cos(b^n\pi x_0)\right) &&\\[8pt]
    &= \sum_{n=0}^{\infty}a^n\frac{\cos(b^n\pi y_m)-\cos(b^n\pi x_0)}{y_m-x_0} &&\\[8pt]
    & = \sum_{n= 0}^{m-1}(ab)^n\frac{\cos(b^n\pi y_m)-\cos(b^n\pi x_0)}{b^n(y_m-x_0)} &&\\
    & \quad\quad\quad + \sum_{n=0}^{\infty}a^{n+m}\frac{\cos(b^{n+m}\pi y_m)-\cos(b^{n+m}\pi x_0)}{y_m-x_0}\\
    &= S_1 + S_2.
\end{align*}
Trabajaremos con estas dos sumas de manera separada. Empecemos con $S_1$: recordando que $\sin(c+d) = \cos(c)\sin(d) + \sin(c)\cos(d)$, tenemos que 
\[\sin\left(\frac{c+d}{2}\right)\sin\left(\frac{c-d}{2}\right) = \frac{1}{2}(\cos(d)-\cos(c))\]
(ver \cref{appendix:seno_dory}). Usando esto, y que $|\frac{\sin(x)}{x}| \leq 1$ para todo $x\neq 0$, lo siguiente se cumple:
\begin{align*}
    |S_{1}| &= \left|\sum_{n= 0}^{m-1}(ab)^n\frac{\cos(b^n\pi y_m)-\cos(b^n\pi x_0)}{b^n(y_m-x_0)}\right| \\[8pt]
    & = \left|\sum_{n = 0}^{m-1}(ab)^n(-\pi)\sin\left(\frac{\pi b^n(y_m+x_0)}{2}\right)\frac{\sin\left(\frac{\pi b^n(y_m-x_0)}{2}\right)}{\frac{\pi b^n(y_m-x_0)}{2}}\right| \\[8pt] \tag{$*$}\label{eq:desigualdad_s1}
    &\leq \sum_{n=0}^{m-1}\pi(ab)^{n} = \pi\frac{(ab)^m-1}{ab-1} \leq \pi\frac{(ab)^m}{ab-1}.
\end{align*}
Ahora, para $S_2$ podemos observar lo siguiente:
\[\cos(\pi b^{m+n}y_m) = \cos\left(\pi b^{m+n}\frac{\alpha_m-1}{b^m}\right) = \cos(\pi b^n (\alpha_m-1)) = -(-1)^{\alpha_m},\]
pues $b,\alpha_m\in \mathbb{Z}$ y $b$ es impar.\\\\
Por otro lado, recordando que $\cos(c+d) = \cos(c)\cos(d) - \sin(c)\sin(d)$, tenemos que:
\begin{align*}
    \cos(\pi b^{m+n}x_0) &= \cos\left(\pi b^{m+n}\frac{\alpha_m + x_{m+1}}{b^m}\right) = \cos(\pi b^{n}(\alpha_m + x_{m+1}))\\
    &= \cos(\pi b^n\alpha_m)\cos(\pi b^n x_{m+1}) - \sin(\pi b^n\alpha_m)\sin(\pi b^nx_{m+1})\\
    &= (-1)^{\alpha_m}\cos(\pi b^n x_{m+1}) - 0\\
    &= (-1)^{\alpha_m}\cos(\pi b^n x_{m+1}).
\end{align*}
Así, podemos reescribir $S_2$ como:
\begin{align*}
    S_2 &= \sum_{n = 0}^{\infty} a^{n+m}\frac{-(-1)^{\alpha_m} - (-1)^{\alpha_m}\cos(\pi b^{n}x_{m+1})}{-\frac{1+x_{m+1}}{b^m}}\\
    &= (ab)^m(-1)^{\alpha_m}\sum_{n = 0}^{\infty}a^n\frac{1+\cos(\pi b^{n}x_{m+1})}{1+x_{m+1}}.
\end{align*}
Observemos que, como cada término en la serie anterior es positivo, entonces la serie es mayor o igual a su primer sumando, es decir,
\[\sum_{n = 0}^{\infty}a^n\frac{1+\cos(\pi b^{n}x_{m+1})}{1+x_{m+1}} \geq \frac{1+\cos(\pi x_{m+1})}{1+x_{m+1}}.\]
Ahora, como $x_{m+1}\in(-\frac{1}{2},\frac{1}{2}]$, entonces $\cos(\pi x_{m+1}) \geq 0$ y
\[\frac{1+\cos(\pi x_{m+1})}{1+x_{m+1}} \geq \frac{1}{1+\frac{1}{2}} = \frac{2}{3}.\]
Uniendo estas últimas dos desigualdades, tenemos que:
\[|S_2| \geq \left|\frac{2}{3}(ab)^m(-1)^{\alpha_m}\right|.\]
Observemos que esta última desigualdad implica la existencia de un $\eta_m\in\mathbb{R}$ tal que $|\eta_m|\geq 1$ y
\[S_2 = \frac{2}{3}(ab)^m(-1)^{\alpha_m}\eta_m.\]
Como $|\eta_m|\geq 1$, por la desigualdad \eqref{eq:desigualdad_s1} tenemos que:
\[|S_1|\leq \left|\pi\,\eta_m\frac{(ab)^m}{ab-1}\right| = \left|(-1)^{\alpha_m}\pi\,\eta_m\frac{(ab)^m}{ab-1}\right|.\]
Por lo tanto, existe $\varepsilon_m\in[-1,1]$ tal que:
\[S_1 = (-1)^{\alpha_m}\pi\,\varepsilon_m\,\eta_m\frac{(ab)^m}{ab-1}.\]
Así,
\[\frac{W(y_m)-W(x_0)}{y_m-x_0} = S_1 + S_2 = (ab)^m(-1)^{\alpha_m}\eta_m \left(\frac{2}{3} + \varepsilon_m \frac{\pi}{ab-1}\right).\]
y por tanto, como $\delta = \left(\frac{2}{3} - \frac{\pi}{ab-1}\right) > 0$, y $ab > 1$,
\[\lim_{m\to\infty} \left| \frac{W(y_m)-W(x_0)}{y_m-x_0}\right | \geq \lim_{m\to\infty}  \delta(ab)^m = \infty.\]
En síntesis, $W$ no es derivable en $x_0$.
\end{proof}
\section{La curva de Peano}

En 1890, Giuseppe Peano publicó la primer curva que rellena el cuadrado unitario (ver \cite{Thim2003ContinuousND}). Antes de llegar a la definición, debemos abordar algunos breves resultados:

\begin{theorem}
    Dado $t\in[0,1]$, existe una sucesión $\{t_n\}_{n\in\mathbb{N}}\subseteq\{0,1,2\}$ tal que
    \[t = \sum_{n = 1}^\infty \frac{t_{n}}{3^n}.\]
    A estas sucesiones la llamaremos \textit{representaciones en base 3} de $t$. 
\end{theorem}

\begin{observation*}
    Las representaciones en base 3 no son únicas para todos los $t\in[0,1]$. Por ejemplo, las sucesiones $\{0,2,2,2,\dots\}$ y $\{1,0,0,0,\dots\}$ son dos representaciones distintas de $1/3$. 
\end{observation*}

\begin{theorem}\label[theorem]{thm:basetres}
    Sean $\{t_n\},\,\{s_n\}\subseteq\{0,1,2\}$ dos representaciones en base 3 de $t\in[0,1]$ distintas. Si $m = \min\{n\in\mathbb{N}\,:\,t_n\neq s_n\}$ y $s_m<t_m$, entonces $t_m-s_m = 1$, y para cada $n>m$, $t_n = 2$ y $s_n = 0$.
\end{theorem}
\begin{proof}
    Notemos que $t_m-s_m \in \{1,2\}$ porque $\{t_m,s_m\} \subseteq \{0,1,2\}$ y $s_m<t_m$. Ahora bien, si $t_m - s_m = 2$, obtenemos el absurdo:
    \[t \geq \sum_{n = 1}^m \frac{t_n}{3^n} = \sum_{n = 1}^{m} \frac{s_n}{3^n} + \frac{2}{3^m} > \sum_{n = 1}^{m} \frac{s_n}{3^n} + \frac{1}{3^m} = \sum_{n = 0}^{m} \frac{s_n}{3^n} + \sum_{n>m}\frac{2}{3^n} \geq \sum_{n = 0}^{\infty} \frac{s_n}{3^n}  =  t.\]
    Por lo tanto, $t_m - s_m = 1$. Así,
    \[0 = \sum_{n = 1}^{\infty}\frac{t_n-s_n}{3^n} = \frac{1}{3^m} + \sum_{n>m}\frac{t_n-s_n}{3^n}.\]
    Por otra parte, si existe algún $n>m$ tal que $s_n-t_n \neq 2$, entonces
    \[\frac{1}{3^m} = \sum_{n>m}\frac{s_n-t_n}{3^n}<\sum_{n>m}\frac{2}{3^n} = \frac{1}{3^m},\]
    lo que es otra contradicción. De esta manera, como $t_n,s_n\in\{0,1,2\}$,  $s_n = 2$ y $t_n = 0$ para todo $n>m$.
\end{proof}
\begin{corollary}
    Cualquier $t \in [0,1]$ tiene, a lo sumo, dos representaciones en base 3.
\end{corollary}


\begin{definition}
    Dado $t\in\{0,1,2\}$ definimos, para $n\in\mathbb{Z},$
    \[A(n,t) = \begin{cases}
        t, &\text{ si $n$ es par},\\
        2-t, & \text{ si $n$ es impar}.
    \end{cases}\]
     Ahora, para $t\in[0,1]$, si $\{t_n\}_{n\in\mathbb{N}}$ es una representación en base 3 de $t$, entonces definimos
    \[B_n(t) = \begin{cases}
        t_1, & \text{si }n = 1 \\[6pt]
        A\!\left(\,\sum\limits_{k=1}^{n-1}t_{2k},\, t_{2n-1}\right), &\text{si } n>1.
        \end{cases}\]
    Finalmente, definimos $\varphi:[0,1]\to[0,1]$ como 
    \[\varphi(t) = \sum_{n = 1}^{\infty}\frac{B_n(t)}{3^n}.\]
    \begin{figure}[h!]
        \centering
        \includegraphics[width=0.48\linewidth]{Cap1/images/peano_phi.png}
        \caption{Gráfica de $\varphi$ en [0,1].}
    \end{figure}
\end{definition}

\begin{proposition}
    La función $\varphi$ está bien definida, es decir, no depende de la representación en base 3 que tomemos.
\end{proposition}
\begin{proof}
    Sean $\{t_n\},\{s_n\}\subseteq\{0,1,2\}$ dos sucesiones distintas pero tales que 
    \[\sum_{n = 1 }^\infty \frac{t_n}{3^n} = \sum_{n = 1 }^\infty \frac{s_n}{3^n}.\] Además, sean
    \[m = \min\{n\in \mathbb{N} : t_n \neq s_n\}, \quad t = \sum_{n=1}^{\infty} \frac{t_n}{3^n} \quad \text{y} \quad s = \sum_{n=1}^{\infty} \frac{s_n}{3^n}\]
    \medskip
    \textbf{Afirmación:} $\varphi(t) = \varphi(s)$.
    \medskip
    
    \noindent Por el \cref{thm:basetres} podemos suponer, sin pérdida de generalidad, que $t_m-s_m = 1$. Entonces $t_n = 0$ y $s_n = 2$, para todo $n>m$. Consideremos los siguientes casos:\\\\
    \textbf{Caso 1:} $m$ es par, supongamos $m = 2q$. 
    Entonces $t_1 = s_1$, y por tanto $B_1(t) = B_1(s)$. Además, para $2\leq n\leq q$:
    \[B_n(t) =  A\!\left(\,\sum\limits_{k=1}^{n-1}t_{2k},\, t_{2n-1}\right) =  A\!\left(\,\sum\limits_{k=1}^{n-1}s_{2k},\, s_{2n-1}\right) = B_n(s)\,.\]
    Por otro lado, para $n > q$:
    \begin{align*}
        B_n(t) &=  A\!\left(\,\sum\limits_{k=1}^{n-1}t_{2k},\, t_{2n-1}\right) = 
         A\!\left(\,\sum\limits_{k=1}^{q}t_{2k},\, 0\right) = A\!\left(\,\sum\limits_{k=1}^{q}s_{2k} + 1,\, 0\right)  \\[6pt]
         & = A\!\left(\,\sum\limits_{k=1}^{q}s_{2k},\, 2\right) = B_n(s) \,.
    \end{align*}
    Y por tanto $\varphi(t) = \varphi(s)$.\\\\
    \textbf{Caso 2:} $m$ es impar, supongamos $m = 2q-1$. Para $2\leq n< q$:
    \[B_n(t) =  A\!\left(\,\sum\limits_{k=1}^{n-1}t_{2k},\, t_{2n-1}\right) =  A\!\left(\,\sum\limits_{k=1}^{n-1}s_{2k},\, s_{2n-1}\right) = B_n(s)\,.\]
    Llamemos $\tau = s_2 + s_4 + \dots + s_{m-1}$, y 
    \[\delta = \begin{cases}
        1, & \text{ si $\tau$ es par, }\\
        -1, & \text{ si $\tau$ es impar.}
    \end{cases}\]Así, para $n = q$,
    \[B_q(t) =  A\!\left(\,\sum\limits_{k=1}^{q-1}t_{2k},\, t_m\right) =  A\!\left(\,\tau,\, s_m + 1\right) = B_q(s) + \delta \,.\]
    Finalmente, para $n > q$,
    \[B_n(t) =  A\!\left(\,\sum\limits_{k=1}^{n-1}t_{2k},\, t_{2n-1}\right) =  A\!\left(\,\tau,\,0\right) = 2-A\!\left(\,\tau,\,2\right)=  2 - B_n(s)= 1 - \delta\,.\]
    Por lo tanto
    \[\varphi(t) -\varphi(s) = \frac{\delta}{3^q} - \sum_{n > q} \frac{2\delta}{3^n} = \delta\left(\frac{1}{3^q} - \sum_{n > q} \frac{2}{3^n}\right) = 0\,.\]
    De donde $\varphi$ está bien definida.
\end{proof}
Con esto, podemos finalmente definir la curva $P$ de Peano:
\begin{definition}
    La curva $P$ de peano está definida como: $P:[0,1]\to[0,1]^2$ dada por
        \[P(t) = \left(\varphi(t),\,3\,\varphi\!\left(\frac{t}{3}\right)\right).\]
\end{definition}

\begin{proposition}
    La curva $P$ de Peano es sobreyectiva, es decir, rellena el cuadrado unitario.
\end{proposition}
\begin{proof}
    Sea $(\alpha,\;\beta)\in[0,1]\times[0,1]$. Además, sean $\{\alpha_n\},\{\beta_n\}\subseteq\{0,1,2\}$ representaciones en base 3 de $\alpha$ y $\beta$, respectivamente. Ahora bien, sean $t_1 = \alpha_1$, $t_2 = \beta_1$, y definamos recursivamente, para $n>2$
    \[t_n = \begin{cases}
        A\!\left(\,\sum\limits_{k=1}^{m}t_{2k-1},\, \beta_{m}\right), & n = 2m \;\,\text{ para algún } m\in\mathbb{N},\\[15pt]
        A\!\left(\,\sum\limits_{k=1}^{m}t_{2k},\, \alpha_{m+1}\right),  & n = 2m+1 \;\,\text{ para algún } m\in\mathbb{N}.
    \end{cases}\]
    Finalmente, sea \[t = \sum_{n = 1}^\infty \frac{t_n}{3^n}.\]
    \textbf{Afirmación:} $P(t) = (\alpha, \beta)$. 
    \medskip
    
    \noindent Sea $n\in\mathbb{N}$ con $n>1$, y llamemos $r = t_2 + t_4 + \dots + t_{2n-2}$. 
    Entonces \[B_n(t) = A\!\left(\,r,\, t_{2n-1}\right) =A\!\left(\,r,\, A\!\left(\,r,\, \alpha_{n}\right)\right) =  \alpha_{n},\] 
    de donde $\varphi(t) = \alpha$. Análogamente, $3\varphi(t/3) = \beta$.
\end{proof}
Como dato cultural, cabe mencionar que la curva de Peano también tiene una construcción geométrica iterativa. La demostración de que son la misma curva se sale del enfoque de este capítulo, pero se puede encontrar en \cite{peano_curve}. Su construcción geométrica se ve como sigue:
\begin{figure}[h!]
    \centering
    \includegraphics[width=0.23\textwidth]{Cap1/images/peano_1.png}  
    \hspace{0.1cm}  
    \includegraphics[width=0.23\textwidth]{Cap1/images/peano_2.png}  
    \hspace{0.1cm}  
    \includegraphics[width=0.23\textwidth]{Cap1/images/peano_3.png}
    \hspace{0.1cm}  
    \includegraphics[width=0.23\textwidth]{Cap1/images/peano_4.png}
    
    \caption{Primeras cuatro iteraciones de la construcción geométrica de la curva de Peano.}
\end{figure}

\begin{theorem}
    $\varphi$ es continua y nunca diferenciable.
\end{theorem}
\begin{proof}
    Sean $t\in[0,1)$, y $\{t_n\}_{n\in\mathbb{N}}$ una representación en base 3 de $t$. Veamos que $\varphi$ es continua en $t$ por la derecha. 
    Sin pérdida de generalidad, supongamos que $t_n$ no tiene una cola formada por puros 2.\\\\
    Sea $\varepsilon > 0$ y $n\in\mathbb{N}$ tal que $3^{-n}<\varepsilon$. Sea 
    \[\delta = \frac{1}{3^{2n}} - \sum_{k > 2n}\frac{t_k}{3^k} > 0.\]
    Observemos que 
    \[t + \delta =  \sum_{k = 1}^\infty\frac{t_k}{3^k} + \frac{1}{3^{2n}} - \sum_{k > 2n}\frac{t_k}{3^k} = \sum_{k = 1}^{2n}\frac{t_k}{3^k} + \sum_{k = 2n+1}^\infty\frac{2}{3^k}.\]
    Es decir, la representación en base 3 de $t + \delta$ coincide con la de $t$ en los primeros $2n$ términos. Por tanto, dado $s\in[t, t+\delta)$ cualquier representación en base $3$ de $s$ debe coincidir con la de $t$ en los primeros $2n$ términos.\\\\
    Llamemos $r = t_2 + t_4 + \dots + t_{2n}$. Sea $s\in[t, t + \delta)$, y tomemos $\{s_n\}_{n\in\mathbb{N}}$ alguna representación en base 3 de $s$. Como $t_k$ y $s_k$ coinciden en los primeros $2n$ térmions, entonces $B_k(t)$ y $B_k(s)$ coinciden en los primeros $n$ términos, de donde
    \begin{align*}
        |\varphi(s) - \varphi(t)| &= \left|\sum_{k = 1}^\infty\frac{B_k(s) -B_k(t)}{3^k}\right| = \left|\sum_{k = n+1}^\infty\frac{B_k(s) -B_k(t)}{3^k}\right| \\
        &\leq \sum_{k = n+1}^\infty\left|\frac{B_k(s) -B_k(t)}{3^k}\right| \leq
       \sum_{k = n+1}^\infty\frac{2}{3^k} = \frac{1}{3^n}<\varepsilon\,.
    \end{align*}
    Y por tanto $\varphi$ es continua por la derecha en $t$.\\\\
    Ahora, para ver que es continua por la izquierda en todo $t\in(0,1]$, tomamos una representación en base 3 de $t$ que tenga una infinidad de digitos no nulos, consideramos 
    \[\delta = \sum_{k = 2n+1}^\infty\frac{t_k}{3^k}\] y repetimos el argumento anterior. Por lo tanto, $\varphi$ es continua en $t$.\\\\
    Finalmente, tomemos $t\in[0,1]$, y veamos que $\varphi$ no es derivable en $t$. Llamemos 
    \[\tau_k^n = \begin{cases}
        |1-t_{2n-1}|, & \text{ si } k = 2n-1, \\
        t_k, & \text{ en otro caso},
    \end{cases}\]
    y definamos
    \[z_n = \sum_{k = 1}^\infty \frac{\tau_k^n}{3^k}.\]
    Así, $|t-z_n| = 1/  3^{2n+1}$, de donde $z_n\to t$. Por otro lado, observemos que $B_k(t) =B_k(z_n)$ para todo $k\neq n$, es decir:
    \[|\varphi(t) - \varphi(z_n)|  = \left|\frac{B_n(t) -B_n(z_n)}{3^n}\right| = \frac{1}{3^n}.\]
    Así, 
    \[\left |\frac{\varphi(t) - \varphi(z_n)}{t-z_n}\right| = 3^{n+1} \to \infty.\]
    De donde $\varphi$ no es derivable en $t$.
\end{proof}
\section{Función \texorpdfstring{$M$}{M} de McCarthy}
Saltando más de 50 años de avance matemático, pasamos al tercer ejemplo; uno que contrastará notablemente con los anteriores no sólo en la construcción en sí, sino también, y principalmente, en la simpleza de la demostración. 

En 1953, un matemático estadounidense llamado John McCarthy publicó un artículo en el que da un ejemplo de una función $M:\mathbb{R}\to\mathbb{R}$ continua y nunca diferenciable (ver \cite{McCarthy}). Está definida como sigue:

\begin{definition}
    Sea $M:\mathbb{R}\to\mathbb{R}$ definida como:
    \[M(x) = \sum_{k = 1}^{\infty}\frac{1}{2^{k}}\,g\left(2^{2^{k}}x\right),\]
    donde $g$ está dada por $g(x) = 1-|x|$, para $x\in[-2,2]$, y $g(x+4) = g(x)$ para todo $x\in \mathbb{R}$.
\end{definition}
\vspace*{-1cm}
\begin{figure}[htp]
    \centering
    \includegraphics[width=10cm]{Cap1/McCarthy.png}
    \caption{Gráfica de $M$ en $[0,1]$.}
    \label{fig:graf_M}
\end{figure}

La atracción de este ejemplo radica en la cantidad de resultados necesarios para comprender la demostración; en específico, sólo la definición y algunos resultados previamente abordados son suficientes para enfrentarse al teorema.

\begin{theorem}
    La función $M$ de McCarthy es continua y nunca diferenciable.
\end{theorem}
\begin{proof}
    Primero veamos que la función es continua en $\mathbb{R}$. 
    Como $|g(x)|\leq 1$ para todo $x\in\mathbb{R}$ y $g(0) = 1$, entonces \[\sup_{x\in\mathbb{R}} 2^{-k}g(2^{2^{k}}x) = 2^{-k}.\]  Además, por la convergencia de la serie $\sum_{k = 1}^{\infty} 2^{-k}$, una combinación del \cref{thm:M_Weier} con el \cref{thm:serie_continua} permite concluir que $M$ es continua.\\\\
    Ahora veamos que $M$ no puede ser diferenciable. Sea $x\in \mathbb{R}$. Para cada $n\in\mathbb{N}$ existen únicos $q_n\in\mathbb{Z}$ y  $y_n\in[0,4)$ tales que $2^{2^{n}}x = 4q_n + y_n$. Ahora, para cada $n\in\mathbb{N}$ definimos $h_n = \alpha_n 2^{-2^{n}}$, donde $\alpha_n$ depende de la siguiente regla:
    \begin{equation*}
        \alpha_n = 
        \begin{cases}
        1, \quad&\text{si }\; y_n\in[0,1)\cup[2,3),\\
        -1, \quad&\text{si }\; y_n\in[1,2)\cup[3,4).
        \end{cases}
    \end{equation*}
    Observemos que, dado $k > n$, como $g$ tiene periodo $4$, lo siguiente se cumple:
    \begin{equation}
        g\left(2^{2^{k}}(x+h_n)\right) - g\left(2^{2^{k}}x\right) = g\left(2^{2^{k}}x\right) - g\left(2^{2^{k}}x\right) = 0\,.
       \tag{$\ast$}\label{eq:estrella}
    \end{equation}

    Observemos que para $x\in[0,4]$, $g(x) = |x-2|-1$. Así, como $y_n +\alpha_n \in[0,4)$, tenemos que: 
    \begin{align*}
        \left|g\left(2^{2^{n}}(x+h_n)\right) - g\left(2^{2^{n}}x\right)  \right| &= \left|g\left(y_n+\alpha_n\right) - g\left(y_n\right)  \right|  \\
        &= |\,|y_n+\alpha_n-2| - |y_n-2|\,| = |\alpha_n| = 1\,.
    \end{align*}
    pues $y_n$ y $y_n+\alpha_n$ son o ambos mayores que $2$, o ambos menores que $2$.
    \noindent Afirmamos que, para $0\leq k<n$, se cumple la siguiente desigualdad:
    \begin{equation}
       \left| g\left(2^{2^{k}}(x+h_n)\right) - g\left(2^{2^{k}}x\right)\right|\leq 2^{2^{k}}2^{-2^{n}}\leq2^{-2^{n-1}}\,.
       \tag{$\ast\ast$}\label{eq:doble-estrella}
    \end{equation}
    Para demostrarla, primero observemos que para cualesquiera $a,b,c\in\mathbb{R}$,
    \[||a+b|-|a||\leq |b|.\]
    Teniendo esto, consideremos los siguientes casos:\\[6pt]
    \textbf{Caso 1:} $y_k + \alpha_n2^{2^k}2^{-2^{n}}\in[0,4]$. Entonces,
    \begin{align*}
        \left| g\left(y_k + \alpha_n2^{2^k}2^{-2^{n}}\right) - g\left(y_k\right)\right| &= 
        \left|\, |y_k -2+ \alpha_n2^{2^k}2^{-2^{n}}| - |y_k-2|\,\right| \\
        &\leq \left| \alpha_n2^{2^k}2^{-2^{n}}\right| = 2^{2^k}2^{-2^n}\,.
    \end{align*}
    \textbf{Caso 2:} $y_k + \alpha_n2^{2^k}2^{-2^{n}}\notin[0,4]$. De aquí se desprenden dos casos más, pero sólo consideraremos uno pues son análogos. Supongamos que $y_k + \alpha_n2^{2^k}2^{-2^{n}}> 4 $. Entonces, como $2^{2^k}2^{-2^n}\leq \frac{1}{2}$, tenemos que $y_k + \alpha_n2^{2^k}2^{-2^{n}}\in [4,5)$ y $y_k\in[3,4)$. Por tanto:
    \begin{align*}
        \left| g\left(y_k + \alpha_n2^{2^k}2^{-2^{n}}\right) - g\left(y_k\right)\right| &= 
        \left| g\left(y_k + \alpha_n2^{2^k}2^{-2^{n}} - 4\right) - g\left(y_k - 4\right)\right| \\
        & =  \left|\, |y_k - 4 + \alpha_n2^{2^k}2^{-2^{n}}| - |y_k - 4|\,\right| \\
        &\leq  \left| \alpha_n2^{2^k}2^{-2^{n}}\right| = 2^{2^k}2^{-2^n}\,.
    \end{align*}    
    Así,
    \begin{flalign}
        \left|\frac{M(x+h_n)-M(x)}{h_n}\right| 
        &= 2^{2^{n}}\left|\sum_{k = 1}^{\infty} \frac{1}{2^{k}}\left( g\left(2^{2^{k}}(x+h_n)\right) - g\left(2^{2^{k}}x\right)\right)\right| &&\notag\\[8pt]
        &= 2^{2^{n}}\left|\sum_{k = 1}^{n} \frac{1}{2^{k}}\left( g\left(2^{2^{k}}(x+h_n)\right) - g\left(2^{2^{k}}x\right)\right)\right| 
        &&\text{por \eqref{eq:estrella}}\notag\\[8pt]
        &\geq 2^{2^{n}}\left( 1- \left|\sum_{k = 1}^{n-1} \frac{1}{2^{k}}\left( g\left(2^{2^{k}}(x+h_n)\right) - g\left(2^{2^{k}}x\right)\right)\right|\right) 
        &&\text{por \eqref{eq:doble-estrella}}\notag\\[8pt]
        &\geq 2^{2^{n}}\left(1- 2^{n}2^{-2^{n-1}}\right) &&\notag\\[8pt]
        &= 2^{2^{n-1}}\left(2^{2^{n-1}}- 2^{n}\right) &&\notag
    \end{flalign}

    De donde 
    \[\lim_{n\to\infty}\left|\frac{M(x+h_n)-M(x)}{h_n}\right| = \infty,\]
    y por tanto $M$ no es diferenciable en $x$.
\end{proof}
\section{Funciones de Lynch}
Llegamos al último (y más bonito, en mi opinión) ejemplo de esta sección: en 1992, un matemático estadounidense llamado Mark Lynch presentó en un artículo una forma general de construir funciones continuas y nunca diferenciables (ver \cite{lynch}). Además de una construcción arbitraria, el poder de este resultado recae, como veremos en el siguiente capítulo, en las conclusiones topológicas que trae sobre el espacio $\mathcal{ND}[0,1]$.

La función de este ejemplo, a diferencia de las anteriores, no tiene una regla de correspondencia explícita. Es decir, usando argumentos topológicos, construiremos una función continua y nunca diferenciable en abstracto. Para lograrlo debemos abordar algunos resultados y definiciones de naturaleza un tanto técnica:
%%%%%%%%%%%%%%%%%%%%%%%%%%%%%%% DAR MÁS INTRODUCCION %%%%%%%%%%%%%%%%%%%%%%%%%%%%%%%%%%%%
\begin{theorem}\label[theorem]{theorem:im_compact}
    Si $a,b\in \mathbb{R}$ cumplen $a<b$ y $f : [a,b] \to \mathbb{R}$, entonces $f$ es continua si y sólo si la gráfica de $f$, $G(f) := \{(x,f(x)) : x\in [a,b]\}$, es un subconjunto compacto de $\mathbb{R}^{2}$.
\end{theorem}
\begin{proposition}\label[proposition]{prop:comp_int}
    Si $n\in \mathbb{N}$ y $\{C_m : m\in \mathbb{N}\}$ es una familia anidada (es decir, $C_{m+1} \subseteq C_m$ para todo $m\in \mathbb{N}$) formada por subconjuntos compactos y no vacíos de $\mathbb{R}^{n}$, entonces \[C =\bigcap_{m\in \mathbb{N}} C_m\] es un subconjunto compacto y no vacío de $\mathbb{R}^{n}$.
\end{proposition} 

\begin{definition} Antes de seguir conviene establecer notación auxiliar para simplificar la exposición de los resultados que vienen a continuación.

\begin{enumerate}
\item[(1)] Si $A \subseteq \mathbb{R}^{2}$ y $x\in \mathbb{R}$, entonces $A[x] = \{y\in \mathbb{R} : (x,y) \in A\}$.

\item[(2)] Para cualquier conjunto acotado y no vacío $C\subseteq \mathbb{R}$, el {\it diámetro} de $C$ es el número real $\text{diam}\, C = \sup\{|x-y| : x,y\in C\}$.

\item[(3)] Si $I \subseteq \mathbb{R}$, $\varepsilon>0$ y $f : I \to \mathbb{R}$ es una función, entonces $$N_\varepsilon(f) = \{(x,f(x)+r) : x\in I \ \wedge \ r\in (-\varepsilon,\varepsilon)\}.$$

\end{enumerate}

\end{definition} 

\begin{proposition}
    Si $C\subseteq \mathbb{R}^2$ es un compacto, entonces $C[x]$ es un compacto para todo $x\in\mathbb{R}$.
\end{proposition}

\begin{lemma}\label[lemma]{lemma:importante}
    Si $f:[a,b]\to \mathbb{R}$ es continua, entonces
\[\overbar{N_{\varepsilon}(f)} = \{(x,\,f(x)+r)\,:\, |r|\leq \varepsilon, \, x\in [a,b]\}\]
\end{lemma}

Como veremos, las rectas serán fundamentales en la construcción de las funciones de Lynch. La importancia de las siguientes proposiciones recae, más que en la relevancia teórica, en la reducción de notación. Sus demostraciones se pueden encontrar en los enunciados \ref{appendix:prop:importante}  y \ref{appendix:prop:import2} del apéndice, respectivamente.
\begin{proposition}\label[proposition]{prop:importante}
    Sean $n \in \mathbb{N}$, $a,b,m,r\in\mathbb{R}$ con $a<b$, $I = [a,b]$  y $f:I\to \mathbb{R}$ dada por $f(x) = mx + r$, donde $|m|> n$. Para todo $\varepsilon>0$ existe $0<\delta<\varepsilon$ tal que, para cada $x\in I$, existe $y\in I$ con $0<|x-y|<\varepsilon$ y que tiene la siguiente propiedad: 
\[\text{ si } p\in \overbar{N_{\delta}(f)}\,[x],\; q\in \overbar{N_{\delta}(f)}\,[y]\; \text{ entonces }\; \left|\frac{p-q}{x-y}\right| > n.\]
\end{proposition}

\begin{definition}\label[definition]{def:poligono_raro}
     Decimos que $P\subseteq \mathbb{R}^{2}$ es una \textit{poligonal} si es la gráfica de una función recta a trozos y continua en $\pi_{1}[P]$, donde $\pi_{1}:\mathbb{R}^{2}\to \mathbb{R}$ es la proyección sobre la primera coordenada.
\end{definition}

\begin{proposition}\label[proposition]{prop:import2}
    Si $a,b\in\mathbb{R}$, con $a<b$, y $f:[a,b]\to\mathbb{R}$ es una recta, entonces para cualesquiera $\varepsilon > 0$ y $n\in \mathbb{N}$ existe una poligonal $P\subseteq \mathbb{R}^{2}$ tal que: 
    \begin{enumerate}
        \item[(1)]{$P\subseteq N_\varepsilon(f)$,}
        \item[(2)]{empieza en $(a,\,f(a))$ y termina en $(b,f(b))$, y}
        \item[(3)]{es unión finita de rectas con pendiente cuyo valor absoluto es mayor a $n$.}
    \end{enumerate}
\end{proposition}

Teniendo esta última proposición enunciada, podemos finalmente pasar a construir una función de Lynch. La estrategia es la siguiente: construiremos una sucesión de compactos anidados en $\mathbb{R}^{2}$, digamos $\{C_n\}_{n\in\mathbb{N}}$, tales que, para todo $n\in \mathbb{N}$,
\begin{enumerate}[label= (\alph*)]
\setlength\itemsep{0.05cm}
    \item{$\pi_{1}[C_{n}] = [0,1]$,}
    \item{diam$\,C_{n}[x]<1/n$, para todo $x\in[0,1]$, y}
    \item{para cada $x\in[0,1]$ existe $y\in[0,1]$, con $0<|x-y|<1/n$ tal que si $p\in C_n[x]$ y $q\in C_n[y]$, entonces $|(p-q)/(x-y)|>n$.}
\end{enumerate}

\begin{example}
    Consideremos $f_{1}:[0,1]\to\mathbb{R}$ una recta con pendiente mayor a 1. Por la \cref{prop:importante}, tenemos que existe una $0 < \delta_{1} < 1$ tal que \[C_{1} = \overbar{N_{\delta_{1}}(f_1)}\] es un compacto no vacío de $\mathbb{R}^{2}$ que cumple (a) -- (c) de la estrategia anterior.
    Procederemos recursivamente para construir cada $C_n$.
    Supongamos que hemos construido $C_1,\dots,\, C_n$. Por construcción, existen $\delta_n > 0$ y una función $f_n:[0,1]\to\mathbb{R}$ recta a trozos y continua tal que
    \[C_n = \overbar{N_{\delta_n}(f_n)}.\]
    Por la \cref{prop:import2} existe una poligonal \[P = \bigcup_{i=1}^{k} R_{i} = G(f_{n+1})\] contenida en  $N_{\delta_n/2}(f_{n})$ tal que la pendiente de cada recta $R_{i}$ tiene valor absoluto mayor a $n+1$ y $\pi_{1}[P] = [0,1]$. 
    Sea \[c = \min\left\{\frac{\delta_n}{2},\, \frac{1}{n+1}\right\}.\] Aplicando la \cref{prop:importante} a cada recta $R_{i}$, tenemos que existe $0<\varepsilon_{i} < c$ tal que para todo $x\in \pi_{1}[R_{i}]$ existe un $y\in \pi_{1}[R_{i}]$ tal que $0<|x-y|<c$ y 
    \[\text{ si } p\in \overbar{N_{c}(R_{i})}\,[x]\;\text{ y }\; q\in \overbar{N_{c}(R_{i})}\,[y]\; \text{ entonces }\; \left|\frac{p-q}{x-y}\right| > n.\]
    Sea $\delta_{n+1} =  \min\,\{\varepsilon_{i}\,|\,i\leq n\}$ y definamos $C_{n+1} = \overbar{N_{\delta_{n+1}}(f_{n+1})}$. Así, $C_{n+1}$ está contenido en $C_{n}$ y cumple (a) -- (c). 
    \begin{figure}[h]
        \centering
        \includegraphics[width=0.6\linewidth]{Cap1/images/Lynch_example.png}
        \caption{Construcción recursiva de $C_n$}
        \label{fig:enter-label}
    \end{figure}
    
    \noindent Sea \[C = \bigcap_{n\in\mathbb{N}}C_n.\] Por la \cref{prop:comp_int}, $C$ es un compacto no vacío en $\mathbb{R}^{2}$. Así mismo, para $x\in [0,1]$, como  $\{C_n[x]\}_{n\in \mathbb{N}}$ es una sucesión decreciente de compactos no vacíos en $\mathbb{R}$, 
    \[C[x] = \bigcap_{n\in\mathbb{N}}C_{n}[x]\neq \emptyset,\]
    de donde $\pi_{1}[C] = [0,1]$. Ahora, como diam$\,C[x] < 1/n$ para todo $n\in\mathbb{N}$, entonces $|C[x]| = 1$. Es decir, para todo $x\in[0,1]$ existe un único $y_x\in\mathbb{R}$ tal que $C[x] = \{y_x\}.$\\\\
    Así, si $f:[0,1]\to\mathbb{R}$ es la función dada por $f(x) = y_x$, notemos que la gráfica de $f$ es $C$, que es compacto, lo cual implica, en virtud del \cref{theorem:im_compact}, que es continua. Sin embargo, $f$ no puede ser derivable en ningún $x\in[0,1]$. En efecto, dados $x\in[0,1]$ y $\varepsilon > 0$, si tomamos $N\in\mathbb{N}$ tal que $0<1/N<\varepsilon$, entonces por (c) existe $y\in[0,1]$ tal que $0<|x-y|<1/N$ y, como $f(x) \in C_N[x]$ y $f(y)\in C_N[y]$, tenemos que
    \[\left|\frac{f(x)-f(y)}{x-y}\right|> N.\]
    Por lo tanto, $f$ no puede ser derivable en ningún punto $x\in[0,1]$.
\end{example}

\begin{observation*}
    El ejemplo anterior nos demuestra que, dada una recta con pendiente mayor a 1, digamos $f_1$, existe una función $f\in\mathcal{ND}[0,1]$ tal que 
\[\sup_{x\in[0,1]}|f(x)-f_1(x)| < \delta_1\]
donde esta $\delta_1$ \textit{fue tomada} de tal forma que $C_1$ cumpliera (a) -- (c). Es decir, tenemos suficiente control sobre esta variable para poder generalizar el ejemplo hacia el siguiente lema:
\end{observation*} 
\begin{lemma}\label[lemma]{lem:a_usar_en_appendice}
    Si $f:[0,1]\to\mathbb{R}$ es una recta con pendiente mayor a 1, entonces para todo $\varepsilon > 0$ existe una función $\varphi\in\mathcal{ND}[0,1]$ tal que 
    \[\sup_{x\in[0,1]}|f(x)-\varphi(x)| < \varepsilon.\]
\end{lemma}
Usando este lema y la \cref{prop:import2}, nos podemos deshacer de la hipótesis \textit{pendiente mayor a 1}, resultando así en el siguiente teorema (su demostración se puede encontrar en el enunciado \ref{appendix:implica_densidad_lynch} del apéndice): 
\begin{theorem}\label[theorem]{thm:implica_densidad_lynch}
    Si $f:[0,1]\to\mathbb{R}$ es una recta, entonces para todo $\varepsilon > 0$ existe una función $\varphi\in\mathcal{ND}[0,1]$ tal que 
    \[\sup_{x\in[0,1]}|f(x)-\varphi(x)| < \varepsilon.\]
\end{theorem}

El siguiente corolario sirve un propósito más práctico que teórico, y su formulación se verá justificada en el siguiente capítulo:
\begin{corollary}\label[corollary]{cor:implica_densidad_lynch}
    Si $f:[0,1]\to\mathbb{R}$ es una recta, entonces para todo $\varepsilon>0$ existe una función $\varphi\in\mathcal{ND}[0,1]$ tal que \[\sup_{x\in[0,1]}|f(x)-\varphi(x)| < \varepsilon,\]
    y además $f(0) = \varphi(0)$, y $f(1) = \varphi(1).$ 
\end{corollary}

%%%% Capítulo 2
\chapter{¿Qué tan grande es \texorpdfstring{$\mathbf{\mathcal{ND}[0,1]}$}{ND[0,1]}?}
Las matemáticas nacieron a partir de un impulso (¿o necesidad?)  por medir las cosas: medir lados, áreas, volúmenes o cantidades, era el trabajo de todos los días de los primeros matemáticos. 
Este impulso primitivo está tan presente en las matemáticas ahora como nunca. Tal es así que, en este capítulo, teniendo demostrado ya que nuestro espacio de interés no es vacío, la siguiente pregunta que atacaremos es: ¿qué tan grande es este espacio? 

Uno podría pensar que la vaguedad de la pregunta da entrada a respuestas de un carácter más subjetivo, pero no es así: la naturaleza de la pregunta no nos impone ninguna restricción para responderla de manera rigurosa, existen varias formas de definirla y abordarla precisamente. 

En primera instancia podemos hablar de su cardinalidad, y, usando el \cref{thm:implica_densidad_lynch} o la función de Weierstrass, podemos demostrar que este espacio es, como mínimo, infinito\footnote{Sin embargo, la cardinalidad no es la mejor de las respuestas a este tipo de preguntas: consideremos, por ejemplo, a los naturales, $\mathbb{N}$, como subconjunto de $\mathbb{R}$; este conjunto es infinito, pero sería difícil defender que es un subconjunto \textit{grande} de $\mathbb{R}$.}. Por otro lado, existen algunas nociones topológicas que nos pueden acercar un poco a una respuesta más sensible: entre ellas destacan la densidad y las categorías de Baire (ambas abordadas en este capítulo). Por último, y más importante para el enfoque de esta tesis, está el estudio desde la teoría de la medida; es este el camino que nos puede llevar a resultados de la forma: 
\begin{center}
    ``Casi todo elemento de un conjunto $A$ está en un conjunto $B$''
\end{center}
en espacios arbitrarios. Es decir, esquivando todos los obstáculos teóricos que traen los espacios dimensionalmente infinitos, la teoría de la medida nos ayudará a darle un enfoque probabilístico al estudio del tamaño de los subconjuntos de éstos.
\section{Algunas nociones topológicas}
A pesar de que el estudio de la topología prescinda de las medidas al estudiar a los espacios, podemos usarla para acercarnos a una noción sensible sobre el tamaño de $\mathcal{ND}[0,1]$. El concepto de densidad, por ejemplo, nos habla de que un espacio está \textit{disperso} en otro. Veamos la definición formal:
\begin{definition}
    Sea $(X,\tau)$ un espacio topológico. Decimos que un subespacio $D\subseteq X$ es \textit{denso} en $X$ si se cumple que $U\cap D\neq\varnothing$, para todo $U\in\tau\setminus\{\varnothing\}$.
\end{definition}

Aterrizando la definición en algo más cercano a nuestro enfoque, tenemos la siguiente proposición que caracteriza la densidad en los espacios métricos. 
\begin{proposition}
    Sean $(X,\,d)$ un espacio métrico y $D\subseteq X$. $D$ es denso en $X$ si y sólo si se cumple que, para cualesquiera $x\in X$ y $\varepsilon > 0$, $B(x,\varepsilon)\cap D\neq \varnothing$. 
\end{proposition}

Es decir, un subespacio $D\subseteq X$ es denso en $X$ si y sólo si todo elemento de $X$ está tan cerca como queramos de un elemento de $D$. Es difícil pensar que toda función continua tenga una vecina nunca derivable tan cerca como la queramos, pero el siguiente teorema desafía esta intuición y nos ofrece otro resultado contraintuitivo:

\begin{theorem}\label[theorem]{theorem:ND_denso}El espacio $\mathcal{ND}[0,1]$ es denso en $(\mathcal{C}[0,1], \; || \cdot ||_{\infty})$.
\end{theorem}

\begin{proof}
    Sean $f\in\mathcal{C}[0,1]$  y $\varepsilon > 0$. Como $f$ es continua en $[0,1]$, entonces es uniformemente continua, por lo que existe $\delta>0$ tal que, para cualesquiera $x,y\in[0,1]$, si $|x-y|<\delta$, entonces 
    \[|f(x)-f(y)|<\frac{\varepsilon}{4}.\]
    Sean $N\in\mathbb{N}$ tal que $1/N<\delta$, y $x_k = k/N$, con $k\in\{0,\dots,N\}$. Ahora, para $k\in\{1,\dots,N\}$, llamemos $g_k:[x_{k-1},x_{k}]\to\mathbb{R}$ a la recta que une a \[(x_{k-1}, f(x_{k-1})) \text{ con } (x_{k}, f(x_{k})).\] 
    De esta forma, $\{x_0,\dots,x_N\}$, es una partición del intervalo $[0,1]$, y 
    \[|g_k(x)-f(x)|<\frac{\varepsilon}{2}\]
    para todo $x\in[x_{k-1}, x_k]$. Por el \cref{cor:implica_densidad_lynch}, para cada $k\in\{1,\dots,N\}$ existe una función $\varphi_k\in\mathcal{ND}[x_{k-1}, x_{k}]$ tal que 
    \[||\varphi_k-g_k||_{\infty}<\frac{\varepsilon}{2}, \quad \varphi(x_{k-1}) = f(x_{k-1})\,\quad\text{y}\quad \varphi(x_{k}) = f(x_{k}).\]
    Sea $\varphi:[0,1]\to\mathbb{R}$ dada por $\varphi(x) = \varphi_k(x)$, donde $x\in[x_{k-1}, x_k]$. De esta forma, \[\varphi\in\mathcal{ND}[0,1]\; \text{ y } \;||\varphi- f||_{\infty}<\varepsilon.\]
\end{proof}

En otras palabras, la gráfica de toda función continua en $[0,1]$ tiene, tan cerca como queramos, la gráfica de una función continua y nunca derivable. A pesar de que este resultado sea atractivo por lo contraintuitivo que es, sigue sin responder nuestra pregunta, pues la densidad de un subespacio depende de la topología del espacio que lo contiene. De aquí que, por ejemplo, podemos tener subespacios densos y unipuntuales:
\begin{example}
    Sea $X$ un conjunto no vacío equipado con la topología
    \[\tau = \{A\subseteq X\,|\, p\in A\}\cup\{\varnothing\}\]
    donde $p\in X$ es algún punto arbitrario. Observemos que, independientemente de la cardinalidad de $X$, el conjunto unitario $D=\{p\}$ es denso en $(X, \,\tau)$. 
\end{example}
Dicho de otro modo, la densidad de un subespacio sólo nos habla de cuán disperso está en el espacio dotado de alguna topología en particular. Ahora, esto último sólo nos expone la dificultad de atacar el problema con la densidad por sí sola. Las siguientes definiciones y resultados usan este concepto para construir más herramienta útil para llegar a una respuesta:  
\begin{definition}
    Sea $(X, \tau)$ un espacio topológico. Dado $A\subseteq X$, decimos que $A$ es \textit{denso en ninguna parte} si se cumple que
    \[\interior(\cerradura(A)) = \varnothing.\]
\end{definition}
\begin{proposition}
    Sean $(X, \tau)$  un espacio topológico y $A\subseteq X$. El conjunto $A$ es denso en ninguna parte si y sólo si $X\setminus \cerradura(A)$ es denso en $X$.
\end{proposition} 
La densidad en ninguna parte, como lo dice su nombre, es algo así como el concepto opuesto a la densidad. En cierto sentido, los conjuntos densos en ninguna parte son ``pequeños'' o ``delgados''. 

Observemos que la proposición anterior nos demuestra que, si $A$ es denso en ninguna parte en $X$, entonces todo abierto $U\subseteq X$ tiene un subconjunto abierto $V\subseteq U$ tal que $V\cap A = \varnothing$, es decir, todos los abiertos tienen partes ``separadas'' de $A$. En otras palabras, los conjuntos densos en ninguna parte no pueden ser vecindad de alguno de sus puntos.
\begin{definition}
    Sean $(X, \tau)$  un espacio topológico y $A\subseteq X$. Decimos que $A$ es \textit{de primera categoría} si es unión numerable de conjuntos densos en ninguna parte. Por el contrario, decimos que $A$ es \textit{de segunda categoría} si no es de primera categoría.
\end{definition}
\begin{example}
    El conjunto de los números racionales, $\mathbb{Q}$, como subespacio de los reales con la topología euclideana, es de primera categoría, pues cada conjunto unipuntual $\{x\}\subseteq\mathbb{R}$ es un denso en ninguna parte, y $|\mathbb{Q}| = \omega$. 
\end{example}
\begin{observation*}
    Sea $(X, \tau)$ un espacio topológico. Si $\{A_{n}\subseteq X\,|\,n\in\mathbb{N}\}$ es una sucesión de conjuntos de primera categoría, entonces $\bigcup_{n\in\mathbb{N}} A_n$ es de primera categoría. 
\end{observation*}
La observación anterior proporciona una técnica para detectar cuándo un conjunto expresado de manera adecuada es de segunda categoría. En efecto, si $(X,\tau)$ es un espacio topológico, $A$ es de segunda categoría en $X$, $B$ es de primera categoría en $X$ y $A = B\cup C$, entonces $C$ es de segunda categoría en $X$.
\newpage
\begin{lemma}
    Sea $(X, \tau)$ un espacio topológico. Si toda intersección numerable de abiertos densos resulta en un conjunto denso, entonces $X$ es de segunda categoría. 
\end{lemma}
\begin{proof}Supongamos que $X$ es de primera categoría. Entonces existe una sucesión de conjuntos densos en ninguna parte, digamos $\{C_n \,|\,n\in\mathbb{N}\}$, tal que
    \[X = \bigcup_{n\in\mathbb{N}}C_n.\]
    Sin pérdida de generalidad, supongamos que cada $C_n$ es cerrado. Así, $U_n = X\setminus C_n$ es abierto y denso, pero esto implicaría que 
    \[\bigcap_{n\in\mathbb{N}} U_n = \bigcap_{n\in\mathbb{N}} (X\setminus C_n) = X\setminus \bigcup_{n\in\mathbb{N}} C_n = \varnothing,\]
    lo que es una contradicción. Por lo tanto, $X$ es de segunda categoría.
\end{proof}
El siguiente teorema (se puede consultar una prueba diferente en \cite{Functional_analysis}) le da aún más peso a la observación anterior:
\begin{theorem}[Baire]
    Todo espacio métrico completo es de segunda categoría.
\end{theorem}
\begin{proof}
    Sea $(X,d)$ un espacio métrico completo. Demostraremos que toda intersección numerable de abiertos densos resulta en un conjunto denso. Sea $(U_n)_{n\in\mathbb{N}}$ una sucesión de abiertos densos en $X$. Llamemos
    \[ G = \bigcap_{n\in\mathbb{N}} U_n.\]
    Sean $x_0\in X$, $r>0$. Como $U_1$ es abierto y denso, entonces $B(x_0,r)\cap U_1$ es un conjunto abierto no vacío. Tomemos $x_1\in B(x_0,r)\cap U_1$ y $0<r_1<r/2$ tal que 
    \[B(x_1,r_1)\subseteq\cerradura (B(x_1,r_1)) \subseteq B(x_0,r)\cap U_1.\]
    Como $U_2$ es abierto y denso, entonces $B(x_1,r_1)\cap U_2$ es un conjunto abierto no vacío. Tomemos $x_2\in B(x_1,r_1)\cap U_2$ y $0<r_2<r_1/2$ tal que
    \[B(x_2,r_2)\subseteq\cerradura (B(x_2,r_2)) \subseteq B(x_1,r_1)\cap U_2.\]
    Así, de manera recursiva, logramos construir dos sucesiones, $\{x_n\,|\,n\in\mathbb{N}\}\subseteq X$ y $\{r_n\,|\,n\in\mathbb{N}\}\subseteq\mathbb{R}$, tales que
    \[r_{n+1}<\frac{r_n}{2},\; \text{ y }\; B(x_{n+1},r_{n+1})\subseteq\cerradura (B(x_{n+1},r_{n+1})) \subseteq B(x_{n},r_{n})\cap U_{n+1}.\]
    Observemos que, dados $n,m\in\mathbb{N}$, si $n\geq m$, entonces
    \[x_n\in \cerradura(B(x_m,r_m)),\; \text{ de donde }\; d(x_n,x_m)\leq r_m \mathbb<r/2^{m}. \tag{$*$}\label{eq:cerradura_baire}\]
    Lo que demuestra que $x_n$ es de Cauchy. Dado que $X$ es completo, existe $y\in X$ tal que $x_n\to y$. Finalmente, por \eqref{eq:cerradura_baire}, $y\in B(x_n,r_n)\subseteq U_n$ para todo $n\in \mathbb{N}$, lo que implica que $y\in B(x_0,r)\cap G$  .
\end{proof}
Entre muchos corolarios, el teorema anterior nos demuestra, por ejemplo, que los irracionales, $\mathbb{I}\subseteq\mathbb{R}$, son de segunda categoría, pues $\mathbb{R} = \mathbb{Q}\cup\mathbb{I}$, $\mathbb{R}$ es un métrico completo y $\mathbb{Q}$ es de primera categoría.

Antes de pasar al teorema importante de esta sección observemos que, usando el hecho de que toda función $f\in\mathcal{C}[0,1]$ es uniformemente continua en $[0,1]$, y dando un argumento similar al que dimos en el comienzo de la demostración del  \cref{theorem:ND_denso}, podemos demostrar el iguiente lema:
\begin{lemma}\label[lemma]{lema:poligonos_denso}
    Sea $\mathcal{P} = \{f\in\mathcal{C}[0,1] : \text{la gráfica de $f$ es una poligonal finita\footnotemark}\}$\footnotetext{Recordemos nuestra definición de poligonal, la \cref{def:poligono_raro}. Por poligonal finita, nos referimos a una formada por una cantidad finita de rectas.}. $\mathcal{P}$ es denso en $\mathcal{C}[0,1]$.
\end{lemma}
\begin{theorem}[Banach - Mazurkiewicz]
    $\mathcal{ND}[0,1]$ es de segunda categoría en $\mathcal{C}[0,1]$. 
\end{theorem}
\begin{proof} Con el fin de reducir la notación, llamemos $\mathcal{N} = \mathcal{ND}[0,1]$ y $X = \mathcal{C}[0,1]$. Sea 
\[\mathcal{A} = \{f\in X\,:\, f \text{ tiene \textit{derivada por la derecha} finita en algún } x\in[0,1]\}.\]
Observemos que $X\setminus \mathcal{A}\subseteq \mathcal{N}$. Además, por el teorema de Baire, $X$ es de segunda categoría. Por tanto, si logramos demostrar que $\mathcal{A}$ es de primera categoría, entonces se seguiría que $\mathcal{N}$ es de segunda categoría. Veamos que $\mathcal{A}$ es de primera categoría. \\\\
Para cada $n\in\mathbb{N}$, tomemos
\[E_n = \left\{f\in X : \exists x\in [0,1-1/n] \ \forall h\in (0,1/n)\left(\ \left|\frac{f(x+h)-f(x)}{h}\right|\leq n \right)\right\}.\]
Notemos que \[\mathcal{A}\subseteq \bigcup _{n\in\mathbb{N}}E_n.\]
\textbf{Afirmación:} cada $E_n$ es denso en ninguna parte.\\\\
Para ver esto, primero veamos que cada $E_n$ es cerrado. Tomemos $g\in\cerradura(E_n)$. Entonces existe una sucesión $\{f_k\}_{k\in\mathbb{N}}\subseteq E_n$ tal que $f_k \xrightarrow{\;u\;} g$. Por definición, para cada $k\in\mathbb{N}$ existe un $x_k\in[0,1-1/n]$ tal que 
\[\left|\frac{f_k(x_k+h)-f_k(x_k)}{h}\right|\leq n,\,\text{ para todo } h\in(0,1/n).\]
Como $\{x_k\}_{k\in\mathbb{N}}$ es una sucesión acotada, entonces tiene una subsucesión convergente a algún $x_0\in[0,1-1/n]$. Sin pérdida de generalidad, supongamos que toda la sucesión converge a $x_0$. Así, como $g$ es continua y $f_n \xrightarrow{\;u\;} g$, tenemos que para todo $h\in (0,1/n)$ 
\[\left|\frac{g(x_0+h)-g(x_0)}{h}\right| = \lim_{k\to \infty} \left|\frac{f_k(x_k+h)-f_k(x_k)}{h}\right|\leq n.\]
En consecuencia, $g\in E_n$ y, por lo tanto $E_n = \cerradura(E_n)$. 
Entonces, para demostrar que $E_n$ es denso en ninguna parte, basta con ver que $\cerradura(X\setminus E_n) = X$.
Observemos que 
\begin{align*}
  X\setminus E_n = \biggl\{f\in X\,:\,\forall x\in\left[0,1-1/n\right]\;\exists\, h\in(0,1/n) \text{ tal que } \left|\frac{f(x+h)-f(x)}{h}\right| > n\biggr\}.  
\end{align*}
Sean $g\in X$ y $\varepsilon> 0$. Por el \cref{lema:poligonos_denso} existe una función $f\in \mathcal{C}[0,1]$ formada por una cantidad finita de rectas con $||f-g||_{\infty}<\varepsilon/2$. Supongamos que las rectas que forman a la gráfica de $f$, $G(f)$, son $R_{1},\dots,R_{m}$.\\\\
Por la \cref{prop:import2}, para cada $R_{i}$ existe una poligonal finita $P_{i}\subseteq N_{\varepsilon/2}(R_{i})$ tal que:
\begin{enumerate}
    \item[(1)] empieza y termina en los mismos puntos que $R_i$, y
    \item[(2)] las rectas que lo forman tienen pendiente con valor absoluto mayor que $n$.
\end{enumerate}
Así
\[P = \bigcup_{i=1}^{m}P_{i}\]
es una poligonal tal que la función que lo forma, digamos $p\in \mathcal{C}[0,1]$, cumple que \\$||g-p||_{\infty} < \varepsilon$ y $\forall x\in[0,1)$ 
\[\lim_{h\to 0^+}\left|\frac{p(x+h)-p(x)}{h}\right| > n.\]
Por esta razón, $p\in X\setminus E_n$, lo cual implica que $\mathcal{N}$ es de segunda categoría en $X$.
\end{proof}
El capítulo anterior nos demostró con cuentas, resultados técnicos y notación inusual, que $\mathcal{ND}[0,1]\neq \varnothing$. Sin embargo, este último teorema, esquivando todas estas incomodidades, logró demostrar un fortalecimiento sustancial de lo anterior, pues las dos categorías de Baire son mutuamente excluyentes, y el conjunto vacío siempre es de primera categoría en cualquier espacio, es decir, el conjunto vacío nunca es de segunda categoría.
\section{Prevalencia y timidez}
% Cuando uno piensa en medidas sobre algún espacio \textit{parecido} a $\mathbb{R}^n$, usualmente, la primera medida que se le viene a la cabeza es la medida de Lebesgue. Teniendo en cuenta el problema que buscamos resolver, uno se preguntaría si podemos extender esta medida a $C[0,1]$, y con eso ver cuánto mide $\mathcal{ND}[0,1]$. Sin embargo, el siguiente argumento rompe la ilusión de resolver nuestro problema tan fácilmente, veamos:
Teniendo ya una intuición topológica sobre el tamaño de $\mathcal{ND}[0,1]$, en esta sección buscaremos acercarnos a una definición más precisa de este concepto; en primera instancia abordaremos la siguiente pregunta: ¿podemos definir una medida ``útil'' en $C[0,1]?$


Es importante remarcar que ``útil'' es la palabra clave de la pregunta, pues, en general, medidas hay en todos los conjuntos, pero no todas tienen una interpretación natural o intuitiva. 
Tomando inspiración en la medida de Lebesgue, buscamos una medida tal que, como mínimo, esté definida en todos los abiertos del espacio, sea invariante bajo traslaciones y no sea constante. 

Un primer examen nos empuja a afirmar que esta medida (o al menos una parecida) debe existir en $\mathcal{C}[0,1]$, pero el siguiente argumento rompe esta ilusión (su demostración se puede encontrar en la \cref{appendix:obs_cap_2}).
\begin{observation*}
    Sea $V$ un $\mathbb{R}$-espacio de Banach dimensionalmente infinito y separable. Si $\mu$ es una medida de borel en $V$ invariante bajo traslaciones, entonces $\mu$ es la constante 0, o todos los abiertos no vacíos en $V$ tienen medida infinita.
\end{observation*}

De nuevo nos encontramos con un obstáculo. Buscamos darle una intuición analítica al tamaño de $\mathcal{ND}[0,1]$, pero una de las herramientas más útiles para esto, las medidas, se queda corta en nuestro caso. Sin embargo, la única razón por la cual buscábamos tener una medida en $\mathcal{C}[0,1]$ era para llegar a tener una definición formal para afirmaciones de la forma \textit{casi toda función continua en [0,1] $\dots$} Es decir, para nuestros propósitos no necesitamos de todo el rigor de tal herramienta, sino que nos bastaría con abstraer algunas de sus propiedades y generalizarlas en otro concepto. 
Puntualmente, considerando a la medida de Lebesgue, algunas de las propiedades que buscamos preservar son las siguientes:

\begin{enumerate}[label = (\arabic*{})]
    \item Los conjuntos de medida cero no tienen interior,
    \item Todo subconjunto de un conjunto con medida cero también tiene medida cero,
    \item Unión numerable de conjuntos de medida cero también tiene medida cero, y
    \item Traslaciones de conjuntos de medida cero también tienen medida cero.
\end{enumerate}

En resumen, buscamos un concepto que generalice a los conjuntos que tienen medida de Lebesgue cero, y que preserve las propiedades esenciales de la medida de Lebesgue. A lo largo de este capítulo desarrollaremos la herramienta necesaria para introducir tal abstracción: la ``timidez'' y la ``prevalencia''.
\begin{notation*}
    A partir de ahora, $V$ será un espacio de Banach sobre $\mathbb{R}$, y $\mathcal{B}(V)$ será el conjunto de los borelianos de  $V$, esto es, la $\sigma$-álgebra generada por la topología de $V$. 
\end{notation*}
\begin{observation*}
    Notemos que $V$ es Haussdorff, y que las funciones suma, $+:V\times V \to V$, y producto por escalares, $\cdot:\mathbb{R}\times V\to V$, son continuas en sus dominios dotados de la topología producto\footnote{Formalmente, $V$ es un espacio vectorial topológico.}. Además, dados $v\in V$ y $S\in \mathcal{B}(V)$, como la función $f_v(x) = x-v$ es continua, entonces 
    \[(S+v) = f_{v}^{-1}[S]\in\mathcal{B}(V).\]
\end{observation*}
\begin{definition}
    Decimos que una medida de Borel $\mu$ es \textit{transversal} a $S\in\mathcal{B}(V)$ si se cumplen las siguientes condiciones:
    \begin{enumerate}
        \item[(1)] Existe un compacto $K\subseteq V$ tal que $0<\mu(K)<\infty$, y
        \item[(2)] $\mu(S + v) = 0$ para todo $v\in V$.  
    \end{enumerate}
\end{definition}
\begin{definition}\label{def:timido}
    Decimos que $B\subseteq V$ es \textit{tímido} si existen $S\in \mathcal{B}(V)$ tal que $B\subseteq S$, y una medida $\mu$ transversal a $S$. Por otro lado, decimos que $B$ es \textit{prevalente} si $V\setminus B$ es tímido.
\end{definition}
En otras palabras, los conjuntos tímidos serán nuestra generalización de los conjuntos de medida cero. Observemos que, por definición, los conjuntos tímidos cumplen la segunda de nuestras propiedades a preservar. Veamos que también satisfacen la cuarta y la primera:
\begin{proposition}
    Si $S\subseteq V$ es tímido, entonces toda traslación de $S$ también es tímida.
\end{proposition}
\begin{proposition}
    Todo conjunto tímido tiene interior vacío.
\end{proposition}
\begin{proof}
    Sean $B\subseteq V$ tímido, y supongamos que $U = \interior B \neq \varnothing$. Existen $S\in\mathcal{B}(V)$ tal que $B\subseteq S$, una medida de Borel $\mu$ transversal a $S$, y un compacto $K\subseteq V$ tal que $0<\mu(K)<\infty$.\\\\ 
    Notemos que $\{U+v\,|\,v\in V\}$ es cubierta abierta de $K$, y por tanto existen $v_1,\dots,v_n\in V$ tales que \[K\subseteq \bigcup_{i=1}^{n}\,(U+v_i).\]
    Como $\mu(K) >0$, entonces existe $i\leq n$ tal que $\mu(U+v_i) > 0$. Sin embargo, como $\mu$ es transversal a $S$, tenemos que $0<\mu(U+v_i)\leq \mu(S+v_i) = 0$, que es absurdo.
\end{proof}
\begin{corollary}
    Todo conjunto prevalente es denso.
\end{corollary}

Antes de demostrar que los conjuntos tímidos satisfacen la tercera propiedad, debemos dar una vuelta por algunos resultados sobre medidas en espacios producto:
\begin{definition}
    Sea $(X,\mathcal{A},\mu)$ un espacio de medida. Decimos que $\mu$ es \textit{$\sigma$-finita} si existe $\{A_{i}\}_{i\in\mathbb{N}}\subseteq\mathcal{A}$ tal que $\mu(A_i)<\infty$ para todo $i\in\mathbb{N}$, y 
    \[\bigcup_{i\in\mathbb{N}}A_i = X.\] 
\end{definition}
\begin{definition}
    Sean $(X,\mathcal{A})$, y $(Y,\mathcal{B})$ dos espacios medibles. Definimos la \textit{$\sigma$-álgebra producto} $\mathcal{A}\otimes\mathcal{B}$ como la $\sigma$-álgebra generada por la colección de todos los \textit{rectángulos medibles}, es decir,
    \[\mathcal{A}\otimes\mathcal{B} = \sigma(\{A\times B\,|\,A\in\mathcal{A},\,B\in\mathcal{B}\}).\]
\end{definition}
Los siguientes dos teoremas son de gran utilidad para medir en espacios producto. Sus demostraciones se pueden consultar en \cite{Measure_theory_cohn}.
\begin{theorem}\label{thm:medida_producto}
    Sean $(X,\mathcal{A},\mu)$, y $(Y,\mathcal{B},\nu)$ dos espacios de medida. Si $\mu$ y $\nu$ son $\sigma$-finitas, entonces existe una única medida $\lambda:\mathcal{A}\otimes\mathcal{B}\to[0,\infty]$ tal que 
    \[\lambda(A\times B) = \mu(A)\nu(B).\]
    para cualesquiera $A\in\mathcal{A}$ y $\,B\in\mathcal{B}$. A esta medida la llamaremos la \textit{medida producto} y la denotaremos por $\mu \times \nu$.
\end{theorem}

El siguiente teorema, un importante antecesor al teorema de Fubini, nos da suficientes herramientas para calcular integrales de funciones medibles en espacios producto. 

\begin{theorem}[Tonelli]\label{thm:Tonelli}
    Sean $(X,\mathcal{A},\mu)$, y $(Y,\mathcal{B},\nu)$ dos espacios de medida, en donde $\mu$ y $\nu$ son medidas $\sigma$-finitas. Si $f:X\times Y\to [0,\infty)$ es medible respecto a $\mu\times \nu$, y definimos $g_{x}:Y\to[0,\infty)$ y $h_{y}:X\to[0,\infty)$ como  $g_{x}(y) = f(x,y) = h_{y}(x)$, entonces:
    \begin{enumerate}
        \item[(1)] Las funciones 
        \[x\mapsto \int_{Y}g_{x}\;\text{d}\nu\quad \text{y}\quad y\mapsto  \int_{X}h_{y}\;\text{d}\mu\] son $\mathcal{A}$ y $\mathcal{B}$ medibles, respectivamente, y
        \item[(2)] $f$ cumple que  $\begin{aligned}
            \;\int_{X\times Y}f \;\text{d}(\mu\times\nu) = \int_{X}\left(\int_{Y}g_x\; \text{d}\nu\right)\text{d} \mu = \int_{Y}\left(\int_{X}h_y\; \text{d}\mu\right)\text{d} \nu.
        \end{aligned}$ 
    \end{enumerate}
\end{theorem}
Para demostrar la tercer condición de los conjuntos tímidos, antes consideraremos un caso más simple: demostraremos que unión finita de conjuntos tímidos resulta en un conjunto tímido. Para ello, teniendo dos medidas transversales a conjuntos tímidos (o a borelianos que los contienen), debemos encontrar una tercera que también sea transversal a ambos:
\begin{notation*}
    Para cada $S\subseteq V$, denotaremos $S^{+} = \{(x,y)\in V\times V\,:\,x+y\in S\}.$
\end{notation*}
Importante notar que, como la función suma de vectores es continua en $V\times V$, para todo boreliano $S\subseteq V$, $S^{+}$ es un boreliano de $V\times V$. Con esto podemos definir la medida buscada en este último propósito:
\begin{definition}
    Sean $\mu$ y $\nu$ dos medidas de Borel $\sigma$-finitas. Definimos \textit{la convolución} $\mu * \nu: \mathcal{B}(V)\to[0,\infty)$ como:
    \[(\mu * \nu) (S) = (\mu\times\nu)\,(S^{+}).\]
\end{definition}
\begin{observation*}
     Si $\mu$ y $\nu$ son dos medidas de Borel $\sigma$-finitas en $V$, entonces su convolución también es una medida de Borel en $V$. Además, por el segundo inciso del  \cref{thm:Tonelli}, $\mu * \nu = \nu *\mu$.
\end{observation*}
\begin{definition}
    Decimos que una medida $\mu:\mathcal{A}\to\mathbb{R}$ está \textit{concentrada} en $P\in \mathcal{A}$ si se cumple que, para todo $S\in \mathcal{A}$,
    \[\mu(S) = \mu(S\cap P).\]
\end{definition}
\begin{lemma}\label{lem:conv_transversal}
    Si $\mu : \mathcal{B}(V) \to [0, \infty)$ es una medida de Borel transversal a $S \in \mathcal{B}(V)$, concentrada en un compacto de medida positiva y finita, entonces para cualquier medida de Borel $\sigma$-finita $\nu : \mathcal{B}(V) \to [0, \infty)$, la convolución $\mu * \nu$ satisface que
    \[(\mu * \nu)(S+z) = 0\]
    para cualquier $z\in V$. Si, además, $\nu$ es finita y no nula, entonces $\mu * \nu$ es transversal a $S$.
\end{lemma}
\begin{proof}
    Para cada $z\in V$, tomemos $f_z:V\times V\to[0,\infty)$ dada por $f_z = \chi_{(S+z)^+}$.
    Por el \cref{thm:Tonelli},
    \begin{align*}
        (\mu * \nu) (S+z) &= (\mu\times\nu)(S^+) =  \int_{V\times V}f_z\;\text{d}(\mu\times\nu) = \int_{V}\int_{V} f_z \;\text{d}\mu \;\text{d}\nu \\[8pt]
        &= \int_{V}\int_{V} \chi_{(S+z-y)^+} \;\text{d}\mu \;\text{d}\nu = \int_{V}\mu(S+z-y)\;\text{d}\nu = 0.
    \end{align*}
    Ahora supongamos que $\nu$ es finita y no nula. Sea $K\subseteq V$ el compacto en el que está concentrada $\mu$. Observemos que    
    \[(\mu*\nu)(K) = (\mu\times\nu)(K^+) \leq \mu(V)\nu(V) = \mu(K)\nu(V).\]
    Por tanto 
    \[0<(\mu*\nu)(K)<\infty,\]
    de donde $\mu*\nu$ es transversal a $S$.
\end{proof}
Teniendo esto, y notando que todo conjunto tímido tiene asociada una medida transversal finita y concentrada en un compacto, podemos demostrar lo siguiente:
\begin{corollary}
    Unión finita de conjuntos tímidos es tímida.
\end{corollary}
Además de acercarnos un poco más a nuestro propósito, este lema responde una importante pregunta que aún no hemos planteado: ¿Quiénes son los conjuntos tímidos en $\mathbb{R}^n$? ¿Qué relación tienen con los conjuntos de medida Lebesgue cero?
\begin{theorem}\label{thm:Timidos_en_Rn}
    Un conjunto $S\subseteq \mathbb{R}^{n}$ es tímido si y sólo si tiene medida de Lebesgue cero.
\end{theorem}
\begin{proof} Sea $S\subseteq\mathbb{R}^n$ y $\lambda:\mathcal{B}(\mathbb{R}^n)\to[0,\infty)$ la medida de Lebesgue en $\mathbb{R}^n$.\\[8pt]
Supongamos que $\lambda(S) = 0$. Entonces $S$ está contenido en algún boreliano con medida de Lebesgue cero, y, como $\lambda$ es invariante bajo traslaciones, $\lambda$ es transversal a tal boreliano. Por lo tanto, $S$ es tímido.\\[8pt] 
Sea $S\subseteq \mathbb{R}^{n}$ tímido, y supongamos sin pérdida de generalidad que es boreliano. Entonces existe una medida de Borel $\mu$ transversal a $S$. Como $\mu$ es transversal a $S$, existe un compacto $K\subseteq \mathbb{R}^{n}$ de medida positiva y finita. Consideremos $\hat{\mu}:\mathcal{B}(\mathbb{R}^n)\to[0,\infty)$ dada por
    \[\hat{\mu}(A) = \mu(A\cap K) / \mu(K).\]
    Así, $\hat{\mu}$ es una medida de Borel finita y transversal a $S$. Entonces, por el \cref{lem:conv_transversal}, $(\hat{\mu} * \lambda) (S) = 0$. 
    Ahora, como $\lambda$ es invariante bajo traslaciones, por el \cref{thm:Tonelli},
    \begin{align*}
        \lambda(S) &= \lambda (S)\hat{\mu}(\mathbb{R}^n) 
        = \int_{\mathbb{R}^n}\lambda(S)\;\text{d}\hat{\mu} = \int_{\mathbb{R}^n}\lambda(S-x)\;\text{d}\hat{\mu} 
        = \int_{\mathbb{R}^n}\int_{\mathbb{R}^n}\chi_{S-x}\;\text{d}\lambda\;\text{d}\hat{\mu}\\[8pt] 
        &= \int_{\mathbb{R}^n}\int_{\mathbb{R}^n}\chi_{S^+}\;\text{d}\lambda\;\text{d}\hat{\mu} 
        = \int_{\mathbb{R}^n\times\mathbb{R}^n}\chi_{S^{+}}\;\text{d}(\hat{\mu}\times\lambda) = (\hat{\mu}*\lambda) (S) = 0.
    \end{align*}
    Es decir, $S$ tiene medida Lebesgue cero.
\end{proof}
En otras palabras, ser ``tímido'' en $\mathbb{R}^n$ no es una generalización burda de ``tener medida Lebesgue cero'', sino que es una definición equivalente. A partir de este teorema podemos generalizar a los conjuntos de medida de Lebesgue plena (es decir, aquellos cuyo complemento tiene medida de Lebesgue cero):
\begin{definition}
    Si $A\subseteq V$ es prevalente, entonces decimos que \textit{casi todo elemento de $V$ pertenece a $A$}.
\end{definition}
Existe un último teorema que debemos enunciar antes de demostrar la cuarta propiedad de los conjuntos tímidos. Éste, generalizando al  \cref{thm:medida_producto}, nos habla sobre medidas en espacios producto de tamaño arbitario:
\begin{definition}
    Sean $\mathcal{A} = \{(X_{\alpha},\,\mathcal{B}_{\alpha})\,|\, \alpha\in I\}$ una familia de espacios medibles y $X = \prod_{\alpha \in I}X_{\alpha}$. Decimos que $E\subseteq X$ es un \textit{conjunto elemental} si existe $F\subseteq I$ finito tal que 
    \[E = \bigcap_{\alpha \in F}\pi_{\alpha}^{-1}[B_{\alpha}],\]
    en donde $\pi_{\alpha}:X\to X_{\alpha}$ es la proyección sobre $X_{\alpha}$, y $B_\alpha\in\mathcal{B}_\alpha$, para cada $\alpha \in I$.  A la familia de todos estos conjuntos la denotaremos por 
    \[\mathcal{E}(X) = \{E\subseteq X\,|\, E \text{ es elemental}\}.\]
\end{definition}
\begin{theorem}\label{thm:medidas_producto_inf}
    Sea $\mathcal{A} = \{(X_{\alpha},\,\mathcal{B}_{\alpha},\,\mu_{\alpha})\,|\, \alpha\in I\}$ una familia de espacios de medida, con $\mu_{\alpha}(X_{\alpha}) = 1$ para todo $\alpha\in I$, y $\mathcal{B}_\alpha$ la $\sigma$-álgebra de Borel en $X_{\alpha}$. 
    Llamemos 
    \[X = \prod_{\alpha\in I}X_{\alpha}.\]
    Entonces existe una única medida $\mu:\sigma(\mathcal{E}(X))\to[0,1]$ tal que, para todo 
    \[E = \bigcap_{\alpha \in F}\pi_{\alpha}^{-1}[B_{\alpha}]\in\mathcal{E}(X),\]
    se cumple que
    \[\mu(E) = \prod_{\alpha \in F} \mu(B_{\alpha}).\]
\end{theorem}
A la medida de este teorema la llamaremos la \textit{medida producto de $X$}. Su demostración se puede consultar en \cite{InfiniteProducts}. Si bien este teorema es intersante por sí mismo, nosotros necesitamos que la medida producto esté definida los borelianos del espacio, no en la $\sigma$-álgebra generada por los elementales.

\begin{proposition}
    Sea $\mathcal{A} = \{(X_{\alpha},, \mathcal{B}_{\alpha}) \mid \alpha \in I\}$ una familia de espacios medibles, donde $\mathcal{B}_\alpha$ es la $\sigma$-álgebra de Borel de $X_\alpha$, y cada $X_\alpha$ es un espacio segundo numerable. Si $|I|\leq \omega$, y \[X = \prod_{\alpha \in I}X_{\alpha}\]
    está dotado de la topología producto, entonces la $\sigma$-álgebra de Borel en $X$ coincide con la $\sigma$-álgebra generada por los conjuntos elementales, es decir,
    \[\mathcal{B}(X) = \sigma(\mathcal{E}(X)).\]
\end{proposition}
\begin{proof}
    Por definición de la topología producto, cada conjunto elemental es un boreliano, de donde $\sigma(\mathcal{E}(X))\subseteq \mathcal{B}(X)$.\\\\
    Ahora, llamemos $C_\alpha$ a la base numerable de cada $X_\alpha$. Como $|I|\leq \omega$, entonces la familia
    \[\mathcal{U} = \left\{\bigcap_{\alpha\in F} \pi^{-1}[U_\alpha]\,:\,F\subseteq I \text{ finito},\; U_{\alpha}\in C_\alpha \right\}\]
    es una base numerable para $X$. Notemos que $\mathcal{U}\subseteq \mathcal{E}(X)$. Por tanto, como $\mathcal{U}$ es numerable, $\mathcal{B}(X)\subseteq \sigma(\mathcal{U})\subseteq \sigma(\mathcal{E}(X))$.
\end{proof}

Finalmente, demostremos la última de las propiedades de los conjuntos tímidos:
\begin{theorem}
    Unión numerable de conjuntos tímidos es tímido.
\end{theorem}
\begin{proof}
    Sea $\{S_n\}_{n\in\mathbb{N}}$ una sucesión de conjuntos tímidos en $V$, y llamemos $\mu_n$ a una medida transversal de cada $S_n$. Sin pérdida de generalidad, supongamos que todo $S_n$ es boreliano. \\\\
    Cada $\mu_n$ tiene asociado un compacto $K_n\subseteq V$ de medida positiva.
    Para cada $n\in \mathbb{N}$, la familia \[\mathcal{U}_n=\{B(x, 2^{-n-2}):x\in K_n\}\] es una cubierta abierta de $K_n$, que es compacto. Por tanto, para cada $n\in\mathbb{N}$ existe una subfamilia finita de $\mathcal{U}_n$ que cubre a $K_n$, llamemos $U^n_i$ a los elementos de esta familia. Observemos que 
    \[K_n = K_n\cap \left(\bigcup_{i = 1}^k\cerradura U^n_i\right) = \bigcup_{i = 1}^kK_n\cap \cerradura U^n_i.\]
    Como cada $U^n_i$ tiene diametro menor que $2^{-n}$, con esto podemos encontrar un subconjunto compacto $U_n\subseteq K_n$ de medida positiva y diámetro menor que $2^{-n}$. Ahora, notemos que para cualquier $v\in V$, la medida $\mu^v_n$ dada por
    \[\mu^v_n(A) = \mu_n(A + v)\]
    también es transversal a $S_n$. Por tanto, sin pérdida de generalidad, podemos suponer que $U_n$ contiene al origen de $V$. Además, por un argumento análogo al que dimos en la ida del \cref{thm:Timidos_en_Rn}, podemos suponer, sin pérdida de generalidad, que $\mu_n(U_n) = \mu_n(V) = 1$. \\\\
    Consideremos, para $n\in\mathbb{N}\cup\{0\}$,
    \[W_n = \prod_{m\neq n} U_m.\]
    Por el teorema de Tychonoff, cada $W_n$ es compacto. Tomemos $f:W_0\to V$ dada por
    \[f((v_n)_{n\in\mathbb{N}}) = \sum_{n=1}^{\infty}v_n.\]
    Observemos que $f$ está bien definida, pues cada $v_n$ tiene norma menor que $2^{-n}$, de donde las sumas parciales de la serie forman una sucesión de Cauchy en $V$, que es completo. Veamos que, además, $f$ es continua: sean $(v_n)_{n\in\mathbb{N}}\in W_0$, $\varepsilon>0$ y $N\in\mathbb{N}$ tal que $2^{2-N}<\varepsilon$. Definamos 
    \[C = \prod_{n=1}^{N}B\left(v_n,\, \frac{\varepsilon}{2N}\right) \times \prod_{n>N}U_n.\]
    Por definición de la topología producto, $C$ es una vecindad abierta de $(v_n)_{n\in\mathbb{N}}$. Tomemos $(w_n)_{n\in\mathbb{N}}\in C$, entonces
    \begin{align*}
        ||f(v_n)-f(w_n)|| &\leq \sum_{n = 1}^{N}\left|\left| v_n - w_n\right|\right| + \sum_{n>N}\left|\left| v_n\right|\right| + \sum_{n>N}\left|\left| w_n\right|\right| \\
        &\leq \sum_{n = 1}^{N}\frac{\varepsilon}{2N} + 2^{-N} + 2^{-N} < \varepsilon,
    \end{align*}
    de donde $f$ es continua. Así, como $W_0$ es compacto, $f[W_0]$ es compacto. \\\\
    Ahora, por el \cref{thm:medidas_producto_inf} podemos llamar $\hat{\mu}_n$ a la medida producto de $W_n$.  Finalmente, consideremos a la medida $\nu_n:\mathcal{B}(V) \to[0,1]$ dada por\footnote{Intentando ahorrar un poco de notación, en esta definición estamos pensando a cada $W_n$ como subespacio de $W_0$.}
    \[\nu_n(A) = \hat{\mu}_n(f^{-1}[A]\cap W_n).\]
    Observemos que, en particular, $\nu_0$ es algo así como la ``convolución infinita'' de todas las $\mu_n$. Además, notemos que, para cualquier $n\in\mathbb{N}$, $\hat{\mu}_0 = \mu_n\times \hat{\mu}_n$, de donde para cualquier boreliano $A \subseteq V$
    \begin{align*}
        \nu_0(A) &= \hat{\mu}_0 (f^{-1}[A])\\[7pt]
        &= (\mu_n\times \hat{\mu}_n)(f^{-1}[A])\\[7pt]
        &= (\mu_n\times \hat{\mu}_n) \left(\left\{(v_m)_{m\in\mathbb{N}} \in W_0\,:\,\sum_{m\in\mathbb{N}}v_m\in A\right\}\right) \\[7pt]
        &= (\mu_n\times \hat{\mu}_n) \left(\left\{(v,\, (v_m)_{m_\neq n})\in U_n\times W_n \,:\,v+\sum_{m\neq n}v_m\in A\right\}\right) \\[7pt]
        &= (\mu_n\times \hat{\mu}_n) \left(\left\{(v,\, w)\in U_n\times  W_n \,:\,v+f|_{W_n}(w) \in A\right\}\right)\\[7pt]
        &=\int_{U_n}\int_{W_n}\chi_{f^{-1}[A-v]\cap W_n}\; \text{d} \hat{\mu}_n\; \text{d}\mu_n\\[7pt]
        &=\int_{U_n}\hat{\mu}_n(f^{-1}[A-v]\cap W_n)\;\text{d}\mu_n\\[7pt]
        &=\int_{U_n}\nu_n(A-v)\;\text{d}\mu_n\\[7pt]    
        &= (\mu_n * \nu_n)(A),
    \end{align*}
     y por tanto, como $\mu_n$ es transversal a cada $S_n$, se tiene que $\nu_0$ es transversal a cada $S_n$. Por lo tanto, $\nu_0$ es transversal a $\bigcup_{n\in\mathbb{N}}S_n$.
\end{proof}


%%% Capítulo 3
\include{Cap3/intro_cap3}
\chapter{Prevalencia de \texorpdfstring{$\mathbf{\mathcal{ND}[0,1]}$}{ND[0,1]}}

Este tercer y último capítulo tiene un solo objetivo: demostrar la prevalencia de $\mathcal{ND}[0,1]$ en $\mathcal{C}[0,1]$. Para lograrlo, primero aterrizaremos un poco más la definición de prevalencia, encontrando una forma equivalente de entenderla que nos facilitará demostrar que algún espacio es prevalente; después atacaremos el problema de frente, introduciendo algo más de notación y algunos resultados técnicos necesarios.

\section{Sondas}
Con la herramienta expuesta hasta ahora, demostrar que un conjunto es prevalente parece ser una tarea muy complicada. En particular, encontrar medidas transversales expone un verdadero reto. El objetivo de esta sección es encontrar formas sencillas, equivalencias o implicaciones que nos faciliten lograrlo.

\begin{definition}
    Sea $\mu$ una medida de Borel sobre algún espacio topológico $(X, \tau)$. Definimos el \textit{soporte de $\mu$} como
    \[\supp(\mu) = \medcap\{X\setminus U\,:\, U\in \tau,\, \mu(U) = 0\}.\]
\end{definition}
Es decir, el soporte de una medida es la intersección de todos los conjuntos cerrados de medida plena. Por tanto, al ser la intersección de una familia de cerrados, el soporte también es un conjunto cerrado.
\begin{observation*} La siguiente es una definición equivalente:
    \[\supp(\mu) = \{x\in X\,:\, \forall V\!\in\mathcal{V}(x),\, \mu(V) \neq 0\}.\]
    En otras palabras, el soporte de una medida es el conjunto de todos los puntos cuyas vecindades siempre tienen medida positiva.
\end{observation*} 
Teniendo estas observaciones en mente, una pregunta que surge naturalmente es: ¿el soporte de una medida también es un conjunto pleno? Es decir, ¿el soporte es el conjunto cerrado y de medida plena más ``chico''? En principio parece que sí, pero no siempre es el caso, veamos:
\begin{example}
    Sean $X = \omega_1$ y $\tau_{X}$ la topología de orden. Consideremos la  función: $\mu:\mathcal{B}(X)\to\{0,1\}$ dada por
    \[\mu(A) = \begin{cases}
        1, \quad \text{ si existe $F\subseteq A$ cerrado y no acotado},\\
        0, \quad \text{ en otro caso}.
    \end{cases}\]
    \textbf{Afirmación:} $\mu$ es medida de Borel y tiene soporte vacío.
    \begin{proof}
        Sea
        \[\mathcal{M} = \{A\subseteq X\,:\, \exists F\subseteq X \text{ cerrado y no acotado tal que } F\subseteq A \text{ ó } F\subseteq X\setminus A \}.\]
        Primero demostraremos que $M$ es una $\sigma$-álgebra. Observemos que, por definición, $\mathcal{M}$ es cerrada bajo complementos, y $X\in \mathcal{M}$. Sea $\{A_n\}_{n\in\mathbb{N}}\subseteq \mathcal{M}$. Veamos que
        \[\bigcup_{n\in\mathbb{N}}A_n\in \mathcal{M}.\]
        Si existe algún $n\in \mathbb{N}$ tal que $A_n$ contiene a un cerrado y no acotado, entonces la unión de todos también lo contiene. Supongamos que, para todo $n\in\mathbb{N}$, $A_n$ no contiene un cerrado y no acotado. Como $A_n\in \mathcal{M}$, entonces existe $C_n \subseteq X\setminus A_n$ cerrado y no acotado. Observemos que
        \[\bigcap_{n\in\mathbb{N}} C_n\subseteq\bigcap_{n\in\mathbb{N}} (X\setminus A_n) = X\setminus\bigcup_{n\in\mathbb{N}}A_n.\]
        Por tanto, como cada $C_n$ es cerrado y $\mathcal{M}$ es cerrada bajo complementos, basta ver que la intersección de los $C_n$ es no acotada. Sea $\alpha\in X$. Consideremos la siguiente sucesión:
        \[C_1,\,C_2,\,\,C_1,\, C_2,\,C_3,\,\,C_1,\, C_2,\,C_3,\,C_4,\,\,\dots\]
        Llamemos $B_k$ al $k$-ésimo elemento de tal sucesión. Como cada $B_k$ es no acotado, podemos construir una sucesión creciente, digamos $\{\beta_k\}_{k\in\mathbb{N}}\subseteq X$, tal que $\alpha < \beta_k$ y $\beta_k\in B_k$, para todo $k\in\mathbb{N}$. Sea 
        \[\beta = \sup_{k\in\mathbb{N}} \beta_k.\]
        Ahora, como cada $C_n$ se repite una infinidad de veces en la sucesión $B_k$, entonces en cada $C_n$ existe una subsucesión de $\beta_k$, digamos $\beta_{n_k}$. Como $\beta_k \to \beta$, entonces $\beta_{n_k}\to \beta$, de donde $\beta\in C_n$, y en consecuencia
        \[\beta \in \bigcap_{n\in\mathbb{N}}C_n\tag{$*$}.\]
        Así, la interesección de los $C_n$ es no acotada, de donde se sigue que $\mathcal{M}$ es cerrada bajo uniones numerables. Con todo esto concluimos que $\mathcal{M}$ es una $\sigma$-álgebra.\\\\
        Ahora veamos que, en realidad, $\mathcal{M} = \mathcal{B}(X)$.\\\\
        Sea $E\subseteq X$ cerrado. Si no es acotado, entonces $E\in \mathcal{M}$, de donde $X\setminus E \in \mathcal{M}$. Por otro lado, si $E$ está acotado, existe $\alpha \in X$ tal que $x<\alpha$, para todo $x\in E$. Así, $[\alpha, \xrightarrow{})\subseteq X\setminus E$, y por tanto $X\setminus E\in \mathcal{M}$, lo cual implica que $E\in \mathcal{M}$. En consecuencia, $\mathcal{B}(X) \subseteq \mathcal{M}$.\\\\
        Ahora consideremos $E\subseteq X$ tal que existe un cerrado y no acotado $F\subseteq E$. Para cada $\alpha < \omega_1$, tomemos $f_\alpha:\alpha \to \mathbb{N}$ una función inyectiva, y definamos $g,h:E\to\mathbb{N}$ dadas por 
        \[h(\eta) = \min(F\setminus \eta) \quad\text{y}\quad g(\eta) = f_{h(\eta)}(\eta).\]
        Sea $A_n = g^{-1}[\{n\}]$. Observemos que, dado $\alpha\in\cerradura(A_n)\setminus A_n$, como $\alpha$ es límite (pues es un punto de acumulación de $A_n$), para cualquier $\alpha^\prime<\alpha$ existen $\eta,\eta^\prime\in A_n$ tales que $\alpha^\prime< \eta^\prime<\eta<\alpha $. Ahora, como $f_{h(\eta)}$ es inyectiva y $g(\eta^\prime) = n = g(\eta)$, entonces $h(\eta^\prime) < h(\eta)$, de donde \[h(\eta^\prime)\in F\cap(\eta^\prime, \eta]\subseteq F\cap(\alpha ^\prime, \alpha].\] 
        Es decir, para todo $\alpha\in\cerradura(A_n)\setminus A_n$, $\alpha \in\cerradura(F) = F$.\\[8pt]
        En consecuencia, $\cerradura(A_n) \subseteq A_n\cup F$, de donde $A_n = \cerradura(A_n)\setminus F$, lo cual implica que $A_n$ es un conjunto de Borel. Como $E = \bigcup_{n\in\mathbb{N}}A_n$, se sigue que $E$ es un conjunto de Borel, y por tanto $\mathcal{M}\subseteq \mathcal{B}(X)$.\\\\
        Ahora, teniendo esto, por $(*)$, $\mu$ es numerablemente aditiva. Esto, pues si $(A_n)_{n\in\mathbb{N}}$ es una sucesión de medibles ajenos dos a dos, entonces a lo más existe uno de medida igual a 1, es decir,
        \[|\{n\in\mathbb{N}\,:\,\mu(A_n) = 1\}|\leq 1,\]
        de donde
        \[\mu\left(\;\bigcup_{n = 1}^{\infty}A_n\right) =\sum_{n = 1}^{\infty}\mu(A_n).\]
        En síntesis, $\mu$ es medida de Borel.
        Finalmente, observemos que para cualquier $x\in X$, $V_x = [0,x+1)\in \mathcal{V}(x)$ y $\mu(V_x) = 0$, lo cual implica que $\supp(\mu) = \varnothing$.
    \end{proof}
\end{example}
\begin{proposition*}
    Si $X$ es segundo numerable, entonces el soporte de toda medida de Borel tiene medida plena.
\end{proposition*}
\begin{proof}
    Sean $(X,\tau)$ un espacio topológico y $\mu$ una medida de Borel en $X$. Como $X$ es segundo numerable, existe $\beta  \subseteq \tau$ base numerable para $X$. Observemos que 
    \[\medcup\{U\in \tau\,:\,\mu(U) = 0\} = \medcup\{U\in \beta\,:\,\mu(U) = 0\}.\]
    Así,
    \[X\setminus \supp(\mu) = \medcup\{U\in \beta\,:\,\mu(U) = 0\}.\]
    De donde $\supp(\mu)$ tiene medida plena.
\end{proof}

\begin{definition} 
    Sean $P$ un subespacio dimensionalmente finito de $V$ y $A = \{v_1,\ldots, v_n\}$ una base para $P$. Si $\lambda$ es la medida de Lebesgue en $\mathbb{R}^{n}$ y $p_A : P \to \mathbb{R}^{n}$ es la función determinada mediante 
    \[p_A(a_1v_1 +\dots+ a_nv_n) = (a_1,\dots,a_n),\] 
    entonces $\mu_A : \mathcal{B}(V) \to \mathbb{R}$ es la función dada por $$\mu_A(S) = \lambda\left(p_A[S\cap P]\right).$$
\end{definition}

\begin{proposition}\label{prop:simpl_sonda}
    Si $P$ es un subespacio dimensionalmente finito de $V$ y $A$ es una base para $P$, entonces $\mu_A$ es una medida de Borel en $V$. Además, si $B$ es una base para $P$, $S\in \mathcal{B}(V)$ y $\mu_A(S) = 0$, entonces $\mu_B(S) = 0$.
\end{proposition}
\begin{proof}
    Primero, notemos que, como $p_A$ es un isomorfismo y $P\in\mathcal{B}(V)$, entonces \[p_A[S\cap P]\in\mathcal{B}(\mathbb{R}^n),\] para todo $S\in\mathcal{B}(V)$. Además, para cualquier familia de borelianos disjuntos dos a dos, digamos $\mathcal{S}\subseteq\mathcal{B}(V)$, 
    \[\mu_A\left(\bigcup_{S\in\mathcal{S}} S\right) = \lambda\left(\bigcup_{S\in\mathcal{S}} p_A[S\cap P]\right) = \sum_{S\in \mathcal{S}}\lambda(p_A[S\cap P]) = \sum_{S\in \mathcal{S}}\mu_A(S).\]
    Finalmente, $\mu_A(\varnothing)=0$ y $\mu_A \geq 0$. Por tanto, $\mu_A$ es una medida de Borel en $V$. \\\\
    Sea $B$ una base para $P$, y supongamos que $\mu_A(S) = 0$, para algún $S\in \mathcal{B}(V)$. Como $p_A(A)$ y $p_B(A)$ son bases en $\mathbb{R}^n$, entonces existe una transformación lineal $C:\mathbb{R}^n\to \mathbb{R}^n$ tal que  \[p_B = C\circ p_A.\] 
    Entonces, por el teorema de cambio de variable,
    \begin{align*}
        \mu_B(S)& = \lambda(p_B[S\cap P])\\[6pt]
        &= \lambda(C\circ p_A[S\cap P])\\[6pt]
        &= |\det C|\cdot\lambda( p_A[S\cap P])\\[6pt] 
        &= |\det C|\cdot \mu_A(S) = 0. \qedhere
    \end{align*}
\end{proof}

\begin{definition} 
    Sean $P$ un subespacio finito-dimensional de $V$ y $T$ un subconjunto de $V$. Decimos que $P$ es una \textit{sonda} de $T$ si existen una base $A$ para $P$ y un conjunto $B\in \mathcal{B}(V)$ con $V\setminus T \subseteq B$ de tal manera que $\mu_A$ está concentrada en $P$ y es transversal a $B$.
\end{definition}

A partir de la \cref{def:timido} podemos llegar al siguiente teorema que, como veremos, será de gran utilidad al intentar demostrar que un conjunto es prevalente:

\begin{theorem}\label{thm:simp_prev}
    Si $T\subseteq V$ tiene una sonda entonces es prevalente. 
\end{theorem}

\begin{observation*}
    Una sonda para un conjunto boreliano $T$ es un subespacio de dimensión finita $P$ que está casi completamente contenido en cualquier traslación de $T$ (respecto a alguna medida de Lebesgue concentrada en $P$).    
\end{observation*}  

Teniendo esto en cuenta, veamos algunos ejemplos que muestran cómo este teorema simplifica la demostración de que un conjunto es prevalente.
\begin{example} 
    Si definimos 
    \[L^1[0,1] = \left\{f:[0,1]\to\mathbb{R}\,:\,\int_0^1|f|\;\text{d}\lambda<\infty\right\},\]
    entonces el conjunto
    \[T = \left\{f\in L^1[0,1]\,:\,\int_0^1 f\,\text{d}\lambda\neq 0 \right\}\]
    es prevalente en $L^1[0,1]$. Es decir, casi toda función Lebesgue-integrable en $[0,1]$ tiene integral no nula.
    \begin{proof}
        Sea $V = L^1[0,1]$. Primero observemos que, como el operador integral $I:L^1[0,1]\to\mathbb{R}$ dado por 
        \[I(f) = \int_0^1f\,\text{d}\lambda\]
        es continuo, entonces $T = I^{-1}[\{0\}]$ es cerrado, y por tanto boreliano.\\\\
        Encontremos una sonda para $T$: sea $P \subseteq L^1[0,1]$ el subespacio de todas las funciones constantes. Para cualquier $g\in V$, sólo existe una $c_g\in P$ tal que 
        \[\int_0^1(g + c_g)\,\text{d}\lambda= 0.\]
        Es decir, para toda $g\in V$, $P\setminus(T-g) = \{c_g\}$. Por tanto, por la observación anterior, $P$ es una sonda de $T$, de donde $T$ es prevalente.
    \end{proof}
\end{example}

\newpage
\begin{example}
    Sea $p\in (1,\infty]$. Casi toda sucesión $\{a_n\}_{n\in\mathbb{N}}\in \ell^p$ satisface que $\sum_{n=1}^{\infty}a_n$ diverge.
    \begin{proof}
        Sea $T$ el conjunto de todas las sucesiones en $\ell^p$ tales que su serie diverge.\\
        \textbf{Afirmación:} $T$ es boreliano. Llamemos $B = \ell^p\setminus T$.\\\\
        Para cada $m\in\mathbb{N}$, llamemos $S_m:\ell^p\to\mathbb{R}$ a la función dada por
        \[S_m(a) = \sum_{k = 1}^{m}a_k.\]
        Como $B$ es el conjunto de todas las sucesiones cuya serie converge, entonces para cada $a\in B$ existe $N_k\in \mathbb{N}$ tal que, para cuales quiera $n,m>N_k$,
        \[|S_n(a)-S_m(a)|<\frac{1}{k}.\]
        A partir de esto podemos ver que
        \[B = \bigcap_{k\in\mathbb{N}}\bigcup_{N\in\mathbb{N}}\bigcap_{n,m\geq N}\left\{a\in \ell^p:|S_n(a) - S_m(a)|<\frac{1}{k}\right\}.\]
        Como cada $S_m$ es continua, entonces cada conjunto de la forma 
        \[\left\{a\in \ell^p:|S_n(a) - S_m(a)|<\frac{1}{k}\right\}\]
        es abierto, y por tanto $B$ es boreliano, de donde $T$ también lo es.
        \\\\Ahora, consideremos $b_n = 1/n$. Como $p>1$, entonces $b=\{b_n\}_{n\in\mathbb{N}}\in \ell^p$.\\
        \textbf{Afirmación:} $P = \langle b\rangle$ es una sonda de $T$.\\\\
        Sea $\{a_n\}_{n\in\mathbb{N}}\in \ell^p$. Observemos que, a lo más, sólo puede existir una $c\in \mathbb{R}$ tal que
        \[\sum_{n=1}^{\infty}(a_n + cb_n)<\infty.\]
        Esto, pues si existen $c_1,c_2\in \mathbb{R}$ distintos y tales que 
        \[\sum_{n=1}^{\infty}(a_n + c_1b_n)<\infty\quad \text{y}\quad \sum_{n=1}^{\infty}(a_n + c_2b_n)<\infty,\]
        entonces se tendría que 
        \[\sum_{n=1}^{\infty}b_n = \frac{1}{c_1-c_2}\left(\sum_{n=1}^{\infty}(a_n + c_1b_n)-\sum_{n=1}^{\infty}(a_n + c_2b_n)\right) < \infty,\]
        lo que es una contradicción. Por tanto, para cada $a\in \ell^p$, a lo más existe una sucesión convergente en $a + P$. Es decir, para cada $a\in \ell^p$
        \[|P\setminus(T - a)|\leq 1.\]
        Y por lo tanto, de manera análoga al ejemplo anterior, $T$ es prevalente.
    \end{proof}
\end{example}

En este punto, ya sólo nos queda una pregunta por resolver: 
\begin{center}
    ¿Es $\mathcal{ND}[0,1]$ un conjunto prevalente en $(\mathcal{C}[0,1],\; ||\cdot||_{\infty})$?
\end{center}
O, dicho de una forma más dramática,
\begin{center}
    ¿Casi toda función continua en $[0,1]$ es nunca derivable?
\end{center}
\section{Prevalencia de \texorpdfstring{$\mathbf{\mathcal{ND}[0,1]}$}{ND[0,1]}}

En esta última sección nos dedicaremos a lograr un solo objetivo: demostrar la prevalencia de $\mathcal{ND}[0,1]$ en $\mathcal{C}[0,1]$. Como se expuso en la sección anterior, existen ejemplos de espacios para los que no es tan difícil encontrar una sonda. Sin embargo, para $\mathcal{ND}[0,1]$ no es el caso, viéndonos así obligados a introducir más notación y algunos resultados bastante técnicos.


Complicando aún más nuestro objetivo, en 1936 Stephan Mazurkiewicz probó que $\mathcal{ND}[0,1]$ no es un subconjunto de Borel de $C[0,1]$ (ver \cite{Mauldin_Non_Borel}). Por esta razón, trabajaremos con otro conjunto un poco más fácil de manejar:
\begin{definition}
    Sea $M>0$. Decimos que una función $f\in C[a,b]$ es \textit{$M$-Lipschitz en $x\in[a,b]$} si para todo $y\in[a,b]$ se cumple que:
    \[|f(x)-f(y)|\leq M|x-y|.\]
    Por otro lado, definimos $\mathcal{NL}_M[a,b]$ como el conjunto de todas las funciones nunca $M$-Lipschitz en $[a,b]$, es decir,
    \[\mathcal{NL}_M[a,b] = \{f\in C[a,b]:\forall x\in[a,b],\; f \text{ no es $M$-Lpischitz en } x\}.\]
    Finalmente, definimos
    \[\mathcal{NL}[a,b] = \bigcap_{n\in\mathbb{N}}\mathcal{NL}_n[a,b].\]
\end{definition}
\begin{observation*}
    $\mathcal{NL}[a,b]$ es el conjunto de las funciones nunca-$M$-Lipschitz en $[a,b]$, para toda constante $M>0$.  
\end{observation*}

Encontremos algunas propiedades de estos espacios:
\begin{proposition}
    Si $M>0$, entonces
    \begin{enumerate}[label = (\arabic*{)}]
        \item $\mathcal{NL}_M[a,b]$ es abierto en $C[a,b]$,
        \item $\mathcal{NL}[a,b]$ es un conjunto boreliano, y
        \item $\mathcal{NL}[a,b] \subseteq \mathcal{ND}[a,b]$.
    \end{enumerate}
\end{proposition}
\begin{proof}\mbox{}\\*
    1) Llamemos \[F = C[a,b]\setminus \mathcal{NL}_M[a,b]\] y tomemos $f \in \cerradura (F)$. Entonces existe $\{f_n\}_{n\in\mathbb{N}}\subseteq F$ tal que $f_n\xrightarrow{u} f$. Como $f_n\notin \mathcal{NL}_M[a,b]$, entonces para cada $n\in\mathbb{N}$ existe $x_n\in [a,b]$ tal que $f_n$ es $M$-Lipschitz en $x_n$. Es decir, para toda $y\in[a,b]$, 
    \[|f_n(x_n)-f_n (y)|\leq M|x_n-y|.\]
    Ahora, como $\{x_n\}_{n\in\mathbb{N}}\subseteq [a,b]$, entonces, por el teorema de Bolzano-Weierstrass, tiene una subsucesión convergente, digamos $x_{n_k}\to x$. Observemos que, para cualesquiera $k\in \mathbb{N}$ y $y\in[a,b]$,
    \begin{align*}
        |f(x)-f(y)| \leq& \, |f(x)-f(x_{n_k})| + |f(x_{n_k})-f_{n_k}(x_{n_k})| \\
        &\quad + |f_{n_k}(x_{n_k})-f_{n_k}(y)|+ |f_{n_k}(y)-f(y)|\\[6pt]
        \leq&\, |f(x)-f(x_{n_k})| + 2||f-f_{n_k}|| + M|x_{n_k} - x| + M|x - y|.
    \end{align*}
    Por tanto, tomando límites, tenemos que 
    \[|f(x)-f(y)| \leq M|x-y|\]
    para todo $y\in[a,b]$. Es decir, $f$ es $M$-Lipschitz en $x\in[a,b]$, de donde $f\in F$.\\\\
    2) Por el inciso anterior, cada $\mathcal{NL}_n[a,b]$ es abierto y, por lo tanto, $\mathcal{NL}[a,b]$ es un conjunto boreliano.\\\\
    3) Sean $f\in \mathcal{NL}[a,b]$ y $x\in[a,b]$. Veamos que $f$ no es derivable en $x$. Como $f$ es continua en $[a,b]$, existe $L>0$ tal que, para toda $y\in[a,b]$ 
    \[|f(x)-f(y)|\leq L.\]
    Por otro lado, como $f\in \mathcal{NL}[a,b]$, entonces, para cada $n\in\mathbb{N}$, existe $y_n\in[a,b]$ tal que
    \[n|x-y_n|<|f(x)-f(y_n)|\leq L.\]
    Por lo tanto $y_n \to x$, de donde $f$ no es derivable en $x$, es decir, $f\in \mathcal{ND}[a,b]$. 
\end{proof}

Considerando el tercer inciso de la proposición anterior, una pregunta que surge naturalmente es la siguiente:
\begin{center}
    ¿Existen funciones continuas y no derivables que sean\\Lipschitz en algún punto de su dominio?
\end{center}
O, más aún,
\begin{center}
    ¿Existen funciones Lipschitz continuas que no sean derivables?
\end{center}
Gracias al teorema de Rademacher (ver \cite{rademacher}) sabemos que la respuesta es afirmativa únicamente en el sentido local. En el caso undimensional, este teorema se enuncia de la siguiente forma:
\begin{theorem}
    Si $U\subseteq\mathbb{R}$ es abierto y $f:U\to\mathbb{R}$ es Lipschitz continua, entonces es derivable en casi todo punto de $U$.
\end{theorem}
Ahora bien, un ejemplo de una función $T\in\mathcal{ND}[0,1]$ que es Lipschitz en algunos puntos es la siguiente (ver \cite{Takagi}):
\begin{theorem}
    La función $T:[0,1]\to\mathbb{R}$ de Waerden-Takagi, definida como 
    \[T(x) = \sum_{n = 0}^\infty \frac{\text{dist}(2^nx,\mathbb{Z})}{2^n},\]
    es continua, nunca derivable, y Lipschitz en algunos puntos, es decir,
    \[T\in\mathcal{ND}[0,1]\setminus\mathcal{NL}[0,1].\]
\end{theorem}


%%%%%%%%% Comentando ejemplo de construcción (reemplazar por la función de Takagi)
\iffalse
La respuesta es afirmativa en el caso local, es decir, existen funciones continuas y no derivables que son Lipschitz en algunos puntos. Sin embargo, no pueden existir funciones continuas no derivables que sean Lipschitz en todo su dominio, pues el conjunto de los puntos de no-derivabilidad de cualquier función Lipschitz tiene medida cero. La prueba de este hecho se sale del enfoque del capítulo, pero se puede consultar en \cite{rademacher}. 
\begin{example}
    Tomemos una función $g:[0,1]\to\mathbb{R}$ $M$-Lipschitz en $0$, para alguna constante $M>0$, y tal que, o bien no es derivable en $0$, o su derivada en $0$ no es nula. Sin pérdida de generalidad, supongamos que
    \[|g(x)-g(0)|<Mx,\]
    para todo $x\in(0,1]$. Además, tomemos $\{a_n:n\in\mathbb{N}\}$ alguna sucesión estrictamente decreciente tal que $a_1 = 1$ y $a_n\to 0$, y definamos
    \[\varepsilon_n = \min\left\{\frac{Ma_k - |g(a_k)-g(0)|}{2}\,: k\in\{n, n+1\}\right\}>0.\]
    Ahora definamos $\{b_n:n\in\mathbb{N}\}\subseteq\mathbb{R}^2$ como sigue:
     \[b_n = \begin{cases}
         (a_n,\,g(a_n)),\quad&\text{si $n$ es impar},\\
         (a_n,\,g(0)),\quad&\text{si $n$ es par.}
     \end{cases}\]
     Con esto, llamemos $P_n$ al segmento que une a los puntos $b_n$ y $b_{n+1}$. Nuestra sucesión de segmentos se podría ver algo así:
     \begin{figure}[H]
        \centering
        \includegraphics[width=9cm]{Cap3/NL_example.png}
        \caption{Sucesión de los segmentos $P_n$.}
        \label{fig:sucesion_ejemplo}
    \end{figure}
    \noindent Sea $g_n:[a_{n+1}, a_n]\to\mathbb{R}$ la recta tal que $G(g_n) = P_n$. Usando el \cref{thm:implica_densidad_lynch} podemos construir una sucesión de funciones continuas y no derivables en cada intervalo $[a_{n+1}, a_n]$, digamos $\{f_n\in\mathcal{ND}[a_{n+1}, a_n]\,:\,n\in\mathbb{N}\}$,
     tal que, para cada $n\in\mathbb{N}$, 
     \[||f_n-g_n||<\varepsilon_n.\]
     Más aún, usando ese mismo teorema podemos demostrar que, en realidad, se puede exigir que cada $f_n$ coincida con $g_n$ en los extremos de su dominio, es decir,
     \[g_n(a_n) =f_n(a_n)\quad \text{y}\quad g_n(a_{n+1}) = f_n (a_{n+1}).\]
     Teniendo eso, y notando que 
     \[\bigcup_{n\in\mathbb{N}} [a_{n+1}, a_n] = (0,1],\]
     podemos definir $f:(0,1]\to\mathbb{R}$ como 
     $f(x) = f_n(x)$, donde $n\in\mathbb{N}$ es tal que $x\in [a_{n+1}, a_n]$. De este modo, siguiendo un argumento análogo al que dimos en el \cref{theorem:ND_denso}, podemos ver que $f\in\mathcal{ND}(0,1]$. \\[8pt]
     Ahora veamos que $f$ es $M$-Lipschitz en $0$. Para esto, primero notemos que el conjunto
     \[\{(x,y)\in\mathbb{R}^2\,:\,0\leq x\leq 1, \;|y-g(0)|<Mx\}\]
     es convexo, y por tanto, como cada $g_n$ es una recta mayor o menor igual que $g(0)$, $|g_n(x)-g(0)|+\varepsilon_n<Mx$
     para todo $x\in[a_{n+1},\,a_n]$. Así, 
     \[|f_n(x)-g(0)|\leq |g_n(x) -g(0)| + \varepsilon_n<Mx,\] para todo $x\in[a_{n+1},a_n]$.
     Por tanto, definiendo $f(0) = 0$, podemos extender continuamente a $f$ a todo el intervalo $[0,1]$. Finalmente, como $f(a_{2n-1}) = g(a_{2n-1})$ y $f(a_{2n}) = g(0)$ para todo $n\in\mathbb{N}$, y $g'(0) \neq 0$, entonces $f$ no es derivable en $0$. En síntesis, $f\in\mathcal{ND}[0,1]\,\setminus\,\mathcal{NL}[0,1]$.
\end{example}
\fi
%%%%%%%% Termina comentario

Regresando al objetivo de este capítulo, consideremos los siguientes enunciados que, siendo de carácter más técnico, nos ayudarán a demostrar la existencia de una sonda 2-dimensional para $\mathcal{NL }[0,1]$.

\begin{lemma}
    Sean $m\in \mathbb{N}$ e $I\subseteq [0,1]$ un intervalo cerrado de longitud $2^{-m}$. Entonces, para cualesquiera $f:I\to \mathbb{R}$ continua, $\theta\in[0,2\pi)$, y $j\in\mathbb{N}$,
    \[\sup f[I] - \inf f[I]\geq 2^m\pi\int_{I}f(x)\cos (2^{m+j}\pi x + \theta)\,\text{d}x.\]
\end{lemma}
\begin{proof}
    Sean $m\in \mathbb{N}$, $\theta\in[0,2\pi)$ e $I\subseteq [0,1]$ un intervalo cerrado de longitud $2^{-m}$.  Observemos que, para cualquier $j\in\mathbb{N}$,
    \[\int_{I}\cos (2^{m+j}\pi x + \theta)\,\text{d}x = 0.\]
    Por tanto, sumar constantes a $f$ no afecta la desigualdad, de donde podemos suponer, sin pérdida de generalidad, que \[\sup f[I] = - \inf f[I].\] 
    Llamemos $K = \sup f[I]$. Así, si $I = [a, a+2^{-m}]$,
    \begin{align*}
        2^m\pi\int_{I}f(x)\cos (2^{m+j}\pi x + \theta)\,\text{d}x &\leq 2^mK\pi\int_{I}|\cos (2^{m+j}\pi x + \theta)|\,\text{d}x\\[6pt]
        & = 2^mK\pi\int_{a}^{a + 2^{-m}}|\cos (2^{m+j}\pi x + \theta)|\,\text{d}x\\[6pt]
        & = 2^{-j}K\int_{0}^{2^j\pi}|\cos (u)|\,\text{d}u\\[8pt]
        &= 2K.
    \end{align*}
    Y esto demuestra la desigualdad.
\end{proof}

\begin{lemma}\label{lem:tecnico_prev_ND}
    Si $g,h:[0,1]\to\mathbb{R}$ están dadas por
    \[g(x) = \sum_{k = 1}^\infty\frac{1}{k^2}\cos(2^k\pi x)\quad\text{y}\quad
    h(x) = \sum_{k = 1}^\infty\frac{1}{k^2}\sin(2^k\pi x), \]
    entonces son continuas y, además, existe $c>0$ tal que, para todo intervalo cerrado $I\subset[0,1]$ de longitud $\varepsilon \leq 1/2$, y para cualesquiera $\alpha, \beta\in\mathbb{R}$ se cumple que:
    \[\sup_{x\in I}(\alpha g(x) + \beta h(x)) - \inf_{x\in I}(\alpha g(x) + \beta h(x))\geq \frac{c\sqrt{\alpha^2 + \beta^2}}{\log^2(\varepsilon)}.\]
\end{lemma}
\begin{proof}
    Observemos que, por la prueba M de Weierstrass (\cref{thm:M_Weier}), $g$ y $h$ son continuas. Sean  $\alpha, \beta\in\mathbb{R}$, $r = \sqrt{\alpha^2 + \beta^2}$, y $f = \alpha g + \beta h$. Si $r = 0$, entonces la desigualdad se cumple. Supongamos que $r> 0$. Existe $\theta \in [0,2\pi]$ tal que
    \[\cos(\theta) = \alpha / r, \;\text{y} \; \sin(\beta) = \beta / r.\]
    Entonces
    \[f(x) = \sum_{k = 1}^\infty\frac{1}{k^2}\left(\alpha\cos(2^k\pi x) + \beta\sin(2^k\pi x)\right) = r\sum_{k = 1}^\infty\frac{1}{k^2}\cos(2^k\pi x + \theta).\]
    Sea $I\subset[0,1]$ un intervalo cerrado de longitud $\varepsilon\leq 1/2$. Tomemos $m\in \mathbb{N}$ tal que $2^{-m}<\varepsilon\leq 2^{1-m}$, y $J\subseteq I$ un intervalo cerrado de longitud $2^{-m}$. Entonces, por el lema anterior, para cualquier $j\in\mathbb{N}$
    \[\sup f[I] - \inf f[I]\geq\sup f[J] - \inf f[J]\geq 2^m\pi\int_{J}f(x)\cos (2^{m+j}\pi x + \theta)\,\text{d}x.\]
    Por el teorema de la convergencia dominada de Lebesgue, 
    \begin{align*}
        \int_{J}f(x)\cos (2^{m+j}\pi x + \theta)\,\text{d}x &= r\int_{J}\sum_{k = 1}^\infty\frac{1}{k^2}\cos(2^k\pi x + \theta)\cos (2^{m+j}\pi x + \theta)\,\text{d}x\\
        &= r\sum_{k = 1}^\infty\frac{1}{k^2}\int_{J}\cos(2^k\pi x + \theta)\cos (2^{m+j}\pi x + \theta)\,\text{d}x.
    \end{align*}
    Observemos que \[\cos(2^k\pi x + \theta)\cos (2^{m+j}\pi x + \theta) =\frac{\cos((2^{m+j}-2^k)\pi x) + \cos((2^{m+j}+2^k)\pi x + 2
    \theta)}{2}.\]
    Por lo tanto, tenemos que
    \begin{align*}
        \sup f[I]& - \inf f[I]\geq \sum_{k = 1}^\infty\frac{2^m\pi r}{k^2}\int_{J}\frac{\cos((2^{m+j}-2^k)\pi x) + \cos((2^{m+j}+2^k)\pi x + 2
    \theta)}{2}\,\text{d}x.
    \end{align*}
    Por otro lado, observemos que para todo $k>m$ (con $k\neq m+j$, en el caso del signo negativo), y para cualquier $\varphi\in\mathbb{R}$
    \[\int_{J}\cos((2^{m+j}+ 2^k)\pi x + \varphi)\,\text{d}x = \int_{J}\cos((2^{m+j}- 2^k)\pi x + \varphi)\,\text{d}x = 0.\]
    Así, 
    \begin{align*}\tag{$*$}\label{eq:dif_supremos}
        \sup f[I] - \inf f[I]&\geq \\
        \frac{\pi r}{2(m+j)^2} +& \sum_{k = 1}^{m}\frac{2^m\pi r}{k^2}\int_{J}\frac{\cos((2^{m+j}-2^k)\pi x) + \cos((2^{m+j}+2^k)\pi x + 2\theta)}{2}\,\text{d}x.
    \end{align*}
    Observemos que, para $a,b\in \mathbb{R}$
    \[|\sin(a+b)-\sin(a)| = 2\left |\sin\left(\frac{b}{2}\right)\cos\left(a+\frac{b}{2}\right)\right |\leq |b|.\]
    Ahora, para $k\leq m$, tomemos $\varphi \in \mathbb{R}$, $w \in\{2^{k}, -2^{k}\}$, y supongamos que $J = [y, y+2^{-m}]$. Entonces, usando esta última observación,
    \begin{align*}
        \int_{J}\cos&((2^{m+j}+w)\pi x + \varphi)\,\text{d}x\\
        & = \frac{\sin((2^{m+j}+w)(y+2^{-m})\pi+\varphi)-\sin((2^{m+j}+w)y\pi+\varphi)}{(2^{m+j}+w)\pi}\\
        & = \frac{\sin((2^{m+j}+w)y\pi+\varphi + 2^{-m}\pi w)-\sin((2^{m+j}+w)y\pi+\varphi)}{(2^{m+j}+w)\pi} \\
        & \geq -\frac{|w|}{2^{m}(2^{m+j}+w)}.
    \end{align*}
    Con esto, de \eqref{eq:dif_supremos} se sigue que
    \begin{align*}
        \sup f[I] - \inf f[I]&\geq \frac{\pi r}{2(m+j)^2} - \sum_{k = 1}^{m}\frac{\pi r}{2k^2}\left(\frac{2^k}{2^{m+j}-2^k} + \frac{2^k}{2^{m+j}+2^k}\right)\\
        %%&= \frac{\pi r}{2(m+j)^2} - \frac{\pi r}{2^{m+j}}\sum_{k = 1}^{m} \frac{2^{k-1}}{k^2}\left(\frac{1}{1-2^{k-m-j}} + \frac{1}{1+2^{k-m-j}}\right)\\
        &= \frac{\pi r}{2(m+j)^2} - \frac{\pi r}{2^{m+j}}\sum_{k = 1}^{m}\frac{2^k}{k^2(1-2^{2k-2m-2j})}\\\tag{$**$}\label{eq:dif_supremos_2}
        &\geq \frac{\pi r}{2(m+j)^2} - \frac{\pi r}{2^m(2^j-1)}\sum_{k = 1}^{m}\frac{2^k}{k^2}.
    \end{align*}
    \textbf{Afirmación:}
    \[\sum_{k = 1}^{n}\frac{2^k}{k^2} \leq 5\frac{2^n}{n^2}.\]
    Haciendo las cuentas, podemos verificar que se cumple para $n=1,2,3,4$. Supongamos que se cumple para algún $n\geq 5$. Entonces
    \[\sum_{k = 1}^{n+1}\frac{2^k}{k^2} \leq 5\frac{2^n}{n^2} + \frac{2^{n+1}}{(n+1)^2} = \left(\frac{5}{2}\left(1+\frac{1}{n}\right)^2 + 1\right)\frac{2^{n+1}}{(n+1)^2} \leq 5\frac{2^{n+1}}{(n+1)^2},\]
    y por tanto se cumple para todo $n\in\mathbb{N}$. Usando esto en \eqref{eq:dif_supremos_2}, tenemos que 
    \[\sup f[I] - \inf f[I] \geq \frac{\pi r}{2(m+j)^2} - \frac{5\pi r}{m^2(2^j-1)}.\]
    Como $2^{-m}<\varepsilon \leq 1/2$, entonces $m\geq 2$. Tomando $j = 10$, tenemos que
    \[\sup f[I] - \inf f[I] \geq \frac{\pi r}{2(6m)^2} - \frac{\pi r}{200m^2} = \frac{2\pi r }{225m^2} \geq \frac{\pi r}{450(m-1)^2}\geq \frac{\log^2 (2)\pi r}{450\log^2 (\varepsilon)}.\]
    Llamando $c =\log^2 (2)\pi/450 $, esto demuestra el lema.
\end{proof}

\begin{theorem}
    Existen $g,h\in C[0,1]$ tales que, para toda función $f\in C[0,1]$,
    \[S_f = \{(a, \, b)\in \mathbb{R}^2\,:\,(f + a g + b h) \notin \mathcal{NL}[0,1]\}\]
    tiene medida de Lebesgue cero.
\end{theorem}
\begin{proof}
    Tomemos $g,h\in C[0,1]$ como en el lema anterior, y $f\in C[0,1]$. Observemos que
    \[S_f = \{(a, \, b)\in \mathbb{R}^2\,:\,(f + a g + b h) \text{ es Lipschitz en algún punto } x\in[0,1]\}.\]
    Para cada $M>0$, sea
    \[S_{f,\, M} = \{(a, \, b)\in \mathbb{R}^2\,:\,(f + a g + b h) \text{ es $M$-Lipschitz en algún punto } x\in[0,1]\}.\]
    Así,
    \[S_f = \bigcup_{n\in\mathbb{N}} S_{f,\, n}.\]
    Sea $\lambda$ la medida de Lebesgue en $\mathbb{R}^2$.\\\\
    \textbf{Afirmación:} $\lambda(S_{f,\,n}) = 0$ para todo $n\in\mathbb{N}$.\\\\
    Sea $n\in\mathbb{N}$. Para cada $m\in\mathbb{N}$, con $m\geq 2$, sea $\varepsilon_{m} = 1/m$. Tomemos $m\geq 2$, y cubramos a $[0,1]$ con intervalos cerrados de longitud $\varepsilon_{m}$. Llamemos $I_1, \dots, I_m$ a cada uno de estos intervalos, y consideremos 
    \[J_k = \{(a, \, b)\in \mathbb{R}^2\,:\,(f + a g + b h) \text{ es $n$-Lipschitz en algún punto } x\in I_k\}.\]
    Así, $S_{f,\,n} = \bigcup_{k = 1}^m J_k$.\\\\
    Ahora, tomemos $(a_1,\,b_1),\,(a_2,\,b_2)\in J_k$. Veamos qué tan grande puede llegar a ser la distancia entre estos dos puntos. Llamemos, para $i = 1,2$, 
    \[f_i = a_i g + b_i h.\]
    Supongamos que $f_1$ y $f_2$ son $n$-Lipschitz en $x_1$, y $x_2$, respectivamente. Así,
    \begin{align*}
        \sup_{x\in I_k}|f_1(x)- f_2(x) - (f_1(x_1)-f_2(x_2))|&\leq \sup_{x\in I_k}|f_1(x)-f_1(x_1)| + \sup_{x\in I_k} |f_2(x)-f_1(x_2)|\\ 
        &\leq 2n\varepsilon_m,
    \end{align*}
    de donde
    \[ \sup_{x\in I_k}(f_1(x)- f_2(x)) - \inf_{x\in I_k}(f_1(x)- f_2(x))\leq 4n\varepsilon_m.\]
    Ahora, por el lema anterior, exite $c >0$ tal que 
    \[\frac{c\sqrt{(\alpha_1 - \alpha_2)^2 + (\beta_1-\beta_2)^2}}{\log^2(\varepsilon_m)}\leq  \sup_{x\in I_k}(f_1(x)- f_2(x)) - \inf_{x\in I_k}(f_1(x)- f_2(x)).\]
    Por lo tanto, 
    \[||(\alpha_1,\,\beta_1) - (\alpha_2,\,\beta_2)||\leq \frac{4n\varepsilon_m\log^2(\varepsilon_m)}{c}.\]
    Es decir, cada $J_k$ está contenido en un rectángulo de lados de longitud $4n\varepsilon_m\log^2(\varepsilon_m)/c$. Entonces 
    \[\lambda (J_k) \leq \left(\frac{4n\varepsilon_m\log^2(\varepsilon_m)}{c}\right)^2 = \frac{16n^2\log^4(m)}{c^2m^2},\]
    por lo cual, para toda $m\geq 2$,
    \[\lambda(S_{f,\,n})\leq \frac{16n^2\log^4(m)}{c^2m} \xrightarrow[m\to\infty]{}0.\]
\end{proof}

\begin{theorem}
    Casi toda función continua en $[0,1]$ es nunca derivable.
\end{theorem}
\begin{proof}
    Por el teorema anterior $P = \langle g,\, h\rangle$ es una sonda de $\mathcal{NL}[0,1]$. Como $\mathcal{NL}[0,1]\subseteq \mathcal{ND}[0,1]$, entonces $\mathcal{ND}[0,1]$ es prevalente. Es decir, casi toda función continua en $[0,1]$ es nunca derivable.
\end{proof}

\renewcommand{\thesection}{A.\arabic{section}}
\chapter*{Apéndice} 
\addcontentsline{toc}{chapter}{Apéndice}

\setcounter{chapter}{10}
\setcounter{section}{0}
\section{Demostraciones del primer capítulo}
\begin{proposition}\label[proposition]{appendix:seno_dory}
    Si $c,d\in\mathbb{R}$, entonces
    \[\sin\left(\frac{c+d}{2}\right)\sin\left(\frac{c-d}{2}\right) = \frac{1}{2}(\cos(d) - \cos(c)).\]
\end{proposition}
\begin{proof}
    Recordando que el seno de una suma sigue la regla
    \[\sin(c+d) =\cos(c)\sin(d) + \sin(c)\cos(d),\]
    tenemos que
    \begin{align*}
        \sin(c+d)\sin(c-d) &= (\cos(c)\sin(d) + \sin(c)\cos(d))\cdot(\cos(c)\sin(d) - \sin(c)\cos(d))\\
        &=\sin^2(c)\cos^2(d) - \cos^2(c)\sin^2(d)\\
        &= \sin^2(c)(\cos^2(d) + \sin^2(d)) - \sin^2(d)(\cos^2(c) + \sin^2(c))\\
        &= \sin^2(c)-\sin^2(d) \\
        &=  \cos^2(d)-\cos^2(c).
    \end{align*}
    Ahora, como $2\cos^2(x) = 1 + \cos(2x)$ para todo $x\in\mathbb{R}$, entonces
    \[\sin(c+d)\sin(c-d) = \frac{1}{2}(\cos(2c) - \cos(2d)).\]
    Lo que demuestra la igualdad buscada.
\end{proof}

\begin{proposition}\label[proposition]{appendix:prop:importante}
    Sean $n \in \mathbb{N}$, $a,b,m,r\in\mathbb{R}$ con $a<b$, $I = [a,b]$  y $f:I\to \mathbb{R}$ dada por $f(x) = mx + r$, donde $|m|> n$. Para todo $\varepsilon>0$ existe $0<\delta<\varepsilon$ tal que, para cada $x\in I$, existe $y\in I$ con $0<|x-y|<\varepsilon$ y que tiene la siguiente propiedad: 
\[\text{ si } p\in \overbar{N_{\delta}(f)}\,[x],\; q\in \overbar{N_{\delta}(f)}\,[y]\; \text{ entonces }\; \left|\frac{p-q}{x-y}\right| > n.\]
\end{proposition}
\begin{proof}
Sean $\varepsilon > 0$ y
\[\delta = \frac{1}{2}\min\left\{\varepsilon, \varepsilon^{2},\, \left(\frac{|m|-n}{2}\right)^2,\, \left(\frac{b-a}{2}\right)^2\right\}.\]
Llamemos $c = \sqrt{\delta}$. Dado $x\in I$, como \[\max\{a-x, b-x\}\geq \frac{b-a}{2} \geq c,\] entonces $x-c\in I$ ó $x+c\in I$. Sin pérdida de generalidad, supongamos que $x+c\in I$ y definamos $y = x+c$. Así $y\in I$ y $0<y-x<\varepsilon$.\\\\
Por el \cref{lemma:importante}, para todo $z\in I$ 
\[\; \overbar{N_{\delta}(f)}\,[z] = [mz+r-\delta,\, mz+r+\delta].\]
Notemos que, como \[\delta \leq \left(\frac{|m|-n}{2}\right)^2 \leq \left(\frac{m}{2}\right)^2,\] se cumple que $mx+r+\delta \leq my + r-\delta$. Por tanto, para cualesquiera
\[ p\in \overbar{N_{\delta}(f)}\,[x]\; \text{ y }\; q\in \overbar{N_{\delta}(f)}\,[y]\]
se tiene que $|p-q| \geq |(mx+r+\delta) - (my+r-\delta)| = |2\delta - mc| = c|2c-m|$, de donde
\[\left|\frac{p-q}{x-y}\right| = \left|\frac{p-q}{c}\right| \geq |2c - m| \geq |m| - 2c > n .\]
\end{proof}

\begin{proposition}\label[proposition]{appendix:prop:import2}
    Si $a,b\in\mathbb{R}$, con $a<b$, y $f:[a,b]\to\mathbb{R}$ es una recta, entonces para cualesquiera $\varepsilon > 0$ y $n\in \mathbb{N}$ existe una poligonal $P\subseteq \mathbb{R}^{2}$ tal que: 
    \begin{enumerate}
        \item[(1)]{$P\subseteq N_\varepsilon(f)$,}
        \item[(2)]{empieza en $(a,\,f(a))$ y termina en $(b,f(b))$, y}
        \item[(3)]{es unión finita de rectas con pendiente cuyo valor absoluto es mayor a $n$.}
    \end{enumerate}
\end{proposition}
\begin{proof}
    Para cada $k\in \mathbb{N}$ tomemos $k$ números equidistantes en $[a,b]$, digamos $\{a_{1},...,a_{k}\}$, tales que 
    \[a = a_{1} < a_{2} < \dots < a_{k-1} < a_{k} = b.\]
    Entonces, para cada $i\in \{1,\dots,k\}$, $a_{i+1}-a_{i} = \frac{1}{k-1}(b-a)$. Consideremos $\delta = \frac{1}{2}\varepsilon$ y $g_{k}:\{a_{1},\cdots,a_{k}\} \to N_{\varepsilon}(f)$ la función dada por
    \[g_k(a_i) = \begin{cases}
            (a_i,\,f(a_i)), & \text{si } i=1 \text{ o } i = m,\\
            (a_i,\,f(a_i) + (-1)^{i}\delta), & \text{en otro caso.}
            \end{cases}\]
    Sea $R_{i}$ la recta que une a $g(a_{i})$ con $g(a_{i+1})$. Entonces $P = \bigcup_{i=1}^{k-1} R_{i}$ es una poligonal contenida en $N_{\varepsilon}(f)$.\\\\
    Por otro lado, si $m$ es la pendiente de $f$, y $m_{i}$ es la pendiente de la recta $R_{i}$, entonces para todo $i\in \{1,...,k-1\}$
    \[m_i = \frac{g(a_{i+1})- g(a_i)}{a_{i+1}-a_i} = m - (-1)^{i}(k-1)\frac{\varepsilon}{b-a},\]
    de donde
    \[|m_{i}| \geq (k-1)\frac{\varepsilon}{(b-a)} - |m|.\]
    Así, para $k$ suficientemente grande, $P = \bigcup_{i=1}^{k-1} R_{i}$ es una poligonal contenida en $N_{\varepsilon}(f)$, donde las pendientes de sus rectas tienen valor absoluto mayor a $n$. 
\end{proof}

\begin{theorem}\label[theorem]{appendix:implica_densidad_lynch}
    Si $f:[0,1]\to\mathbb{R}$ es una recta, entonces para todo $\varepsilon > 0$ existe una función $\varphi\in\mathcal{ND}[0,1]$ tal que 
    \[\sup_{x\in[0,1]}|f(x)-\varphi(x)| < \varepsilon.\]
\end{theorem}
\begin{proof}
    Sean $\varepsilon >0$ y $f:[0,1]\to\mathbb{R}$ una recta. Por la \cref{appendix:prop:import2}, existe una poligonal $P\subseteq N_{\varepsilon / 2}$ que empieza en $(0,f(0))$, termina en $(1,f(1))$ y es unión finita de rectas con pendiente mayor a $1$. Por el \cref{lem:a_usar_en_appendice}, para cada una de estas rectas, digamos $g_i:[a_i,b_i]\to\mathbb{R}$, $0<i\leq n$, existe $\varphi_{i}\in\mathcal{ND}[a_i,b_i]$ tal que 
    \[||g_i-\varphi_i||_{\infty}<\frac{\varepsilon}{4n}.\]
    Sea $\varphi:[0,1]\to\mathbb{R}$ dada por:
    \[\varphi(x) = \begin{cases}
        \varphi_{1}(x), \quad &\text{si } \; x\in[a_1, a_{2}], \\
        \varphi_i(x) + \sum\limits_{k=1}^{i-1}(\varphi_k(a_{k+1}) - \varphi_{k+1}(a_{k+1})), \quad &\text{si } \; \exists \,1<i< k \text{ tal que } x\in(a_i, a_{i+1}].
    \end{cases}\]
    Observemos que $\varphi$ no es derivable en ningún punto, pues ninguna $\varphi_i$ lo es. Por otro lado, notemos que $\varphi$ es continua en cada intervalo abierto de la forma $(a_i, a_{i+1})$, y también lo es en $[a_1,a_2]$. De este modo, para ver que es continua en todo el intervalo $[0,1]$ solo falta demostrar que lo es en cada $a_i$, para $i>2$. Veamos, tomando límite por la izquierda tenemos que:
    \begin{align*}
        \lim_{x\to a_{i}^{-}}\varphi(x) =\varphi_{i-1}(a_i) +  \sum\limits_{k=1}^{i-2}(\varphi_k(a_{k+1}) - \varphi_{k+1}(a_{k+1})).
    \end{align*}
    Ahora, tomando límite por la derecha:
    \begin{align*}
        \lim_{x\to a_{i}^{+}}\varphi(x) &= \lim_{x\to a_{i}^{+}} \varphi_{i}(x)+\sum\limits_{k=1}^{i-1}(\varphi_k(a_{k+1}) - \varphi_{k+1}(a_{k+1})) \\
        &=\varphi_{i}(a_i) +  \sum\limits_{k=1}^{i-1}(\varphi_k(a_{k+1}) - \varphi_{k+1}(a_{k+1})) \\
        &=\varphi_{i-1}(a_i) +  \sum\limits_{k=1}^{i-2}(\varphi_k(a_{k+1}) - \varphi_{k+1}(a_{k+1})) = \lim_{x\to a_{i}^{-}}\varphi(x),
    \end{align*}
    pues $\varphi_{i-1}$ es continua. Por lo tanto, $\varphi\in\mathcal{ND}[0,1]$.
    \noindent Ahora veamos que $||f-\varphi||_{\infty} \leq \varepsilon$. Sea $x\in[0,1]$ y supongamos que $x\in(a_i,a_{i+1}]$. Entonces,
    \begin{align*}
        |\varphi(x)-f(x)| &= \left|\varphi_i(x) + \sum\limits_{k=1}^{i-1}(\varphi_k(a_{k+1}) - \varphi_{k+1}(a_{k+1})) - f(x)\right|\\
        &\leq |\varphi_i(x)-g_i(x)| +|g_i(x)-f(x)|  \\
        & \qquad\qquad  + \sum\limits_{k=1}^{i-1}\left|\varphi_k(a_{k+1})-g_k(a_{k+1}) + g_{k}(a_{k+1}) - \varphi_{k+1}(a_{k+1})\right|\\
        &<\frac{\varepsilon}{4n} + \frac{\varepsilon}{2} + \sum\limits_{k=1}^{i-1}\left|\varphi_k(a_{k+1})-g_k(a_{k+1})|+ |g_{k+1}(a_{k+1}) - \varphi_{k+1}(a_{k+1})\right|\\
        &<\frac{\varepsilon}{2} + \sum\limits_{k=1}^{i}\frac{\varepsilon}{2n}\\
        &\leq\varepsilon.
    \end{align*}
    Es decir, $||f-\varphi||_{\infty} < \varepsilon$.
\end{proof}


\newpage
\section{Demostraciones del segundo capítulo}

\begin{proposition}\label[proposition]{appendix:obs_cap_2}
    Sea $V$ un $\mathbb{R}$-espacio de Banach dimensionalmente infinito y separable. Si $\mu$ es una medida de borel en $V$ invariante bajo traslaciones, entonces $\mu$ es la constante 0, o todos los abiertos no vacíos en $V$ tienen medida infinita.
\end{proposition}
\begin{proof}
    Supongamos que $\mu$ no es constante, y tomemos $x_0\in V$. Primero, como $V$ es separable y $\mu$ es invariante bajo traslaciones, entonces 
    \[\text{si existe } s>0 \text{ tal que } \mu(B(x_0,\,s)) = 0, \text{ entonces } \mu \equiv 0 .\]
     Como $\mu\not\equiv 0$, entonces $\mu(B(x_0,s)) > 0$, para todo $s>0$. Nuestro objetivo es demostrar lo siguiente: para cada $r>0$ existe una sucesión $\{x_{n}\}_{n\in\mathbb{N}}\subseteq V$ tal que todas las bolas de la forma $B(x_{n}, r/4)$ son disjuntas entre sí y están contenidas en $B(x_0,r)$.\\\\
    Procederemos recursivamente: supongamos que ya encontramos $x_{0},x_{1}, \, \dots ,\,x_{n}$. Observemos que el subespacio  \[F = \langle\{ x_{0}, \, \dots ,\,x_{n}\} \rangle\]
    es un subconjunto propio de $V$ cerrado, lo cual asegura la existencia de un $y\in V\setminus F$ tal que $\text{d}(y,\,F) > 0$.\\\\
    Sea $f:\mathbb{R}\to\mathbb{R}$ dada por $f(\lambda) = \text{d}(\lambda y,\, F)$. Observemos que para todo $\lambda \neq 0$ 
    \[f(\lambda)  = \inf\{||\lambda y - x||\,:\, x\in F\}  = |\lambda|\inf\{ ||y - x/\lambda||\,:\, x\in F\} = |\lambda| f(1).\]
    Si tomamos
    \[\lambda_0 = \frac{5}{8}\frac{r}{\text{d}(y,F)},\]
    entonces, por la observación anterior,
    \[\frac{r}{2}<f(\lambda_0) <\frac{3r}{4}.\]
    Así, por la definición de $f$, existe $z\in F$ tal que  
    \[\frac{r}{2}< ||\lambda_0y-z|| <\frac{3r}{4}.\] Sea $x_{n+1} = x_0+z-\lambda_0y$. De esta manera, \[B(x_{n+1},\,r/4)\subseteq B(x_0,\,r) \;\text{ y } \;||x_{n+1}-x_j||>r/2,\] 
    para todo $j\leq n$. Por recursión, la sucesión buscada existe, y por construcción se sigue que: 
    \[\sum_{n=0}^{\infty} \mu(B(x_n, r/4)) = \mu\left(\bigcup_{n=0}^{\infty} B(x_n, r/4) \right)\leq \mu(B(x_0, r)).\]
    Por tanto, como $\mu(B(x_n,r/4))$ es positivo y constante para todo $n\in \mathbb{N}$, se tiene que $\mu(B(x_0,\,r)) = \infty$. 
\end{proof}

\printbibliography[title={Bibliografía\let\thefootnote\relax\footnote{* El nombre del autor se presenta con un asterisco para indicar que no revisamos directamente esta fuente. La información se obtuvo a través de su citación en otro trabajo.}}]


\end{document}
