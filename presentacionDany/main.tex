%----------------------------------------------------------------------------------------
%    PACKAGES AND THEMES
%----------------------------------------------------------------------------------------

\documentclass[aspectratio=169,xcolor=dvipsnames]{beamer}
\usetheme{SimpleDarkBlue}

\usepackage{hyperref}
\usepackage{graphicx} % Allows including images
\usepackage{booktabs} % Allows the use of \toprule, \midrule and \bottomrule in tables
\usepackage{multicol}
\usepackage[most]{tcolorbox}

\renewenvironment{examples}[1][Ejemplos]{%
    \begin{exampleblock}{#1}%
}{%
    \end{exampleblock}%
}

\def\CL{\mathop{\rm CL}\nolimits}
\def\Sel{\mathop{\rm Sel}\nolimits}
\def\K{\mathop{\rm K\hspace{0mm}}\nolimits}
%----------------------------------------------------------------------------------------
%    TITLE PAGE
%----------------------------------------------------------------------------------------

\title{Selecciones en topología y teoría de la medida}
%\subtitle{Daniela Medrano Gutiérrez}

\author{Presenta: \\ Daniela Medrano Gutiérrez \\ \hspace{.5cm} \\Director de tesis: \\ M. en C. Alejandro Ríos Herrejón \hspace{.5cm}}

\institute
{
    Licenciatura en Matemáticas Aplicadas \\
    Universidad Panamericana % Your institution for the title page
}

\date{4 de marzo de 2025} % Date, can be changed to a custom date

%----------------------------------------------------------------------------------------
%    PRESENTATION SLIDES
%----------------------------------------------------------------------------------------

\begin{document}

\begin{frame}
    \titlepage
\end{frame}

\begin{frame}{Contexto histórico}
    \centering
    La teoría de hiperespacios es una línea de investigación en topología que surgió a principios del siglo \textsc{XX} con los trabajos de Felix Hausdorff y de Leopold Vietoris. En 1951, Ernest Michael marcó un punto de inflexión al introducir el concepto de selección y desde hace más de 35 años, México se ha unido a los esfuerzos en esta área, colaborando con Japón y Bulgaria para contribuir al desarrollo de la teoría. En virtud de lo anterior, el presente trabajo es un texto introductorio que abarca esencialmente tres temas: aspectos generales de la topología de Vietoris en los hiperespacios, selecciones continuas definidas en ellos y medidas exteriores generadas a partir de selecciones débiles. 
\end{frame}

\begin{frame}{Índice}
    % Throughout your presentation, if you choose to use \section{} and \subsection{} commands, these will automatically be printed on this slide as an overview of your presentation
    \begin{multicols}{2}
        \tableofcontents
    \end{multicols}
\end{frame}

%------------------------------------------------
\section{Preliminares}
\subsection{Teoría de conjuntos}
\subsection{Topología general}
\subsection{Espacios topológicos ordenados}
%------------------------------------------------

%------------------------------------------------
\section{Hiperespacios}
\subsection{La topología de Vietoris}

\begin{frame}
    \begin{tcolorbox}[colframe=Orchid, colback=Orchid, coltitle=white, sharp corners, boxrule=0.8mm, width=\textwidth]
        \centering
        \Huge \textbf{Hiperespacios}
    \end{tcolorbox}
\end{frame}

\begin{frame}{La topología de Vietoris}
    \begin{examples}[Definición]
        Un \textit{hiperespacio} de un espacio topológico $X$ es un subconjunto de $P(X) \setminus \{\emptyset\}$ que está dotado de una topología. 
    \end{examples}

    \begin{examples}[Definición]
        A lo largo de esta presentación, utilizaremos el símbolo $\CL(X)$ para representar la familia de todos los subconjuntos cerrados y no vacíos de $X$.
    \end{examples}

\end{frame}

\begin{frame}{La topología de Vietoris}
    Si $n\in \omega$ y $\{S_i : i\leq n\} \subseteq P(X)$, el conjunto $\langle S_0,\ldots, S_n\rangle$ queda determinado como $$\langle S_{0}, \dotsc, S_{n}\rangle := \left\{A \in \CL (X) : A \subseteq \bigcup_{i \leq n} S_{i} \ \wedge \ \forall i \leq n (A \cap S_{i} \neq \emptyset) \right\}.$$ A la colección $\langle S_0, \ldots, S_n\rangle$ le llamaremos el {\it vietórico generado por $\{S_i : i\leq n\}$}.

    \begin{block}{Proposición}
        La colección $$\mathcal{B}{V} := \left\{\langle U{0}, \dotsc, U_{n}\rangle : n \in \omega \ \wedge \ \left\{U_{i}: i \leq n\right\} \subseteq \tau_{X}\right\}$$ es base para una topología en $\CL (X)$ conocida como la \textit{topología de Vietoris}.
    \end{block}

\end{frame}

%------------------------------------------------
\subsection{Resultados fundamentales}

\begin{frame}{Axiomas de separación}
    \begin{block}{Proposición}
    Si $X$ es un espacio topológico, entonces las siguientes afirmaciones son ciertas:
    \begin{enumerate}
        \item $\CL (X)$ siempre es un espacio $T_{0}$.
        \item Si $X$ es un espacio $T_1$, entonces $\CL (X)$ es un espacio $T_{1}$. 
        \item Si $X$ es un espacio $T_1$, entonces $X$ es un espacio regular si y sólo si $\CL (X)$ es un espacio de Hausdorff. 
    \end{enumerate}
    \end{block}
\end{frame}

\begin{frame}{Axiomas de separación}    
    \begin{block}{Proposición}
        Si $X$ es un espacio $T_{1}$, entonces $X$ es completamente regular si y sólo si $\CL (X)$ es completamente Hausdorff.
    \end{block}

    \begin{block}{Proposición}
        Si $X$ es un espacio $T_{1}$, los siguientes enunciados son equivalentes.

\begin{enumerate}
\item $X$ es normal.
\item $\CL (X)$ es completamente regular.
\item $\CL (X)$ es regular.
\end{enumerate}
    \end{block}
\end{frame}

\begin{frame}{Compacidad}
    \begin{block}{Teorema}
        Si $X$ es un espacio topológico, entonces $X$ es compacto si y sólo si $\CL (X)$ es compacto.
    \end{block}
    \begin{block}{Corolario}
        Si $X$ es un espacio $T_{1}$, entonces $X$ es un espacio compacto de Hausdorff si y sólo si $\CL(X)$ es un espacio compacto de Hausdorff.
    \end{block}
\end{frame}

\begin{frame}{Compacidad local}
    \begin{examples}[Definición]
        Un espacio topológico $X$ es \textit{localmente compacto} si todo punto en $X$ pertenece a un abierto con cerradura compacta.
    \end{examples}
    \begin{block}{Proposición}
        Si $X$ es un espacio $T_{1}$ y $\CL (X)$ es localmente compacto, entonces $X$ es localmente compacto. 
    \end{block}
    \begin{block}{Corolario}
        Si $X$ es $T_3$ y $A\in \CL(X)$, entonces $A$ es un punto de compacidad local en $\CL(X)$ si y sólo si existe $U\in \tau_X$ con $A\subseteq U$ y $\overline{U}$ compacta.
    \end{block}
\end{frame}

\begin{frame}{Compacidad local}
    En [E. Michael, 1951] se establece que si $X$ es un espacio topológico y $A \in \CL (X)$, entonces $A$ tiene una vecindad compacta en $\CL (X)$ si y sólo si $A$ es un subconjunto compacto de $X$. Esta equivalencia es falsa.

    \begin{block}{Proposición}
        Si $A$ es un subconjunto compacto de $\mathbb S$, entonces $A$ no es un punto de compacidad local en $\CL (\mathbb{S})$.
    \end{block}
\end{frame}

\begin{frame}{Compacidad local}
    \begin{block}{Corolario ([C. Constantini, S. Levi, J. Pelant, 2002])}
        Los siguientes enunciados son equivalentes para un espacio $X$ que es $T_3$.

\begin{enumerate}
\item $X$ es localmente compacto.
\item Para cualquier $x\in X$ se satisface que $\{x\}$ es un punto de compacidad local en $\CL(X)$.
\end{enumerate}
    \end{block}
    
\begin{block}{Corolario ([C. Constantini, S. Levi, J. Pelant, 2002])}
Los siguientes enunciados son equivalentes para un espacio $X$ que es $T_3$.

\begin{enumerate}
\item $X$ es compacto.
\item $\CL(X)$ es localmente compacto.
\item $X$ es un punto de compacidad local en $\CL(X)$.
\item $\CL(X)$ es compacto.
\end{enumerate}    
\end{block}
\end{frame}

\begin{frame}{Metrizabilidad}    
    \begin{block}{Teorema}
        Si $X$ es un espacio $T_1$, entonces los siguientes enunciados son equivalentes.

\begin{enumerate}
\item $X$ es compacto y metrizable.
\item $X$ es segundo numerable, compacto y $T_{2}$.
\item $\CL (X)$ es segundo numerable, compacto y $T_{2}$.
\item $\CL (X)$ es segundo numerable y $T_{2}$.
\item $\CL (X)$ es compacto y metrizable.
\item $\CL (X)$ es metrizable.
\end{enumerate}
    \end{block}
\end{frame}

\begin{frame}{Espacio de subconjuntos finitos}    
    \begin{examples}[Definición]
        Si $X$ es un espacio topológico y $n\in \mathbb{N}$, el \textit{$n-$ésimo producto simétrico} de $X$ es el hiperespacio
\[\mathcal{F}_{n}(X) := \{A \in \CL (X) : |A| \leq n\}.\]
Además, el \textit{espacio de subconjuntos finitos} de $X$ es el hiperespacio
\[\mathcal{F}(X) := \bigcup_{n \in \mathbb N} \mathcal{F}_{n}(X)\]
    \end{examples}

    \begin{block}{Proposición}
        Si $X$ es un espacio $T_1$, entonces $X$ es homeomorfo a $\mathcal{F}_1(X)$.
    \end{block}
\end{frame}

\begin{frame}{Conexidad}    
    \begin{block}{Teorema}
        Si $X$ es un espacio $T_{1}$ y $\mathcal{F}(X) \subseteq \mathcal{H} \subseteq \CL (X)$, entonces los siguientes enunciados son equivalentes. 

\begin{enumerate}
\item $X$ es conexo.
\item $\CL (X)$ es conexo.
\item $\mathcal{H}$ es conexo.
\item $\mathcal{F} (X)$ es conexo.
\item Existe $n \in \mathbb N$ tal que $\mathcal{F}_{n}(X)$ es conexo.
\item Para toda $n \in \mathbb N$ se cumple que $\mathcal{F}_{n}(X)$ conexo.
\end{enumerate}
    \end{block}

    \begin{block}{Teorema}
        Si $X$ es conexo y $T_2$, entonces $[X]^{n}$ es conexo para cualquier $n\in \mathbb{N}$.
    \end{block}
\end{frame}

%------------------------------------------------
\subsection{Selecciones}

\begin{frame}{Selecciones}
    \begin{examples}[Definición]
        Sean $\mathcal{H}$ un subespacio de $\CL (X)$ y $f: \mathcal{H} \to X$ una función. Diremos que $f$ es una \textit{selección} si para cualquier $A \in \mathcal{H}$ se satisface que $f(A) \in A$.
    \end{examples}

    \begin{examples}[Definición]
        Para cualquier subespacio $\mathcal{H}$ de $\CL(X)$ se definen
        \begin{align*}
\Sel(\mathcal{H}) &:= \left\{f : \mathcal{H} \to X \ \mid \ f \ \text{es una selecci\'on}\right\} \quad \text{y} \\ \Sel^{c}(\mathcal{H}) &:= \left\{f \in \Sel(\mathcal{H}) : f \ \text{es continua}\right\}.
\end{align*}
    \end{examples}

    \begin{examples}[Definición]
        Sean $X$ un espacio topológico, $\mathcal{F}{2}(X) \subseteq \mathcal{H} \subseteq \CL (X)$ y $f\in \Sel(\mathcal{H})$. Para cualesquiera $x, y \in X$ se define la relación $x <{f} y$ mediante la regla $x \neq y$ y $f\left(\{x,y\}\right) = x$. 
    \end{examples}
    
\end{frame}

\begin{frame}{El orden $<_f$}
    \begin{block}{Lema}
        Sea $X$ un espacio $T_1$. Si $\mathcal{F}2(X) \subseteq \mathcal{H} \subseteq \CL (X)$ y $f\in \Sel^{c}(\mathcal{H})$, entonces para cualquier $x \in X$ se satisface que $(\gets, x){f}$ es abierto en $X$.
    \end{block}

    A partir de este momento y hasta el final del capítulo, todos los espacios serán de Hausdorff. 

    \begin{block}{Lema}
        Si $\mathcal{F}2(X) \subseteq \mathcal{H} \subseteq \CL (X)$ y $f\in \Sel^{c}(\mathcal{H})$, entonces para cualquier $x \in X$ se satisface que $(x, \to){f}$ es abierto en $X$.
    \end{block}

    \begin{block}{Lema}
        Sean $X$ un espacio conexo y $\mathcal{F}_2(X) \subseteq \mathcal{H} \subseteq \CL (X)$. Si $f\in \Sel^{c}(\mathcal{H})$, entonces $<_f$ es un orden débil para $X$.
    \end{block}
\end{frame}

\begin{frame}{El error en la demostración de Ernest Michael}
    \begin{block}{Teorema}
        Si $X$ es un espacio conexo, $n\geq 2$ y $f\in \Sel_{n}^{c}(X)$, entonces para cualquier $A\in \mathcal{F}{n}(X)$ se satisface que $f(A) = \min{<_f} A$.
    \end{block}
    \begin{block}{Corolario}
        Si $X$ es un espacio conexo, $\mathcal{F}(X) \subseteq \mathcal{H} \subseteq \CL (X)$ y $f\in \Sel^{c}(\mathcal{H})$, entonces para cualquier $A\in \mathcal{F}(X)$ se cumple que $f(A) = \min_{<_f} A$.
    \end{block}
\end{frame}

\begin{frame}{Una selección siempre elige al mínimo}
    \begin{block}{Proposición}
        Sean $X$ un espacio topológico y $\mathcal{F}(X) \subseteq \mathcal{H} \subseteq \CL(X)$. Los siguientes enunciados son equivalentes para $f\in \Sel^{c}(\mathcal{H})$.

\begin{enumerate}
\item $f(A) = \min_{<_f} A$ para cualquier $A\in \mathcal{H}$.
\item $f(A) = \min_{<_f} A$ siempre que $A\in \mathcal{F}(X)$.

\end{enumerate}
    \end{block}

    \begin{block}{Corolario}
        Si $X$ es un espacio conexo, $\mathcal{F}(X) \subseteq  \mathcal{H} \subseteq \CL (X)$ y $f\in \Sel^{c}(\mathcal{H})$, entonces para todo $A \in \mathcal{H}$ se satisface que $f(A) = \min_{<_f} A$. 
    \end{block}
\end{frame}

\begin{frame}{Condiciones para tener selecciones continuas}
    \begin{block}{Teorema}
        Sea $X$ un espacio conexo o localmente conexo.
\begin{enumerate}
\item Los siguientes enunciados son equivalentes.
\begin{enumerate}
\item Existe una selección continua $f: \CL (X) \to X$.
\item Existe un orden débil $<$ para $X$ y cualquier elemento de $\CL (X)$ tiene elemento $<$-mínimo.
\end{enumerate}
\item Los siguientes enunciados son equivalentes. 
\begin{enumerate}
\item Existe una selección continua $f: \K (X) \to X$.
\item Existe un orden débil para $X$.
\end{enumerate}
\end{enumerate}
    \end{block}
\end{frame}

%------------------------------------------------
\section{Topologías generadas por selecciones}
%------------------------------------------------

\begin{frame}
    \begin{tcolorbox}[colframe=Orchid, colback=Orchid, coltitle=white, sharp corners, boxrule=0.8mm, width=\textwidth]
        \centering
        \Huge \textbf{Topologías generadas por selecciones}
    \end{tcolorbox}
\end{frame}

\subsection{Caracterizaciones para selecciones continuas en $\mathcal{F}_2(X)$}

\begin{frame}{Caracterizaciones para selecciones continuas en $\mathcal{F}_2(X)$}
    \begin{block}{Teorema}
        Si $X$ es un espacio de Hausdorff y $f \in \Sel_2(X)$, entonces las siguientes condiciones son equivalentes. 

\begin{enumerate}
\item $f \in \Sel_2^c(X)$.
\item Para cualesquiera $x, y \in X$ con $x <_f y$, existen $U\in \tau_X$ y $V\in \tau_X$ con $x\in U$, $y\in V$ y $U <_f V$. 
\end{enumerate}

    \end{block}

    \begin{block}{Teorema}
        Si $(X, \tau_X)$ es un espacio $T_1$ y $f \in \Sel_2^c(X, \tau_X)$, entonces las siguientes condiciones son equivalentes.

\begin{enumerate}
\item $X$ es de Hausdorff.
\item $f^{*} \in \Sel_2^{c}(X, \tau_X)$.
\item $\tau_f \subseteq \tau_X$.
\end{enumerate}

    \end{block}
\end{frame}

%------------------------------------------------
\subsection{Monotonía, monotonía débil y regularidad transitiva}

\begin{frame}{Monotonía, monotonía débil y regularidad transitiva}
Sean $X$ un espacio topológico, $\mathcal{F}_2(X) \subseteq \mathcal{H} \subseteq \CL(X)$ y $f\in \Sel(\mathcal{H})$.
    \begin{examples}[Definición]
        Decimos que $f$ es {\it monótona} si para cualesquiera $A\in \mathcal{H}$ y $B\in \mathcal{H}$ la condición $f(A) \in B \subseteq A$ implica que $f(A) = f(B)$. 
    \end{examples}

    \begin{examples}[Definición]
        Además, $f$ es {\it débilmente monótona} si $f(A\cup B) = f(A)$ siempre que $A\in \mathcal{H}$ y $B\in \mathcal{H}$ cumplen $f(A) = f(B)$. 
    \end{examples}

    \begin{examples}[Definición]
        Finalmente, $f$ es {\it transitivamente regular} si la relación $<_f$ es transitiva en $X$.
    \end{examples}

\end{frame}

\begin{frame}{Monotonía, monotonía débil y regularidad transitiva}

    \begin{block}{Proposición}
        Considere los siguientes enunciados para $\mathcal{F}(X) \subseteq \mathcal{H} \subseteq \CL(X)$ y $f\in \Sel(\mathcal{H})$. 

\begin{enumerate}\label{equivalencias_monotonia}
\item $f$ es monótona. 
\item $f$ es débilmente monótona y transitivamente regular.
\item $f(A) = \min_{<_f} A$ para toda $A \in \mathcal{H}$.
\end{enumerate}
Tenemos que (1) es equivalente a (3), (1) implica (2), y, cuando $f$ es continua, (2) implica (3).
    \end{block}

\end{frame}

\begin{frame}{No toda selección monótona es continua}

    \begin{block}{Proposición}
        Sea $X$ un espacio topológico.

\begin{enumerate}
\item Si $X$ es $T_1$ y no es discreto, entonces $\CL(X)$ tiene una selección monótona discontinua.
\item Si $f: \CL(X) \to X$ es una selección monótona, $A\in \CL(X) \cap \tau_X$ y $f(A)$ es un punto aislado, entonces $f$ es continua en $A$.
\item Si $X$ es un espacio discreto, entonces cualquier selección monótona en $\CL(X)$ es continua. Además, existen selecciones en $\CL(X)$ que son continuas y no monótonas. 

\end{enumerate}
    \end{block}

    \begin{block}{Proposición}
        Si $(X, \tau_X)$ es un espacio topológico y $f$ es una selección monótona para $\CL (X, \tau_X)$ tal que $\tau_f \subseteq \tau_X$, entonces $f \in \Sel^c(X, \tau_X)$. 
    \end{block}

\end{frame}

%------------------------------------------------
\subsection{La modificación de Sorgenfrey}

\begin{frame}{La modificación de Sorgenfrey}
        \begin{block}{Teorema}
        Los siguientes enunciados son equivalentes para un espacio de Hausdorff $(X,\tau_X)$.

\begin{enumerate}
\item $\CL (X, \tau_X)$ tiene una selección monótona y continua.
\item $(X, \tau_X)$ es Sorgenfrey bien ordenable.
\end{enumerate}
    \end{block}

    Cabe mencionar que sigue siendo un problema abierto determinar si la condición (1), cuando se omite la monotonía, es suficiente para implicar la propiedad (2).
\end{frame}

%------------------------------------------------
\subsection{Propiedades de separación}

\begin{frame}{Separación}
    \begin{block}{Proposición}
        Si $X$ es un conjunto y $f \in \Sel_2(X)$, entonces $\tau_f$ es una topología de Hausdorff para $X$.
    \end{block}

    \begin{block}{Proposición ([V. Gutev, T. Nogura, 2005])}
        Si $X$ es un conjunto y $f \in \Sel_2(X)$, entonces $\tau_f$ es una topología regular para $X$.
    \end{block}

    \begin{alertblock}{Ejemplo ([S. García Ferreira, A. H. Tomita, 2008])}
        Existe una selección $f \in \Sel_2(\mathbb P)$ tal que el espacio $(\mathbb P, \tau_f)$ no es normal.
    \end{alertblock}

    \begin{block}{Teorema ([M. Hrušák, I. Martínez Ruiz, 2010])}
        Si $X$ es un conjunto y $f \in \Sel_2(X)$, entonces $\tau_f$ es una topología completamente regular para $X$.
    \end{block}

\end{frame}

%------------------------------------------------
\section{Medidas exteriores generadas por selecciones}

\begin{frame}
    \begin{tcolorbox}[colframe=Orchid, colback=Orchid, coltitle=white, sharp corners, boxrule=0.8mm, width=\textwidth]
        \centering
        \Huge \textbf{Medidas exteriores generadas por selecciones}
    \end{tcolorbox}
\end{frame}

\begin{frame}{Medidas exteriores generadas por selecciones}
    \begin{examples}[Definición]
        Sea $f\in \Sel_2(\mathbb{R})$. Un conjunto $A\subseteq \mathbb{R}$ es {\it cubierto por $f$} si existe una colección $\{(a_n,b_n]f : n\in \omega\}$ de tal manera que $A \subseteq \bigcup{n\in \omega} (a_n,b_n]f$. La {\it medida exterior inducida por $f$} es la función $\lambda{f}^{} : P(\mathbb{R})\to \mathbb{R}$ definida como $$\lambda_{f}^{}(A) := \begin{cases} \inf\left\{\sum_{n\in \omega} |b_n-a_n| : A\subseteq \bigcup_{n\in \omega} (a_n,b_n]_{f}\right\}, & \text{si} \ A \ \text{es cubierto por} \ f, \\
\infty, & \text{si} \ A \ \text{no es cubierto por} \ f.
\end{cases}$$
    \end{examples}
\end{frame}

%------------------------------------------------
\subsection{Medidas exteriores de subconjuntos numerables de $\mathbb R$}

\begin{frame}{Medidas exteriores de subconjuntos numerables de $\mathbb R$}
    \begin{block}{Proposición}
        Si $f \in \Sel_2( \mathbb R)$, entonces $\lambda_f^{*}$ es una medida exterior para $\mathbb R$.
    \end{block}

    \begin{block}{Teorema}
        Si $f \in \Sel_2 (\mathbb R)$ y $x \in \mathbb R$, entonces $\lambda_f^(\{x\}) \in \{0,\infty\}$. Más aún, $\lambda_f^(\{x\}) = \infty$ si y sólo si $(\gets,x)_f = \emptyset$.
    \end{block}

    \begin{block}{Corolario}
        Si $f \in \Sel_2( \mathbb R)$ y $E \in [\mathbb R]^{\leq \omega}$, entonces $\lambda_f^(E) \in \{ 0, \infty\}$. Más aún, $\lambda_f^(E) = \infty$ si y sólo si existe $x\in E$ con $(\gets,x)_f = \emptyset$.
    \end{block}
\end{frame}

%------------------------------------------------
\subsection{Características inverosímiles}

\begin{frame}{Características inverosímiles}
    \begin{alertblock}{Ejemplo}
        Existe $f \in \Sel_2( \mathbb R)$ de tal manera que $\lambda_f^*$ no es invariante bajo traslaciones. 
    \end{alertblock}

    \begin{alertblock}{Ejemplo}
        Si $E$ es un subconjunto acotado, entonces existen $f \in \Sel_2(\mathbb R)$ y $z \in \mathbb R$ con $E + z \in \mathcal{N}_f$ y $\lambda^(E) = \lambda_f^(E)$. 
    \end{alertblock}

    \begin{alertblock}{Ejemplo}
        Existe $f \in \Sel_2(\mathbb R)$ con $\mathcal{N}_f = P(\mathbb R)$. 
    \end{alertblock}
\end{frame}

\begin{frame}{Características inverosímiles}
    \begin{alertblock}{Ejemplo}
        Si $A$ es un subconjunto no vacío de $\mathbb{R}$, entonces existe $f \in \Sel_2(\mathbb R)$ con $\lambda_f^*(A) = \infty$.
    \end{alertblock}

    \begin{alertblock}{Ejemplo}
        Existe $f \in \Sel_2(\mathbb R)$ de tal manera que $\lambda_f^{*}((0,3]) = 2$.
    \end{alertblock}

    \begin{alertblock}{Ejemplo}
        Sea $A \in [\mathbb R]^{\mathfrak{c}}$. Si $a \in \mathbb R \setminus A$ y $b \in \mathbb R \setminus A$ cumplen $a < b$, entonces existe $f \in \Sel_2(\mathbb R)$ con $\lambda_f^*(A) = b -a$.
    \end{alertblock}
\end{frame}

%------------------------------------------------
\subsection{Equivalencias de la Hipótesis del Continuo}

\begin{frame}{Equivalencias de la Hipótesis del Continuo}
    \begin{block}{Teorema}
        Los siguientes enunciados son equivalentes.

\begin{enumerate}
\item \textsf{CH}.
\item Existe $f \in \Sel_2( \mathbb R)$ con $|(\gets, x)_f | \leq \omega$ para cada $x \in \mathbb R$.
\item Existe $f \in \Sel_2( \mathbb R)$ con $|(x, \to)_f | \leq \omega$ para cada $x \in \mathbb R$.
\item Existe $f \in \Sel_2( \mathbb R)$ que satisface $|(x, y]_f | \leq \omega$ para cualesquiera $x \in \mathbb R$ y $y \in \mathbb R$.
\item Existen $f \in \Sel_2( \mathbb R)$ y $D \in [\mathbb R]^{\mathfrak{c}}$ y $|(x, y]_f \cap D | \leq \omega$ para cualesquiera $x \in \mathbb R$ y $y \in \mathbb R$.
\item Existe $f \in \Sel_2( \mathbb R)$ con $\lambda_f^* = \mu$.
\end{enumerate}
    \end{block}
\end{frame}

\begin{frame}{Bibliografía}
    \begin{enumerate}
        \item C. Constantini, S. Levi, J. Pelant, \textit{Compactness and local compactness in hyperspaces}, Topol. Appl., \textbf{123} (2002), 573--608.
        \item S. García Ferreira, A. H. Tomita, \textit{A non-normal topology generated by a two-point selection}, Topol. Appl., \textbf{155} (2008), 1105--1110.
        \item V. Gutev, T. Nogura, \textit{A topology generated by selections}, Topol. Appl., \textbf{153} (2005), 900--911.
        \item  M. Hrušák, I. Martínez Ruiz, \textit{Spaces determined by selections}, Topol. Appl., \textbf{157} (2010), 1448--1453.
        \item E. Michael, \textit{Topologies on spaces of subsets}, Trans. Amer. Math. Soc., \textbf{71} (1951), 152–182.
    \end{enumerate}
\end{frame}

\begin{frame}{Muchas gracias a todos por venir}
    \begin{figure}[h!]
    \centering
    \includegraphics[width=0.3\textwidth]{gato.jpg}
    \label{fig:mi_imagen}
\end{figure}
\centering \Large Bendito seas, martes de titulación
\end{frame}

\end{document}